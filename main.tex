\documentclass{article}
\usepackage[utf8]{inputenc}

\title{CS diagrams}
\author{GATE Overflow}
\date{August 2022}
\usepackage[]{qrcode}
\usepackage{tikz}
\usepackage{karnaugh-map}
\usepackage{tikz-timing}[2009/05/15]
\usepackage{natbib}
\usepackage{graphicx}
\usepackage{rubikcube,rubikrotation,rubikpatterns}
\usepackage{circuitikz}
\usepackage{pgf}
\usepackage{makecell}
\usepackage{verbatim}
\usepackage{ulem,tikz}
\usepackage{amsmath}
\usepackage{pifont}
\usepackage{amssymb}
\usetikzlibrary{patterns}
\usepackage{pgf-pie}
\usepackage{xcolor}
\usepackage[margin=0.5in]{geometry}
\usepackage{pgfplots}
\usepgfplotslibrary{fillbetween}
\usepackage[active,tightpage]{preview}
\PreviewEnvironment{tikzpicture}
%\PreviewEnvironment{qrcode}
%\usepackage{tikz-qtree}
\usepackage{tkz-euclide}
\usetikzlibrary{decorations.pathmorphing,backgrounds,fit,petri}
\usetikzlibrary{positioning,arrows.meta}
%\usepackage{tikz-qtree}
\usetikzlibrary{trees,calc,arrows.meta,positioning,decorations.pathreplacing,bending}
\usetikzlibrary{arrows}
\usetikzlibrary{shapes.geometric}
\usepackage{pgfplots}
\newcommand{\dname}{$D_0$}
\newcommand{\qname}{$Q_0$}
 \newcommand{\inpa}{$J_0$}
  \newcommand{\inpb}{$K_0$}
  \newcommand{\outa}{$Q_0$}
  \newcommand{\outb}{$\overline{Q_0}$}
\newcommand{\midarrow}{\tikz \draw[-triangle 90] (0,0) -- +(.1,0);}
\newcommand{\implyarrow}{%
  \mathrel{\raisebox{1.3ex}{\rotatebox[origin=c]{90}{\mathhexbox37F}}}}

\usetikzlibrary{trees,calc,quotes} % this is to allow the fork right path


\setlength\PreviewBorder{5pt}%
\usetikzlibrary{decorations.pathmorphing,backgrounds,fit,petri}
\usetikzlibrary{arrows,shapes.gates.logic.US,shapes.gates.logic.IEC,calc,automata, patterns, angles, quotes}
\usetikzlibrary{decorations.pathreplacing,decorations.markings}
\usetikzlibrary{positioning,arrows.meta}

\usetikzlibrary{intersections}
%\usepackage{tkz-euclide}
\tikzset{
  multiplexer/.pic ={
    \coordinate (-out) at (.8,0);
    \coordinate (-in-up) at (-0.8,.5);
    \coordinate (-in-down) at (-0.8,-.5);
     \coordinate (-select) at (0,-1);
    \draw[pic actions] (-.5,1.25) -- ++(1,-.5) -- ++(0,-1.5) -- ++(-1,-.5) --cycle;
    \draw[pic actions] (.5,0) -- (-out) (-in-up) -- +(.3,0) (-in-down) -- +(.3,0);
    \node{ \tikzpictext};
  }
}
\def \setA{ (0,0) circle (1cm) }
\def \setB{ (1.5,0) circle (1cm) }
\def \setC{ (60:1.5) circle (1cm) }
\def \setU{ (-2, -1.5) rectangle (3.5, 2.75) }
%\pgfplotsset{compat=1.11}
%\usepgfplotslibrary{fillbetween}
\begin{comment}
:Title: Timing diagram with the tikz-timing package
:Slug: tikz-timing

This example shows how to make a timing diagram with the `tikz-timing package`_. 
This timing diagram was used by the package author in a recent work and 
shows several clock and pulse signals. The relationship between the clock and 
signal edges is shown using horizontal lines.

Timing diagrams like this can be done using the ``tikztimingtable`` 
environment which has the same syntax as a ``tabular`` environment with two 
columns. The first column holds the signal name, the second one the timing 
characters. See the `package manual`_  for detailed information about them.

.. _tikz-timing package: http://www.ctan.org/tex-archive/help/Catalogue/entries/tikz-timing.html
.. _package manual: http://www.ctan.org/tex-archive/graphics/pgf/contrib/tikz-timing/tikz-timing.pdf

\end{comment}

\pagestyle{empty}
\def\degr{${}^\circ$}

\newif\ifcomment

% Activate the following line to compile document with comments:
\commenttrue
\newcommand{\implicantcostats}[4][0]{
    \draw[rounded corners=3pt, fill=#4, opacity=0.3] ($(rf.east |- #2.north)+(90:#1)$)-| ($(#2.east)+(0:#1)$) |- ($(rf.east |- #3.south)+(-90:#1)$);
    \draw[rounded corners=3pt, fill=#4, opacity=0.3] ($(cf.west |- #2.north)+(90:#1)$) -| ($(#3.west)+(180:#1)$) |- ($(cf.west |- #3.south)+(-90:#1)$);
}

%group top-bottom borders
%#1 - Optional. Space between node and grouping line. Default=0
%#2 - top left node
%#3 - bottom right node
%#4 - filling color
\newcommand{\implicantdaltbaix}[4][0]{
    \draw[rounded corners=3pt, fill=#4, opacity=0.3] ($(cf.south -| #2.west)+(180:#1)$) |- ($(#2.south)+(-90:#1)$) -| ($(cf.south -| #3.east)+(0:#1)$);
    \draw[rounded corners=3pt, fill=#4, opacity=0.3] ($(rf.north -| #2.west)+(180:#1)$) |- ($(#3.north)+(90:#1)$) -| ($(rf.north -| #3.east)+(0:#1)$);
}

%group corners
%#1 - Optional. Space between node and grouping line. Default=0
%#2 - filling color
\newcommand{\implicantcantons}[2][0]{
    \draw[rounded corners=3pt, opacity=.3] ($(rf.east |- 0.south)+(-90:#1)$) -| ($(0.east |- cf.south)+(0:#1)$);
    \draw[rounded corners=3pt, opacity=.3] ($(rf.east |- 8.north)+(90:#1)$) -| ($(8.east |- rf.north)+(0:#1)$);
    \draw[rounded corners=3pt, opacity=.3] ($(cf.west |- 2.south)+(-90:#1)$) -| ($(2.west |- cf.south)+(180:#1)$);
    \draw[rounded corners=3pt, opacity=.3] ($(cf.west |- 10.north)+(90:#1)$) -| ($(10.west |- rf.north)+(180:#1)$);
    \fill[rounded corners=3pt, fill=#2, opacity=.3] ($(rf.east |- 0.south)+(-90:#1)$) -|  ($(0.east |- cf.south)+(0:#1)$) [sharp corners] ($(rf.east |- 0.south)+(-90:#1)$) |-  ($(0.east |- cf.south)+(0:#1)$) ;
    \fill[rounded corners=3pt, fill=#2, opacity=.3] ($(rf.east |- 8.north)+(90:#1)$) -| ($(8.east |- rf.north)+(0:#1)$) [sharp corners] ($(rf.east |- 8.north)+(90:#1)$) |- ($(8.east |- rf.north)+(0:#1)$) ;
    \fill[rounded corners=3pt, fill=#2, opacity=.3] ($(cf.west |- 2.south)+(-90:#1)$) -| ($(2.west |- cf.south)+(180:#1)$) [sharp corners]($(cf.west |- 2.south)+(-90:#1)$) |- ($(2.west |- cf.south)+(180:#1)$) ;
    \fill[rounded corners=3pt, fill=#2, opacity=.3] ($(cf.west |- 10.north)+(90:#1)$) -| ($(10.west |- rf.north)+(180:#1)$) [sharp corners] ($(cf.west |- 10.north)+(90:#1)$) |- ($(10.west |- rf.north)+(180:#1)$) ;
}

\usetikzlibrary{matrix,calc}

%isolated term
%#1 - Optional. Space between node and grouping line. Default=0
%#2 - node
%#3 - filling color
\newcommand{\implicantsol}[3][0]{
    \draw[rounded corners=3pt, fill=#3, opacity=0.3] ($(#2.north west)+(135:#1)$) rectangle ($(#2.south east)+(-45:#1)$);

    }

%Defines 8 or 16 values (0,1,X)
\newcommand{\contingutz}[1]{%
\foreach \x [count=\xi from 0]  in {#1}
     \path (\xi) node {\x};
}

%Places 1 in listed positions
\newcommand{\mintermsz}[1]{%
    \foreach \x in {#1}
        \path (\x) node {1};
}

%Places 0 in listed positions
\newcommand{\maxtermsz}[1]{%
    \foreach \x in {#1}
        \path (\x) node {0};
}

%Places X in listed positions
\newcommand{\indeterminatsz}[1]{%
    \foreach \x in {#1}
        \path (\x) node {X};
}



\newcommand{\stargraph}[2]{\begin{tikzpicture}
    \node[circle,draw=black,minimum size=0.75cm] at (360:0mm) (center) {0};
    \foreach \n in {1,...,#1}{
        \node[circle,draw=black,minimum size=0.75cm] at ({-(\n-1)*240/#1}:#2cm) (n\n) {\n};
        \draw (center)->(n\n);
        %\node at (0,-#2*1.5) ;%{$K_{1,#1}$}; % delete line to remove label
    }
     \node[circle,draw=black,minimum size=0.75cm] at ({-(#1)*240/(#1)}:#2cm) (n887) {$\ldots$};
        \draw[densely dotted] (center)->(n887);
        \node[circle,draw=black] at ({-(#1+1)*240/(#1)}:#2cm) (n888) {$\ldots$};
        \draw[densely dotted] (center)->(n888);
        \node[circle,draw=black,minimum size=0.75cm] at ({1*240/#1}:#2cm) (n889) {\tiny{n-1}};
        \draw (center)->(n889);
\end{tikzpicture}}

\usepackage{pgfplotstable}
\pgfplotsset{compat=1.11,
        /pgfplots/ybar legend/.style={
        /pgfplots/legend image code/.code={%
        %\draw[##1,/tikz/.cd,yshift=-0.25em]
                %(0cm,0cm) rectangle (3pt,0.8em);},
        \draw[##1,/tikz/.cd,bar width=3pt,yshift=-0.2em,bar shift=0pt]
                plot coordinates {(0cm,0.8em)};},
},
}


\begin{document}

\maketitle

%% https://gateoverflow.in/39667/gate-cse-2016-set-1-question-10#41407

\begin{tikzpicture}[line width=.7pt,scale=0.7]
\begin{scope}
  \draw[fill=gray!20] (0,0)node[xshift=0.2cm,yshift=0.75cm]{$\textbf{[0]}$} rectangle(1,0.5);
  \draw[fill=gray!20] (1,0)node[xshift=0.25cm,yshift=0.75cm]{$\textbf{[1]}$} rectangle(2,0.5);
  \draw[fill=gray!20] (2,0)node[xshift=.3cm,yshift=0.75cm]{$\textbf{[2]}$} rectangle(3,0.5);
  \draw[fill=gray!20] (3,0)node[xshift=0.35cm,yshift=0.75cm]{$\textbf{[3]}$} rectangle(4,0.5);
  \draw[fill=gray!20] (4,0)node[xshift=0.4cm,yshift=0.75cm]{$\textbf{[4]}$} rectangle(5,0.5);
  \draw[fill=gray!20] (5,0)node[xshift=0.45cm,yshift=0.75cm]{$\textbf{[5]}$} rectangle(6,0.5);
  
  \path (0.5,0.25)node{$4$} (1.5,.25)node{$5$} (2.5,.25)node{$3$} (3.5,.25)node{$7$} (4.5,.25)node{$9$} (5.5,.25)node{$6$} ;
  
  %\draw[->,>=latex,thick] (-1.5,0.25) -- (-0.5,0.25);
  %\draw[->,>=latex,thick] (6.5,0.25) -- (7.5,0.25);
  \draw[->,>=latex,thick] (0.5,-0.2) -- (0.5,-0.7);
  \draw[->,>=latex,thick] (6.5,-0.2) -- (6.5,-0.7);
  
  \node[xshift=0.5cm,yshift=-0.8cm]{$\textbf{Front}$};
  \node[xshift=4.5cm,yshift=-0.8cm]{$\textbf{Rear}$};
\end{scope}

\begin{scope}[yshift=-3.5cm]
  \draw[fill=gray!20] (0,0)node[xshift=0.2cm,yshift=0.75cm]{$\textbf{[0]}$} rectangle(1,0.5);
  \draw[fill=gray!20] (1,0)node[xshift=0.25cm,yshift=0.75cm]{$\textbf{[1]}$} rectangle(2,0.5);
  \draw[fill=gray!20] (2,0)node[xshift=.30cm,yshift=0.75cm]{$\textbf{[2]}$} rectangle(3,0.5);
  \draw[fill=gray!20] (3,0)node[xshift=0.35cm,yshift=0.75cm]{$\textbf{[3]}$} rectangle(4,0.5);
  \draw[fill=gray!20] (4,0)node[xshift=0.40cm,yshift=0.75cm]{$\textbf{[4]}$} rectangle(5,0.5);
  \draw[fill=gray!20] (5,0)node[xshift=0.45cm,yshift=0.75cm]{$\textbf{[5]}$} rectangle(6,0.5);
  
  \path (0.5,0.25)node{5} (1.5,.25)node{$3$} (2.5,.25)node{$7$} (3.5,.25)node{$9$} (4.5,.25)node{$6$} (5.5,.25)node{} ;
  
 % \draw[->,>=latex,thick] (-1.5,0.25) -- (-0.5,0.25);
 % \draw[->,>=latex,thick] (6.5,0.25) -- (7.5,0.25);
  \draw[->,>=latex,thick] (0.5,-0.2) -- (0.5,-0.7);
  \draw[->,>=latex,thick] (5.5,-0.2) -- (5.5,-0.7);
  
  \node[xshift=0.5cm,yshift=-0.8cm]{$\textbf{Front}$};
  \node[xshift=4cm,yshift=-0.8cm]{$\textbf{Rear}$};
  \node[xshift=-2cm,yshift=1cm]{$\underline{\textbf{DEQUEUE}}$};
\end{scope}
\end{tikzpicture}

\begin{tikzpicture}[line width=.7pt,scale=0.7]
\begin{scope}
  \draw[fill=gray!20] (0,0)node[xshift=0.2cm,yshift=0.75cm]{$\textbf{[0]}$} rectangle(1,0.5);
  \draw[fill=gray!20] (1,0)node[xshift=0.25cm,yshift=0.75cm]{$\textbf{[1]}$} rectangle(2,0.5);
  \draw[fill=gray!20] (2,0)node[xshift=.3cm,yshift=0.75cm]{$\textbf{[2]}$} rectangle(3,0.5);
  \draw[fill=gray!20] (3,0)node[xshift=0.35cm,yshift=0.75cm]{$\textbf{[3]}$} rectangle(4,0.5);
  \draw[fill=gray!20] (4,0)node[xshift=0.4cm,yshift=0.75cm]{$\textbf{[4]}$} rectangle(5,0.5);
  \draw[fill=gray!20] (5,0)node[xshift=0.45cm,yshift=0.75cm]{$\textbf{[5]}$} rectangle(6,0.5);
  
  \path (0.5,0.25)node{$4$} (1.5,.25)node{$5$} (2.5,.25)node{$3$} (3.5,.25)node{$7$} (4.5,.25)node{$9$} (5.5,.25)node{$6$} ;
  
  %\draw[->,>=latex,thick] (-1.5,0.25) -- (-0.5,0.25);
  %\draw[->,>=latex,thick] (6.5,0.25) -- (7.5,0.25);
  \draw[->,>=latex,thick] (0.5,-0.2) -- (0.5,-0.7);
  \draw[->,>=latex,thick] (5.5,-0.2) -- (5.5,-0.7);
  
  \node[xshift=0.5cm,yshift=-0.8cm]{$\textbf{Front}$};
  \node[xshift=4cm,yshift=-0.8cm]{$\textbf{Rear}$};
\end{scope}

\begin{scope}[yshift=-4.5cm]
  \draw[fill=gray!20] (0,0)node[xshift=0.20cm,yshift=0.75cm]{$\textbf{[0]}$} rectangle(1,0.5);
  \draw[fill=gray!20] (1,0)node[xshift=0.25cm,yshift=0.75cm]{$\textbf{[1]}$} rectangle(2,0.5);
  \draw[fill=gray!20] (2,0)node[xshift=.30cm,yshift=0.75cm]{$\textbf{[2]}$} rectangle(3,0.5);
  \draw[fill=gray!20] (3,0)node[xshift=0.35cm,yshift=0.75cm]{$\textbf{[3]}$} rectangle(4,0.5);
  \draw[fill=gray!20] (4,0)node[xshift=0.40cm,yshift=0.75cm]{$\textbf{[4]}$} rectangle(5,0.5);
  \draw[fill=gray!20] (5,0)node[xshift=0.45cm,yshift=0.75cm]{$\textbf{[5]}$} rectangle(6,0.5);
  
  \path (0.5,0.25)node{${}$} (1.5,.25)node{$5$} (2.5,.25)node{$3$} (3.5,.25)node{$7$} (4.5,.25)node{$9$} (5.5,.25)node{$6$} ;
  
  \draw[->,>=latex,thick] (-1.5,0.25) -- (-0.5,0.25);
  \draw[->,>=latex,thick] (6.5,0.25) --  (7.5,0.25);
  \draw[->,>=latex,thick] (1.5,-0.2) --  (1.5,-0.7);
  \draw[->,>=latex,thick] (5.5,-0.2) --  (5.5,-0.7);
  
  \node[xshift=4cm,yshift=-0.8cm]{$\textbf{Rear}$};
  \node[xshift=1cm,yshift=-0.8cm]{$\textbf{Front}$};
  \node[xshift=-2cm,yshift=1cm]{$\underline{\textbf{DEQUEUE}_{\text{opt}}} $};
  
  \draw[dashed](-5.3,-1.5) rectangle (8.5,2);
  
  
  
\end{scope}
  
\end{tikzpicture}

%https://gateoverflow.in/39716/gate-cse-2016-set-1-question-48?show=40194#a40194
 \tikzset{
    rubberduck/.style={
        draw=black!90,
        fill = white,
        shape=isosceles triangle,
        minimum height=5.5cm,
        minimum width=2cm,
        shape border rotate=#1,
        isosceles triangle stretches,
        inner sep=0pt,ultra thick
    },
    rubber/.style={rubberduck=+120},
    ducky/.style={rubberduck=-90}}
 
 
 \begin{tikzpicture}[scale=0.6,transform shape]
      \foreach \x in {10,20,25,30,35,40,45,50,55,60,65,70,75,80,85,90,95,100}
        \pgfmathsetmacro{\tmp}{0.1*\x}
        \draw[color=black!40,ultra thick] (0,0) circle (\tmp cm); 
        \draw (0,0) node[circle,inner sep=1.5pt,fill,ultra thick] {} circle [radius=10];
        
 
    
 \node[draw,circle,minimum size=4cm,ultra thick,fill = gray!30] at (0,0){};
 \node[draw,circle,minimum size=2cm,ultra thick,fill=white] at (0,0){};
       
       
 \draw[-,line width = 1mm](0,0.5) -- (0,-.5);
           \draw[-,line width = 1mm](0.5,0) -- (-0.5,0);
           
\node[draw,circle,minimum size=1cm,ultra thick] at (-1.5,0){};
\node[draw,circle,minimum size=0.8cm,ultra thick] at (-1.5,0){};
            
\node[draw,circle,minimum size=1cm,ultra thick] at (1.5,0){};
\node[draw,circle,minimum size=0.8cm,ultra thick] at (1.5,0){};
            
 \node[draw,circle,minimum size=1cm,ultra thick] at (0,1.5){};
\node[draw,circle,minimum size=0.8cm,ultra thick] at (0,1.5){};
            
\node[draw,circle,minimum size=1cm,ultra thick] at (0,-1.5){};
\node[draw,circle,minimum size=0.8cm,ultra thick] at (0,-1.5){};
     
\draw[draw=black,ultra thick] (-1.3,-5.8) rectangle ++(1,0.9);

%R2
\draw[draw=black,ultra thick,color = black!80,rotate=-24.5,fill = white] (-9.8,-6.8) rectangle ++(6,2.5);
%R1
\draw[draw=black,ultra thick,color = black!80,rotate=-24.5] (-5.8,-6.6) rectangle ++(2,2);
%rr
\draw[draw=black,ultra thick,color = black!80,rotate=-24.5,rounded corners=20pt,fill=white] (-12.3,-7.1)  rectangle ++(3,3);

     \draw (-6.3,-3.7) node[circle,inner sep=1.5pt,fill,ultra thick] {} circle [radius=0.15];
      \draw (-5.9,-2.8) node[circle,inner sep=1.5pt,fill,ultra thick] {} circle [radius=0.15];
    
\draw (-7.5,-3.3) node[circle,inner sep=1.5pt,fill,ultra thick] {} circle [radius=0.15];
      \draw (-7.1,-2.4) node[circle,inner sep=1.5pt,fill,ultra thick] {} circle [radius=0.15];
        \node[rubber,rotate=-110] at (-5,-3.8){};
   %%%anular   
   \draw[fill=gray!50] (-12.2,-0.6) node[circle,inner sep=1.5pt,fill,ultra thick,] {} circle [radius=1];
\draw[fill=white] (-12.2,-0.6) node[circle,inner sep=1.5pt,ultra thick] {$\bigoplus$} circle [radius=0.8];

\node[text width=2.5cm] at (-14,1.6) {\textbf{\Large{Rotary Actuator}}};
\node[text width=3cm] at (7,-9.5) {\textbf{\Large{Hard Disk}}};

\node[text width=2.5cm] at (-10.5,-6) {\textbf{\Large{Suspension Arm}}};

\node[text width=2.5cm] at (-1,-11) {\textbf{\Large{Slider}}};

\draw[->,ultra thick] (-1.5,-10.5) -- (-0.9,-5.5);
\draw[->,ultra thick] (-10,-5.5) -- (-4.9,-4.5);


 \path [draw=black,rotate=90,<->, ultra thick,line width = 0.8mm ] (0.65,12.2) arc (5:185:1.3cm);
   

      
  \end{tikzpicture}




% https://gateoverflow.in/39669/gate-cse-2016-set-1-question-11


\begin{tikzpicture}[shorten >=1pt,node distance=2.5cm,on grid,auto,scale=0.7,transform shape,font=\large]


    \node[state] (q_0)  {$a$};
    \node[state] (q_1) [above right of=q_0]  {$b$};
    \node[state] (q_2) [right of=q_1] {$c$};
    \node[state] (q_3) [below right of=q_0] {$d$};
    \node[state] (q_4) [right of=q_3] {$e$};
    \node[state] (q_5) [below right of=q_2] {$f$};

     \path[->,-Latex]
       
        (q_0) edge node {} (q_1)
              edge node {} (q_3)
        (q_1) edge node {} (q_2)
        (q_2) edge node {} (q_5)
        (q_3) edge node {} (q_4)
        (q_4) edge node {} (q_5) ;
             
\end{tikzpicture}

%https://gateoverflow.in/39592/gate-cse-2016-set-2-question-50
\begin{tikzpicture}[scale = 0.6]
\node[ xshift=2.5cm,yshift=-0.9cm,font=\color{black}] {\textbf{Cache size (MB)}};
  \node[rotate=90,xshift=2cm,yshift=1cm,font=\color{black}] {\textbf{Miss rate (\%)}};
\draw[help lines, color=black!50, dashed] (0,0) grid (9,9);
\draw[thick] (0,0)--(9,0) node[right]{};
\draw[thick] (0,0)--(0,9) node[above]{};
\draw[black] (1,8) -- (2,6) -- (3,4);
\draw[black] (3,4) -- (4,3.5) -- (5,3) -- (6,2.5) -- (7,2) -- (8,1.5);
\filldraw (1,8) circle (1.5pt);
\filldraw (2,6) circle (1.5pt);
\filldraw (3,4) circle (1.5pt);
\filldraw (4,3.5) circle (1.5pt);
\filldraw (5,3) circle (1.5pt);
\filldraw (6,2.5) circle (1.5pt);
\filldraw (7,2) circle (1.5pt);
\filldraw (8,1.5) circle (1.5pt);
\node[below] at (0,0) {$0$};
\node[below] at (1,0) {$10$};
\node[below] at (2,0) {$20$};
\node[below] at (3,0) {$30$};
\node[below] at (4,0) {$40$};
\node[below] at (5,0) {$50$};
\node[below] at (6,0) {$60$};
\node[below] at (7,0) {$70$};
\node[below] at (8,0) {$80$};
\node[below] at (9,0) {$90$};

\node[left] at (0,1) {$10$};
\node[left] at (0,2) {$20$};
\node[left] at (0,3) {$30$};
\node[left] at (0,4) {$40$};
\node[left] at (0,5) {$50$};
\node[left] at (0,6) {$60$};
\node[left] at (0,7) {$70$};
\node[left] at (0,8) {$80$};
\node[left] at (0,9) {$90$};

\end{tikzpicture}

% https://gateoverflow.in/39585/gate-cse-2016-set-2-question-34?show=40018

\begin{tikzpicture}[scale=0.6,transform shape,node distance = 1.5cm,on grid,auto,
    every state/.style = {draw = black, fill = white},every initial by arrow/.style = {font = \Large,text = black,
        thick,-stealth}]

 \node[state] (q_0)  {$1$};
 \node[state] (q_1)  [below left of=q_0]  {$2$};
 \node[state] (q_2)  [below left of=q_1]  {$3$};
 \node[state] (q_3)  [below left of=q_2]  {$4$};
 \node[state] (q_4)  [below left of=q_3]  {$5$};
 \node[state] (q_5)  [below left of=q_4]  {$6$};
 \node[state] (q_6)  [below left of=q_5]  {$7$};
 \node[state] (q_7)  [below left of=q_6]  {$8$};
 \node[state] (q_8)  [below left of=q_7]  {$9$};
 \node[state,style={draw=white}] (q_9)  [below right of=q_7,xshift=-0.5cm,yshift=-1cm]  {1};
  \node[state,style={draw=white}] (q_9a)  [right of=q_9]  {3};
 \node[state,style={draw=white}] (q_10)  [right of=q_9a]  {$7$};
 \node[state,style={draw=white}] (q_11)  [right of=q_10]  {$15$};
 \node[state,style={draw=white}] (q_12)  [right of=q_11]  {$31$};
 \node[state,style={draw=white}] (q_13)  [right of=q_12]  {$63$};
 \node[state,style={draw=white}] (q_14)  [right of=q_13]  {$127$};
 \node[state,style={draw=white}] (q_15)  [right of=q_14]  {$255$};
 
 
 \path [-stealth, thick]
 
 (q_0) edge node {1} (q_1)
 (q_1) edge node {2} (q_2)
 (q_2) edge node {3} (q_3)
 (q_3) edge node {4} (q_4)
 (q_4) edge node {5} (q_5)
 (q_5) edge node {6} (q_6)
 (q_6) edge node {7} (q_7)
 (q_7) edge node {8} (q_8);
  \path [-stealth, dotted]
   (q_7)    edge node {} (q_9)
 (q_6) edge node {} (q_9a)
 (q_5) edge node {} (q_10)
 (q_4) edge node {} (q_11)
 (q_3) edge node {} (q_12)
 (q_2) edge node {} (q_13)
 (q_1) edge node {} (q_14)
 (q_0) edge node {} (q_15);
 
\end{tikzpicture}



%https://gateoverflow.in/108484/gate2016-aptitude-set-3-ga-6
\begin{tikzpicture}[scale=1]
  \tikzset{venn circle/.style={draw,circle,minimum width=3.5cm,fill=#1,opacity=0.6}}
   \begin{scope}[blend group = soft light]

  \node [venn circle = red!45!white,font=\Large] (A) at (0,0) {$15$};
  \node [venn circle = blue!40!white,font=\Large] (B) at (60:2.2cm) {$13$};
  \node [venn circle = cyan!45!white,font=\Large] (C) at (0:2.2cm) {$19$};
  \end{scope}
  \node[left,font=\Large] at (barycentric cs:A=1/2,B=1/2 ) {$44$}; 
  \node[below,font=\Large] at (barycentric cs:A=1/2,C=1/2 ) {$17$};   
  \node[right,font=\Large] at (barycentric cs:B=1/2,C=1/2 ) {$12$};   
  \node[below,font=\Large] at (barycentric cs:A=1/3,B=1/3,C=1/3 ){7};
  \node[above] at (110:3.8cm) {\Large{Read Books}};
  \node[above] at (-4:5.3cm) {\Large{Watch TV}};
  \node[above] at (10:-3.5cm) {\Large{Play Sports}};
  
\end{tikzpicture}  



%https://gateoverflow.in/39727/gate2016-1-40?show=69673#a69673
% diagram G
\begin{tikzpicture}[thick,node distance = 3cm,auto,scale = 0.1]
\node[state](1){};
\node[state](2)[right of = 1]{};
\node[state](3)[below of = 1]{};
\node[state](4)[below of = 2]{};

\path (1) edge node{$4$} (2);
\path (1) edge[left] node {$1$} (3);
\path (1) edge node {$3$} (4);
\path (2) edge node {$5$} (4);
\path (3) edge[below] node{$2$} (4);
\node[below] at (15,-38){G} ;

\end{tikzpicture}


%https://gateoverflow.in/39727/gate2016-1-40?show=69673#a69673
% diagram MST
\begin{tikzpicture}[thick,node distance = 3cm,auto,scale = 0.1]
\node[state](1){};
\node[state](2)[right of = 1]{};
\node[state](3)[below of = 1]{};
\node[state](4)[below of = 2]{};

\path (1) edge node{$4$} (2);
\path (1) edge[left] node {$1$} (3);
\path (3) edge[below] node{$2$} (4);
\node[below] at (15,-38){MST} ;

\end{tikzpicture}

%https://gateoverflow.in/39731/gate-cse-2016-set-1-question-38#44143
%https://gateoverflow.in/39731/gate2016-1-38?show=44143#a44143
\begin{tikzpicture}[<->,>=latex,,thick,node distance = 3cm,auto,scale = 0.1]
\node[state](1){$1$};
\node[state](2)[below left of = 1]{$2$};
\node[state](3)[below right of = 1]{$3$};
\node[state](4)[below right of = 2]{$4$};

\path (1) edge[above,pos = 0.6] node{$2$} (2);
\path (1) edge[above,pos = 0.6] node{$8$} (3);
\path (2) edge[left,pos = 0.6] node{$8$} (4);
\path (4) edge[right,pos = 0.4] node{$x$} (3);
\path (1) edge[left=30,pos = 0.5,bend right=10] node{$5$} (4);
\path (2) edge[below=30,pos = 0.5,bend right=10] node{$5$} (3);

\end{tikzpicture}



%%%%%%%%https://gateoverflow.in/39581/gate2016-2-39

\usetikzlibrary{shapes,arrows}

% Define block styles
\tikzstyle{decision} = [diamond, draw,  
    text width=4.5em, text badly centered, node distance=3cm, inner sep=3pt]
\tikzstyle{block} = [rectangle, draw,  
    text width=7em, text centered, minimum height=2em,inner sep=4pt]
\tikzstyle{line} = [draw, -latex']
\tikzstyle{cloud} = [draw, rounded rectangle, node distance=3cm,
    minimum height=2em,inner sep=4pt]
    
\begin{tikzpicture}[node distance = 2cm, auto,transform shape,scale=0.4,font=\huge]
    % Place nodes
   
    \node [cloud,] (init) {Start};
    \node [block, below of=init] (1) { A(n/2)};
   
    \node [decision, below of=1,node distance = 2.5cm] (decide) {};
     \node [block, left of=decide, node distance=3cm] (2) {A(n/2)};
       \node [cloud,left of = 2,node distance=3.5cm] (return1) {Return};
    \node [block, right of = decide, node distance=3cm] (3) {A(n/2)};
    \node [block, right of = 3, node distance=3.5cm] (4) {A(n/2)};
       \node [decision, right of=4] (decide1) {};
          \node [cloud,right  of = decide1 ] (return2) {Return};
           \node [block, below of = decide1, node distance=2.5cm] (5) {A(n/2)};
           
       \node [decision, below of=5,node distance = 2.5cm] (decide3) {};
       \node [cloud,right  of = decide3,node distance=3cm ] (return3) {Return};
       \node [block, left of = decide3, node distance=3cm] (6) {A(n/2)};
         \node [cloud,left  of = 6,node distance=3.5cm ] (return4) {Return};
       
    % Draw edges
   \path [line] (init) -- (1);
   \path [line] (1) -- (decide);
  \path [line] (decide) -- (2);
   \path [line] (2) --(return1);
 \path [line] (decide) -- (3);
  \path [line] (3) --(4);
   \path [line] (4) -- (decide1);
  \path [line] (decide1) -- (return2);
  
  \path [line] (decide1) -- (5);
  
  \path [line] (5) -- (decide3);
  
  \path [line] (decide3) -- (6);
    \path [line] (6) -- (return4);
      \path [line] (decide3) -- (return3);
 %   \path [line,dashed] (system) |- (evaluate);
\end{tikzpicture}

%%%%%%%%%https://gateoverflow.in/39727/gate2016-1-40?show=78456#a78456
\usetikzlibrary{shapes.multipart}
\usetikzlibrary{arrows}
\newcommand{\AxisRotator}[1][rotate=0]{%
    \tikz [x=0.25cm,y=0.60cm,line width=.2ex,-stealth,#1] \draw (0,0) arc (-150:150:1 and 1);%
}
\newcommand{\boundellipse}[3]% center, xdim, ydim
{(#1) ellipse (#2 and #3)
}

\begin{tikzpicture}[transform shape , scale=0.7]
\draw[fill = gray!30] \boundellipse{-1.5,-1.3}{2.5}{3};
\node [] at(-3,0.6){$S$};
 \node[draw,circle,minimum size= 0.65cm] at (0,0)(A) {$v$};
\node[draw,circle,minimum size= 0.65cm] at (0,-3) (B){};
\node[draw,circle,minimum size= 0.65cm] at (3,-3) (C){};
\node[draw,circle,minimum size= 0.65cm] at (3,0)(D) {$w$};
\node[draw,circle,minimum size= 0.65cm] at (-1.5,-1.5)(E) {};
\node[draw,circle,minimum size= 0.65cm] at (4.5,-1.5) (F){};

\node[draw,circle,minimum size= 0.65cm] at (6,-3)(G) {};
\node[draw,circle,minimum size= 0.65cm] at (-3,-3)(H) {};
\draw [- ,dashed](A) --  (B) --node[below]{$e$'} (C) -- (D);
\draw[-](D)--node[above]{$e$} (A) -- (E) -- (B);
\draw [-] (D) -- (F) -- (C);
\draw (F) -- (G);
\draw[- ] (E) -- (H);
%\draw[-,bend ] (E) -- (D);
  \path[every node/.style={font=\sffamily\small}]
  (E) edge[bend left=90] node [above,pos=0.6] {$e$''} (D);
  \node[rotate=180] at (1.5,-1.5) {\AxisRotator};
    \node[] at (1.5,-2.5) {Cycle $C$};

\end{tikzpicture}

%part 2
\begin{tikzpicture}[transform shape , scale=0.7]
\draw[fill = gray!30] \boundellipse{-1.5,-1.3}{2.5}{3};
\node [] at(-3,0.6){$S$};
 \node[draw,circle,minimum size= 0.65cm] at (0,0)(A) {$v$};
\node[draw,circle,minimum size= 0.65cm] at (0,-3) (B){};
\node[draw,circle,minimum size= 0.65cm] at (3,-3) (C){};
\node[draw,circle,minimum size= 0.65cm] at (3,0)(D) {$w$};
\node[draw,circle,minimum size= 0.65cm] at (-1.5,-1.5)(E) {};
\node[draw,circle,minimum size= 0.65cm] at (4.5,-1.5) (F){};

\node[draw,circle,minimum size= 0.65cm] at (6,-3)(G) {};
\node[draw,circle,minimum size= 0.65cm] at (-3,-3)(H) {};
\draw [- ,dashed](A) --  (B) --node[below]{$e$'} (C) -- (D);
\draw[-](D)--node[above]{$e$} node[midway]{$\bf{\times}$} (A) -- (E) -- (B);
\draw [-] (D) -- (F) -- (C);
\draw (F) -- (G);
\draw[- ] (E) -- (H);
%\draw[-,bend ] (E) -- (D);
  \path[every node/.style={font=\sffamily\small}]
  (E) edge[bend left=90] node [above,pos=0.6] {$e$''} (D);
  \node[rotate=180] at (1.5,-1.5) {\AxisRotator};
    \node[] at (1.5,-2.5) {Cycle $C$};
\node[draw,rectangle,rounded corners,yshift=-0.5cm] at (C.south)(T) {$e$' must replace $e$ in MST};
\end{tikzpicture}


%%%%%%%%https://gateoverflow.in/39667/gate2016-1-10?show=39680#a39680

\usetikzlibrary{shapes.multipart,chains, positioning}

\begin{tikzpicture}
    [
      double link/.style n args=2{% page 726
        on chain,
        rectangle split,
        rectangle split horizontal,
        rectangle split parts=2,
        rectangle split part fill={white,gray},
        draw,
        anchor=center,
        text height=1.5ex,
        node contents={#1\nodepart[fill={gray}]{two}#2},
      },
      start chain=going right,transform shape,scale=1
    ]
   \node []at (-0.5,1.4) {\bf{Circular linked-list:}};
    \node (c) [join={by <-}, double link={6}{300}];
    \node [join={by ->}, double link={9}{400}];
    \node (a) [join={by ->}, double link={12}{200}];
   
    \node [join={by ->}, double link={5}{700}];
    \node (a) [join={by ->}, double link={-10}{100}];
    \draw [->] (7.2,0) -- (7.7,0) -- (7.7,-0.8) -- (-.9,-0.8) -- (-.9,-0.8) -- (-.9,0)-- (-.6,0); 
    \node[above] at (-.4,0.2){100};
 \node[above] at (1.2,0.2){300};
  \node[above] at (2.8,0.2){400};
   \node[above] at (4.5,0.2){200};
    \node[above] at (6,0.2){700};
     \node[] at (6.5,-0.5){Tail Node};
     \node[draw]at (7,1) {100};
    % \draw[->] (7,0.7) --node [right] {head} (7,0.3);
     \draw[->] (7,0.7) --node [right] {Head Node} (7,0.3);
     \end{tikzpicture}

\begin{tikzpicture}
\draw 

(0, 5) node[circle, black,draw, fill = black](a){}
(2, 7) node[circle, black, draw, fill = black](b){}
(4, 5) node[circle, black, draw, fill = black](c){};



\draw[-] (a) -- node[above] {} (b);
\draw[-] (b) -- node[above] {} (c);
\draw[-] (c) -- node[above] {} (a);


\end{tikzpicture}

\begin{tikzpicture}
\draw 

(0, 5) node[circle, black,draw, fill = black](a){}
(2, 7) node[circle, black, draw, fill = black](b){}
(4, 5) node[circle, black, draw, fill = black](c){};



\draw[->, thick] (a) -- node[above] {} (b);
\draw[->, thick] (c) -- node[above] {} (b);
\draw[->,thick] (a) -- node[above] {} (c);


\end{tikzpicture}

%https://gateoverflow.in/110918/gate2016-aptitude-set-7-ga-8

  \def\firstcircle{(-210:1.75cm) circle (2.5cm)}
  \def\secondcircle{(-330:1.75cm) circle (2.5cm)}
   \def\thirdcircle{(-210:1.50cm) circle (1.75cm)}
  \def\fourthcircle{(-330:1.50cm) circle (1.75cm)}
 
    \begin{tikzpicture}
      \begin{scope}
 
    \clip \fourthcircle;
   
    \fill[cyan] \thirdcircle;
    
    
      \end{scope}
     
        
      \draw \firstcircle node[text=black,below] {$H$};
      \draw \secondcircle node [text=black,below left] {$E$};
       \draw \thirdcircle node[text=black ] at (-2,2.6) {$M$};
      \draw \fourthcircle node [text=black] at  (2,2.6){$BH$};
    
    \end{tikzpicture}
  
  
  %
\begin{tikzpicture}
  [
    scale=1,
    font=\footnotesize,
    level 1/.style={level distance=16mm,sibling distance=40mm},
    level 2/.style={level distance=12mm,sibling distance=20mm},
    level 3/.style={level distance=12mm,sibling distance=10mm},
    solid node/.style={circle,draw,inner sep=1,fill=black},
  ]

  \node(0)[solid node, label=A]{}
  child{node(1)[solid node,label=left:{$A$}]{}
    child{node[solid node,label=left:{$A$}]{}
      child{node[solid node,label=below:{$A$}]{} edge from parent node [left]{$\frac{1}{3}$}}
      child{node[solid node,label=below:{$B$}]{} edge from parent node [right]{$\frac{2}{3}$}}
      edge from parent node [left]{$\frac{1}{3}$}
    }
    child{node[solid node,label=right:{$B$}]{}
      child{node[solid node,label=below:{$A$}]{} edge from parent node [left]{$\frac{1}{2}$}}
      child{ node[solid node,label=below:{$B$}]{} edge from parent node [right]{$\frac{1}{2}$}}
      edge from parent node [right]{$\frac{2}{3}$}
    }
    edge from parent node [left, yshift=3]{$\frac{1}{3}$}
  }
  child{node(2)[solid node,label=right:{$B$}]{}
    child{node[solid node,label=left:{$A$}]{}
      child{node[solid node,label=below:{$A$}]{} edge from parent node [left]{$\frac{1}{3}$}}
      child{ node[solid node,label=below:{$B$}]{} edge from parent node [right]{$\frac{2}{3}$}}
      edge from parent node [left]{$\frac{1}{2}$}
    }
    child{node[solid node,label=right:{$B$}]{}
      child{node[solid node,label=below:{$A$}]{} edge from parent node [left]{$\frac{1}{2}$}}
      child{ node[solid node,label=below:{$B$}]{} edge from parent node [right]{$\frac{1}{2}$}}
      edge from parent node [right]{$\frac{1}{2}$}
    }
    edge from parent node [right, yshift=3]{$\frac{2}{3}$}
  };

\end{tikzpicture}


%https://gateoverflow.in/39732/gate-cse-2016-set-1-question-43
\begin{tikzpicture}[>=stealth',shorten >=1pt,node distance=3cm,on grid,auto]
   \node[state,initial,accepting,initial text=] (q_0)     {};
   \node[state] (q_1) [right of = q_0] {};
   \node[state,accepting] (q_2) [right of = q_1] {};
   
   \path[->]
   (q_0) edge [loop above,text width= 2cm]      node {$a,X/XX$ $a,Z/XZ$} ()
         edge                                   node {$b,X/\epsilon$} (q_1)
   (q_1) edge [loop above]       node {$b,X/\epsilon$} ()
         edge                   node {$\epsilon,Z/Z$} (q_2);
\end{tikzpicture}

%https://gateoverflow.in/39562/gate2016-2-16
\begin{tikzpicture}[>=stealth',shorten >=1pt,node distance=2cm,on grid,auto]
   \node[state,initial,initial text=] (q_0)     {};
   \node[state,accepting] (q_1) [right of = q_0] {};
  
   
   \path[->]
   (q_0) edge   node {0,1} (q_1)
   (q_1) edge [loop above] node {0,1} ();
   
\end{tikzpicture}

%https://gateoverflow.in/39598/gate2016-2-46#39695
\begin{tikzpicture}[>=stealth',shorten >=1pt,node distance=2cm,on grid,auto,thick]
    \node       (q)  [left of =A] {\textbf{G1}};
   \node[state] (A)     {D};
   \node[state] (B) [below of=A] {L};
   \node[state] (C) [below right of=B] {E}; 
   \node        (D) [below left of=A] {\textbf{int}};
   \node        (E) [below right of=A] {\textbf{;}};
   \node        (F) [below left of =B] {\textbf{id[}};
   \node        (G) [below left of =C] {\textbf{num][}};
   \node[state] (H) [below  of =C] {E};
   \node        (I) [below of=H] {\textbf{num]}};
  
 \path[->]
 (A) edge                node {} (E)
     edge                node {} (D)
     edge                node {} (B)
 (B) edge                node {} (F)
     edge                node {} (C)
 (C) edge                node {} (G)
     edge                node {} (H)
 (H) edge                node {} (I);
   
\end{tikzpicture}

%https://gateoverflow.in/39598/gate2016-2-46#39695
\begin{tikzpicture}[>=stealth',shorten >=1pt,node distance=2cm,on grid,auto,thick]
    \node       (q)  [left of =A] {\textbf{G2}};
   \node[state] (A)     {D};
   \node[state] (B) [below of=A] {L};
   \node[state] (C) [below right of=B] {E}; 
   \node        (D) [below left of=A] {\textbf{int}};
   \node        (E) [below right of=A] {\textbf{;}};
   \node        (F) [below left of =B] {\textbf{id}};
   \node [state]       (G) [below left of =C] {E};
   \node         (H) [below  of =C] {\textbf{[num]}};
   \node        (I) [below of=G] {\textbf{[num]}};
  
 \path[->]
 (A) edge                node {} (E)
     edge                node {} (D)
     edge                node {} (B)
 (B) edge                node {} (F)
     edge                node {} (C)
 (C) edge                node {} (G)
     edge                node {} (H)
 (G) edge                node {} (I);
   
\end{tikzpicture}


%https://gateoverflow.in/39535/gate-cse-2016-set-2-question-ga-10

\begin{tikzpicture}

\newcommand*{\XAxisMin}{-4}
\newcommand*{\XAxisMax}{4}
\newcommand*{\YAxisMin}{-2.5}
\newcommand*{\YAxisMax}{2.5}
 %axis on top=true,
\begin{axis}[
%axis equal=true,
ymajorgrids,
 width=0.6\linewidth,
  height=0.4\linewidth,
   line width=1.5,
    axis y line=center, axis x line=middle,
    xmin=\XAxisMin, xmax=\XAxisMax, ymin=\YAxisMin, ymax=\YAxisMax, 
    xtick=data,
ytick={-2.5,-2,-1.5,-1,-0.5,0,0.5,1,1.5,2,2.5},
xtick={-4,-3,-2,-1,0,1,2,3,4},
inner axis line style={-},
    xlabel={$x$},
     ylabel={$f(x)$},
] 

 \addplot[blue] coordinates
      {(-3,-2) (1,2) (3,0)};
     

\end{axis} 
\end{tikzpicture} 

%%https://gateoverflow.in/108289/gate2016-me-2-ga-5?show=216978#a216978
\begin{tikzpicture}[scale = 0.8]
\draw[] (0,2.75)-- (0,0) -- (3,0) -- (3,2.75) -- (1.5,5) -- (0,2.75);

\node[below] at (1.5,0){$a$};
\node[right] at (3,1.5){$a$};
\node[left] at (0,1.5){$a$};
\node[left] at (0.70,4){$a$};
\node[right] at (2.25,4){$a$};
\end{tikzpicture}






%https://gateoverflow.in/302808/gate-cse-2019-question-40#327311
\begin{tikzpicture}[level/.style={sibling distance=25mm/#1},level 2/.style={sibling distance=22mm},font = \sffamily,scale=0.4]
    \node [draw, circle,minimum size= 8mm]{7}
        child {node [draw, circle,minimum size= 8mm]{4}
            child {node [draw, circle,minimum size= 8mm]{1}}
            child {node [draw, circle,minimum size= 8mm]{3}}
        }
        child {node [draw, circle,minimum size= 8mm]{6}
            child {node [draw, circle,minimum size= 8mm]{2}}
            child {node [draw, circle,minimum size= 8mm]{5}}
        }
    ;
\end{tikzpicture}

%https://gateoverflow.in/302808/gate-cse-2019-question-40#327311
\begin{tikzpicture}[level/.style={sibling distance=55mm/#1},font = \sffamily,scale=0.4]    
    \node [draw, circle,minimum size= 8mm]{7}
        child {node [draw, circle,minimum size= 8mm]{4}
            child {node [draw, circle,minimum size= 8mm]{1}
                child[missing]{}
                child{node [draw, circle,minimum size=8mm]{3}
                    child{node [draw, circle,minimum size=8mm]{2}}
                    child[missing]{}
                }
            }
            child {node [draw, circle,minimum size= 8mm]{6}
                child{node [draw, circle,minimum size=8mm]{5}}
                child[missing]{}
            }
        }
        child[missing] {}
    ;
\end{tikzpicture}

%https://gateoverflow.in/302802/gate-cse-2019-question-46?show=323153
\begin{tikzpicture}[->,>=stealth,scale=0.8, level/.style={circle,draw,>=Stealth},
level 1/.style={sibling distance=40mm},level 2/.style={sibling distance=20mm},level 3/.style={sibling distance=12mm},minimum size=0.50cm,iv/.style={draw,rectangle,minimum size=15pt,inner sep=0pt,text=black},ev/.style={draw,circle,minimum size=15pt,inner sep=0pt,text=black,fill=yellow}]
\node[ev,name=A]{A}  
               child {node[ev]{B}
               child {node[ev]{D}
               child {node[ev]{H}}
               child {node[ev]{I}}
               }
               child {node[ev]{E}
               child {node[ev]{J}}
               child {node[ev]{K}}
               }
               }
               child {node[ev,name=C]{C}
               child {node[ev]{F}
               child {node[ev]{L}}
               child {node[ev]{M}}
               }
               child {node[ev,name=G]{G}
               child {node[ev]{N}}
               child {node[ev,name=O]{O}}
               }
};
%let \p1 = (A);
\draw[<-,>=latex,shorten >=2pt,node distance=2cm,on grid,auto,color=red] (A) -- (A -| 4.5,0);
\draw[<-,>=latex,shorten >=2pt,node distance=2cm,on grid,auto,color=red](C) -- (C -| 4.5,0);
\draw[<-,>=latex,shorten >=2pt,node distance=2cm,on grid,auto,color=red](G) -- (G-|4.5,0);
\draw[<-,>=latex,shorten >=2pt,node distance=2cm,on grid,auto,color=red](O) -- (O-|4.5,0);
\node [text width=2.5cm,align=left] at (A-|6,0.0) {Level $0$};
\node [text width=2.5cm,align=left] at (C-|6,0) {Level $1$};
\node [text width=2.5cm,align=left] at (G-|6,0) {Level $2$};
\node [text width=2.5cm,align=left] at (O-|6,0) {Level $3$};
\node at (-0.1,-5.2) {Depth of the tree (d) = $3$};
\end{tikzpicture}

%https://gateoverflow.in/302800/gate2019-48?show=303078#a303078

    \begin{tikzpicture}[
    >=stealth,scale=0.3,transform shape,
    bullet/.style={
        fill=black,
        circle,
        minimum width=1pt,
        inner sep=1pt
    },
    projection/.style={
        ->,
        thick,
        shorten <=2pt,
        shorten >=2pt
    },
    every fit/.style={
        ellipse,
        draw,
        inner sep=0pt
    },
        setA/.style={fill=red,circle,inner sep=4pt}
    ]
 
  

    \draw[fill=cyan!50] (0,2.5) ellipse (1.5cm and 6cm);
    \draw[fill=cyan!50] (7,2.5) ellipse (1.02cm and 2.2cm);
   \node[font=\Huge] at (0,9){$\Sigma^*$};
\node[font=\Huge] at (7,6) {${f_1,f_2,f_3,\ldots,f_{120}}$};
  \node[setA] (a) at (0,6.5) {};
    \node[setA,below of = a] (b) at (0,6) {};
      \node[setA,below of = b] (c) at (0,5) {};
        \node[setA,below of = c] (d) at (0,4) {};
          \node[setA,below of = d] (e) at (0,3) {};
            \node[setA,below of = e] (f) at (0,2) {}; 
            \node[setA,below of = f] (g) at (0,1) {};
              \node[setA,below of = g] (h) at (0,0) {}; 
            \node[setA,below of = h] (i) at (0,-1) {};

 \node[setA] (a) at (7,4.1) {};
    \node[setA,below of = a] (b) at (7,4) {};
      \node[setA,below of = b] (c) at (7,3) {};
 
    \draw[projection] (0.2,6.5) -- (7,4.1);
     \draw[projection] (0.2,5) -- (7,4.1);
      \draw[projection] (0.2,-1) -- (7,2);
 

   % \draw[projection] (0.3,2.5) -- (3.7,2.5);
   % \draw[projection] (4.3,2.5) -- (7.7,2.5);
    \end{tikzpicture}
 


  
 %%%%% ugcnet2-june2019 By Lakshman Patel %% 
%https://gateoverflow.in/316278/ugcnet-june-2019-ii-1?show=316349#a316349
\tikzset{state/.style={draw,circle}}
\begin{tikzpicture}[shorten >=0.01pt,node distance = 1.5cm,auto,scale = 0.5,transform shape]

\node[](1){5};
\node[](2)[right of = 1]{3};
\node[](3)[above of = 1]{15};
\node[](4)[above of = 2]{9};
\node[](5)[above of = 4]{45};
\node[](6)[right of = 4]{24};

\path (1) edge[] node {}(3);
\path (2) edge[] node {}(3);
\path (2) edge[] node {}(4);
\path (2) edge[] node {}(6);
\path (3) edge[] node {}(5);
\path (4) edge[] node {}(5);
\end{tikzpicture}


%https://gateoverflow.in/316188/ugcnet-june-2019-ii-91
\begin{tikzpicture}[cv/.style={minimum size = 18pt,inner sep=2pt,draw,circle},iv/.style={minimum size = 15pt,inner sep=1.5pt,draw,rectangle},level 1/.style={sibling distance=10cm},level 2/.style={sibling distance=5.3cm},level 3/.style={sibling distance=2.5cm},level 4/.style={sibling distance=1.2cm},transform shape,scale = 0.5]
\node[iv]{}
    child {node[cv]{}
          child {node[iv]{}
                 child {node[cv]{}
                       child {node[iv]{}}
                       child {node [iv]{}}}
                 child {node [cv]{}
                       child {node[iv]{}}
                       child {node [iv]{}}}}
          child {node [iv]{}
                child {node[cv]{}
                       child {node[iv]{}}
                       child {node [iv]{}}}
                 child {node [cv]{}
                       child {node[iv]{}}
                       child {node [iv]{}}}}}
    child {node [cv]{}
          child {node[iv]{}
                 child {node[cv]{}
                       child {node[iv]{}}
                       child {node [iv]{}}}
                 child {node [cv]{}
                       child {node[iv]{}}
                       child {node [iv]{}}}}
          child {node [iv]{}
                child {node[cv]{}
                       child {node[iv]{}}
                       child {node [iv]{}}}
                 child {node [cv]{}
                       child {node[iv]{}}
                       child {node [iv]{}}}}};
\node at (10,0) {$\textbf{MAX}$};
\node at (10,-1) {$\textbf{MIN}$};   
\node at (10,-2) {$\textbf{MAX}$};   
\node at (10,-3) {$\textbf{MIN}$};

\node at (-9.5,-6.5) {7}; 
\node at (-8.3,-6.5) {9};
\node at (-7,-6.5) {6};
\node at (-4.2,-6.5) {11};
\node at (-3,-6.5) {12};
\node at (0.5,-6.5) {5};
\node at (3,-6.5) {4}; 

\end{tikzpicture}

%https://gateoverflow.in/316204/ugcnet-june-2019-ii-75?show=317159#a317159
\begin{tikzpicture}[shorten >=0.01pt,->,>=latex,node distance = 2cm,auto,scale = 0.5,transform shape]

\node[state,initial,initial text = ](1){P};
\node[state](2)[right of = 1]{Q};
\node[state,accepting](3)[right of = 2]{R};

\path (1) edge[loop above] node {a}(1);
\path (1) edge[] node {b}(2);
\path (2) edge[] node {a,b}(3);

\end{tikzpicture}

%https://gateoverflow.in/316245/ugcnet-june-2019-ii-34?show=317133#a317133
\begin{tikzpicture}[scale = 0.5,thick,transform shape]
\draw[color=green!60!black,fill=green!25!white] (0,0) circle (10cm);
\draw[color=blue!60!black,fill = blue!30!white] (1,-0.2) circle (7.8cm);
\draw[color=red!60!black,fill = red!25!white] (1.7,-0.5) circle (5.5cm);
\draw[color=orange!60!black,fill = orange!30!white] (2.2,-0.8) circle (3.2cm);

\node[very thick] at (1.3,0.3){\Huge{\textbf{BCNF}}};
\node[very thick] at (0,3.2) {\Huge{\textbf{3 NF}}};
\node[very thick] at (-1.2,5.8) {\Huge{\textbf{2 NF}}};
\node[very thick] at (-2.5,8.2) {\Huge{\textbf{1 NF}}};

\node at (18,8) {\Huge{\textbf{A relation in BCNF is also in 3NF, 2NF, 1NF.}}};
\node at (18,5) {\Huge{\textbf{A relation in 3NF is also in 2NF, 1NF.}}};
\node at (18,2) {\Huge{\textbf{A relation in 2NF is also in 1NF.}}};
\end{tikzpicture}

%https://gateoverflow.in/313448/gate2019-ce-1-ga-10?show=314121#a314121 
\begin{tikzpicture}[scale=0.5]
\node[thick,draw,circle,color= cyan,minimum size= 0.5cm] at (-2,0) {Q} ;
\node[thick,draw,color= red,minimum size= 0.5cm] at (-4,0) {P} ;
\node[] at (-3.15,0) {=};
\node[] at (3,0) {S};
\node[] at (3.5,0) {=};
\node[] at (3.5,-4) {T};
\node[] at (-3.15,-4) {R};
\draw[<->,color=green] (-4,0.5) -- (-4,1) -- (3,1) -- (3,0.5);
\draw[->,thick,color=blue] (-3.15,-0.2) -- (-3.15,-3.5) ;
\draw[->,thick,color= blue] (3.5,-0.2) -- (3.5,-3.5) ;
\node[thick,draw,circle,color= cyan,minimum size= 0.5cm] at (-3,-5) {} ;
\node[thick,draw,color= red,minimum size= 0.5cm] at (-3,-6.6) {} ;
\draw[->] (-2.5,-5)--(-2,-5) node[right]{Female};
\draw[->] (-2.5,-6.5)--(-2,-6.5) node[right]{Male};
\end{tikzpicture}


%https://gateoverflow.in/302866/gate2019-ga-7
\begin{tikzpicture}[scale=0.6]
\draw[black,thick] (0,0) -- (-7,0);
\draw[black,thick] (0,0) -- (-3.5,-3);
\draw[black,thick] (-7,0) -- (-3.5,-3);
\draw [black,thick] (0,0) rectangle (-2,-6);
\draw [black, thick] (-1,-6) circle [radius=2];
\node at (-6,0.4) {Teachers};
\node at (2.2,-3) {Administrators};
\node at (-4.7,-6) {Researchers};
      
\node at (-3.5,-1.5) {$70$};
\node at (-1.3,-0.6) {$10$};
\node at (-1,-3) {$20$};
\node at (-1,-5.1) {$20$};
\node at (-1,-7) {$40$};

\end{tikzpicture}

 %
   % this is to allow the fork right path

%https://gateoverflow.in/302807/gate2019-41?show=303043#a303043

\usetikzlibrary{positioning}
\tikzset{
  gray box/.style={
    fill=white!20,
    draw=black,
    minimum width={2*#1ex},
    minimum height={3em},
  },
  annotation/.style={
    anchor=north,
  }
}
\begin{tikzpicture}[node distance=-0.5pt]
  \node [gray box=1] (p1) {\(P1\)}; ;

  \node [gray box=1, right=of p1] (p2) {\(P2\)};
  \node [gray box=3.5, right=of p2] (p3) {\(P1\)};
    \node [gray box=4.5, right=of p3] (p4) {};
 
 \node [annotation] at (p1.south west) {0};
  \node [annotation] at (p1.south east) {1};
 \node [annotation] at (p2.south east) {2};
  \node [annotation] at (p3.south east) {4};
%  \node [annotation] at (p4.south east) {};


\end{tikzpicture}
%https://gateoverflow.in/302798/gate2019-50?show=302798#q30279
 
  \begin{karnaugh-map}[4][4][1][][][baseline=(current bounding box.north),scale=.3]
        \minterms{0,2,8,10,7,5,13,15}
      
        \autoterms[0]
        \implicantedge{1}{3}{9}{11}
        \implicantedge{4}{12}{6}{14}
       % \implicant{4}{5}
        \node at (-0.25,4.7) {CD};           % row name
        \node at (-0.6,4.15) {AB};            % column name
        \draw[ultra thin] (0,4) -- (-1,4.75);  % diagonal line
    \end{karnaugh-map}
    
    
   

%https://gateoverflow.in/302801/gate2019-47?show=303770#a303770
\begin{figure}

\begin{tikzpicture}[scale=0.8]
\tiny
\tiny
\draw [->](0,0) -> (6.5,0);

\draw [->](0,0) -- (0,2);
\draw [thick] (1,0)--(1,0.4) -- (6,0.4);
\draw [dashed] (0,0.4)--(1,0.4);
\draw [thick] (6,0.4)--(6,0);

\node [below] at (0,0) {$0$};

\node [below] at (1,0) {$1$};

\node [below] at (6,0) {$6$};

\node [left] at (0,0.4) {$1 \over 5 $};
\end{tikzpicture}

\end{figure}


\begin{tikzpicture}[scale=0.8]

\tiny

\tiny

\draw[->] (0,0) -- (6.5,0);

\draw[->] (0,0) -- (0,2);
\draw[thick] (2,0.4) -- (2,0);

\draw [thick] (1,0)--(1,0.4) -- (6,0.4);

\draw [dashed] (0,0.4)--(1,0.4);

\draw [thick] (6,0.4)--(6,0);


\node [below] at (0,0) {$0$};
\node [below] at (2,0) {$2$};
\node [below] at (1,0) {$1$};

\node [below] at (6,0) {$6$};

\draw[fill=gray]  (2,0) -- (2,0.4) -- (6,0.4) -- (6,0) -- cycle;

\node [left] at (0,0.4) {$1 \over 5 $};

\end{tikzpicture}




%%%%%%%%%%https://gateoverflow.in/313438/gate2019-ce-1-ga-3?show=314085#a314085

\begin{tikzpicture}[thick]

\coordinate (O) at (4,0);
\coordinate (A) at (0,0);
\coordinate (B) at (4,4);
\draw (O)--(A)--(B)--cycle;

\tkzLabelAngle[pos = 0.35](A,O,B){}


\tkzLabelAngle[pos = 0.6](B,A,O){$45$}

\tkzLabelAngle[pos = 0.5](O,B,A){}
\draw[-] (4,2) -- (4,7.5);
\draw[-] (0,0) -- (-2,0);
\draw[-] (4,0) -- (6,0);
\draw[->] (4,7) -- (5,7);
\node[right] at (5,7){Pole};
\draw[->] (2.5,2.5) -- (2.5,3.5);
\node[above] at (2.5,3.5){Ladder};

\draw[<-](6,0) -- (6,3.5);
\node[above] at (6,3.5){h};
\draw[->](6,4)-- (6,7.5);


\draw[<-](0,-0.5) -- (1.5,-0.5);
\node[above] at (1.6,-.5){6m};
\draw[->](1.8,-0.5)-- (4,-0.5);


\draw[<-](4.8,0) -- (4.8,1.5);
\node[above] at (4.8,1.5){$\frac{h}{5}$};
\draw[->](4.8,2.1)-- (4.8,4);




\draw[->] (-1.6,0) |- (-.6,-1);
\node[right] at (-.6,-1){Horizontal Ground };

  \draw (1,0) arc (0:45:1);
%   \begin{axis}
 
%   \addplot [thick,color=blue,,mark=o,fill=blue, 
%                     fill opacity=0.05] coordinates {
%           (-2,0)
%             (-2, -.5)
%             (6, -.5)
%             (6,0) };
% \end{axis}

\end{tikzpicture}

%https://gateoverflow.in/302818/gate2019-30?show=302818#q302818

\begin{tikzpicture}[label distance=1mm]
\node (A) at (0, 0)  {\small{$f_1$}};
\node (B) at (0, -0.36) {\small{$f_2$}};
\node (C) at (0, -1.8) {\small{$f_3$}};
\node (D) at (5.5, -1.6) {\small{$f$}};
\node[and gate US, draw, anchor=input 1] at ($(A) + (1.0, 0.0)$) (and1) {AND};
\node[xor gate US, draw, anchor=input 1] at ($(C) + (3.2, 0.4)$) (xor1) {XOR};
\draw (A) -- (and1.input 1);
\draw (B) -- (and1.input 2);
\draw (and1.output) -- ([xshift=0.5cm]and1.output) |- (xor1.input 1);
\draw (C) -- (xor1.input 2);
\draw (xor1.output) -- ([xshift=0.8cm]xor1.output);
\end{tikzpicture}


%%https://gateoverflow.in/313528/gate2019-ec-ga-7
%first image
\begin{tikzpicture}[scale = 0.8,font = \bfseries]
\begin{axis}[
ybar=8pt, %space between adjacent bars is 0
enlarge x limits=0.5,
enlarge y limits=false,
legend style={at={(0.5,-0.15),ybar=area},
anchor=north,legend columns=-1.15},
title = {Proportion of illiterates (\%)},
ymajorgrids = true,
%xmajorgrids = true,
ylabel=,
symbolic x coords={Female, Male},
xtick=data,
ytick={0,20,...,80,100},
nodes near coords,
bar width=0.90cm,
every node near coord/.append style={font=\boldmath},
nodes near coords align={vertical},
every x tick label/.append style={font=\bfseries},
every y tick label/.append style={font=\boldmath},
%tick label style={font=\boldmath}, 
label style={color=black,font=\bfseries},
width=10cm,
height=6cm,
ymax=100,
ymin=0
]
\addplot
[fill=black,draw=black]
coordinates {(Female,60) (Male,50)};
\addplot
[fill=gray,draw=gray]
coordinates {(Female,40) (Male,40)};
\legend{2001\;\;,2011}
\end{axis}

\draw[] (-1,-1.70) rectangle (9,5.5);
\node[below] at (0,-1.75) {Panel (a)};

\end{tikzpicture}

%second image

\begin{tikzpicture}[scale = 0.8]

\begin{scope}
\draw[draw=black,line width=0.15mm,thick] (0,0) -- (-56:2cm) arc[start angle=-56,end angle=88,radius=2cm] -- cycle;

\draw[draw=black,line width=0.15mm,thick] (0,0) -- (88:2cm) arc[start angle=88,end angle=304,radius=2cm] -- cycle;

\node[below] at (30:1.15cm) {\textbf{40\%}};
\node[above] at (30:1.05cm) {\textbf{Female}};

\node[below] at (170:1cm) {\textbf{60\%}};
\node[above] at (170:1cm) {\textbf{Male}};

\draw[] (-2.5,-2.5) rectangle (2.5,3);
\node[above] at (0,2.25) {$2001$};
\node[below] at (-1.5,-2.5) {\textbf{Panel (b)}};

\end{scope}

\begin{scope}[xshift = 6cm]
\draw[draw=black,line width=0.15mm,thick] (0,0) -- (-90:2cm) arc[start angle=-90,end angle=90,radius=2cm] -- cycle;

\draw[draw=black,line width=0.15mm,thick] (0,0) -- (90:2cm) arc[start angle=90,end angle=270,radius=2cm] -- cycle;

\node[below] at (8:1.15cm) {\textbf{50\%}};
\node[above] at (8:1.05cm) {\textbf{Female}};

\node[below] at (170:1cm) {\textbf{50\%}};
\node[above] at (170:1cm) {\textbf{Male}};

\draw[] (-2.5,-2.5) rectangle (2.5,3);
\node[above] at (0,2.25) {$2011$};
\node[below] at (-1.5,-2.5) {\textbf{Panel (c)}};
\end{scope}

\end{tikzpicture}




% https://gateoverflow.in/302846/gate-cse-2019-question-2

\begin{tikzpicture}[minimum height=2cm,scale=2] 
        \node[and gate US, draw,logic gate inputs=nniin] (A) {}; 
        \foreach \a [evaluate=\a as \aeval using int(6-\a)]in {1,...,5}
        {
            \draw (A.input \a -| -1,0)node [left]{$A_{1\aeval}$} -- (A.input \a) ; 
            }
        \draw (A.output) -- ([xshift=0.8cm]A.output) node [anchor=west]{CS};
    \end{tikzpicture} 

%%%%https://gateoverflow.in/302799/gate-cse-2019-question-49?show=303878#a303878
\begin{tikzpicture}[scale=0.7,transform shape,font=\large]

%%%Switch1
\node[rectangle,draw,minimum width = 30mm,minimum height = 30mm,anchor=center,fill=lightgray,ultra thick] (n1) {Switch1};
\node[circle,draw,minimum size = 1cm,thick,left=1.4cm of n1] (c1) {$M_1$};
\node[circle,draw,minimum size = 1cm,thick, above= 2cm of c1, xshift=2.1cm] (c2) {$M_2$};
\node[circle,draw,minimum width = 1cm,thick, above= 2cm of c1, xshift=3.3cm] (c3) {$M_3$};
\node[circle,draw,minimum width = 1cm,thick, above= 2cm of c1, xshift=4.5cm] (c4) {$M_4$};

\node[circle,draw,minimum width = 1cm,thick, below= 2cm of c1, xshift=2.1cm] (c5) {$M_5$};
\node[circle,draw,minimum width = 1cm, thick, below= 2cm of c1, xshift=3.3cm] (c6) {$M_6$};
\node[circle,draw,minimum width = 1cm,thick, below= 2cm of c1, xshift=4.5cm] (c7) {$M_7$};

\draw [ thick] (n1) -- (c1);
\draw [ thick] (n1) -- (c2.south);
\draw [ thick] (n1) -- (c3.south);
\draw [thick] (n1) -- (c4.south);
\draw [ thick] (n1) -- (c5.north);
\draw [thick] (n1) -- (c6.north);
\draw [ thick] (n1) -- (c7.north);
%%%Switch2
\node[rectangle,draw,minimum width = 30mm,minimum height = 30mm,anchor=center,right=20mm of n1,fill=lightgray,ultra thick] (n2) {Switch2};

\node[circle,draw, ,minimum size = 1cm,thick,above= 1cm of n2,xshift=-1.2cm] (c8) {$M_8$};
\node[circle,draw,minimum width = 1cm, thick, above= 1cm of n2, xshift=0cm] (c9) {$M_9$};
\node[circle,draw,minimum size = 1cm,thick, above= 1cm of n2, xshift=1.2cm] (c10) {$M_{10}$};

\node[circle,draw,minimum width = 1cm,thick, below= 1cm of n2, xshift=-1.2cm] (c11) {$M_{11}$};
\node[circle,draw,minimum width = 1cm,thick, below= 1cm of n2, xshift=0cm] (c12) {$M_{12}$};
\node[circle,draw,minimum width = 1cm,thick, below= 1cm of n2, xshift=1.2cm] (c13) {$M_{13}$};

\draw [thick] (n2) -- (c8.south);
\draw [ thick] (n2) -- (c9.south);
\draw [ thick] (n2) -- (c10.south);
\draw [thick] (n2) -- (c11.north);
\draw [thick] (n2) -- (c12.north);
\draw [thick] (n2) -- (c13.north);

%%%Switch3
\node[rectangle,draw,minimum width = 30mm,minimum height = 30mm,anchor=center,right=20mm of n2,fill=lightgray,ultra thick] (n3) {Switch3};

\node[circle,draw, ,minimum size = 1cm,thick,above= 1cm of n3,xshift=-1.2cm] (c14) {$M_{14}$};
\node[circle,draw,minimum width = 1cm,thick, above= 1cm of n3, xshift=0cm] (c15) {$M_{15}$};
\node[circle,draw,minimum size = 1cm,thick, above= 1cm of n3, xshift=1.2cm] (c16) {};

\node[circle,draw,minimum width = 1cm,thick, below= 1cm of n3, xshift=-1.2cm] (c17) {};
\node[circle,draw,minimum width = 1cm,thick, below= 1cm of n3, xshift=0cm] (c18) {};
\node[circle,draw,minimum width = 1cm,thick, below= 1cm of n3, xshift=1.2cm] (c19) {};

\node[circle,draw,minimum size = 1cm,thick,right=1.4cm of n3] (c20) {};

\draw [thick] (n3) -- (c14.south);
\draw [thick] (n3) -- (c15.south);
\draw [ thick] (n3) -- (c16.south);
\draw [thick] (n3) -- (c17.north);
\draw [thick] (n3) -- (c18.north);
\draw [ thick] (n3) -- (c19.north);
\draw [thick] (n3.east) -- (c20.west);

\draw [thick] (n1.east) -- (n2.west);
\draw [thick] (n2.east) -- (n3.west);

\end{tikzpicture}

%https://gateoverflow.in/118703/gate-cse-2017-set-1-question-04
\begin{tikzpicture}[scale=0.8]
\begin{semilogyaxis}[height=10cm,width=15cm,
        scaled ticks=false,font=\Large,
        log ticks with fixed point,
        ylabel near ticks,
ytick={1,100,1000,10000,100000,1000000,10000000,100000000,1000000000},
         max space between ticks=1000pt,
        %try min ticks=5,
       grid=both,
        axis lines=left,
        axis line style=-,
        enlargelimits={upper=0.3},
        ylabel = Number of Operations,
        xlabel = Number of Elements,
        ymin=1,ymax=,xmin=0,xmax=100,domain=2:100,
        log basis y =2,
        yticklabels={1,100,1000,10k,100k,1M,10M,100M,1G,10G,100G},
       % ylabel shift = 2cm
    ]
        \addplot [brown,thick,samples at={0,1,2,4,6,8,10}] {factorial(x)}  node[above,left]{$n!$};
        \addplot [blue,thick,samples at = {0,4,8,12,16,20}] {pow(2,x)}       node[above right]{$2^n$};
        \addplot [red,thick] {x^2}           node[above left]{$n^2$};
        \addplot [black!60!green,thick] {x*(ln(x)/ln(2))}       node[above left]{$n\log(n)$};
        \addplot [purple,thick] {x}             node[above left]{$n$};
        \addplot [black!60!gray,thick] {ln(x)/ln(2)}         node[above left]{$\log(n)$};
        \addplot [violet,thick] {2}             node[above,xshift = 0cm]{$2=O(1)$};
\end{semilogyaxis}  
\end{tikzpicture}
%\end{document}


%https://gateoverflow.in/118253/gate-cse-2017-set-2-question-13#310456
%https://gateoverflow.in/118253/gate2017-2-13?show=310456#a310456
\begin{tikzpicture}[draw, minimum width=1.3cm, minimum height=0.8cm,scale = 0.5]
    \node[draw] (in) at (0,0) {Front};
    \node[draw] (out) at (5,0) {};
    \node[draw] (out) at (10,0) {};
    \node[draw] (out2) at (15,0) {Rear};
    \node[draw] (out) at (0,-4) {P};
    \node at (4,-4) {?};
     \draw[->,>=latex] (1.3,0) -- (3.7,0);
     \draw[->,>=latex] (6.3,0) -- (8.7,0);
     \draw[->,>=latex] (11.3,0) -- (13.7,0);
     \draw[->,>=latex] (1.3,-4) -- (3.7,-4);
  
  \draw [->,>=latex] plot [smooth, tension = 0.5] coordinates { (out2.east) ($(out2.east) + (1.5,0.4)$) 
  ($($($(out2.east) + (1.5,0.6)$)!0.3! (8,2)$) + (0,0.5)$)
  (8,2.2) 
  ($($($(in.west) + (-1.5,0.6)$)!0.3! (8,2)$) + (0,0.5)$)
  ($(in.west) + (-1.5,0.6)$) (in.west) } node {};
  
\end{tikzpicture}

       
       

%https://gateoverflow.in/118378/gate2017-2-36?show=118714#a118714

\usetikzlibrary{trees}

\begin{tikzpicture}[level distance=2.5cm,
level 1/.style={sibling distance=6.5cm},
level 2/.style={sibling distance=3cm},
level 3/.style={sibling distance=2cm},
scale=0.6]
\begin{scope}
\tikzstyle{every node}=[circle,draw]

\node{12}
    child {
    node {8} 
        child {
            node {6}
            child { node {2}}
            child { node {7}}}
        child { node {9}
            child [missing]
            child {node {10}}}
}
child {
 node{16} 
    child {
    node {15}}
    child { node {19}
        child { node {17}}
        child {node{20}}
    }
 }
  ;
\end{scope}

\node at (-5,-2.5){2\quad 6\quad 7 };
\node at (-1.8,-2.5){9\quad 10 };
\node at (2.1,-2.5){15 };
\node at (5.5,-2.5){17\quad 19\quad 20 };
\node at (-5.6,-5){2 };
\node at (-3.9,-5){7 };
\node at (-0.8,-5){10 };
\node at (2.8,-5){17 };
\node at (5.9,-5){20 };
\end{tikzpicture}


%https://gateoverflow.in/118292/gate-cse-2017-set-1-question-12#120564
%https://gateoverflow.in/118292/gate2017-1-12?show=120564#a120564


\begin{tikzpicture}[stack/.style={
  rectangle split, rectangle split parts=5, draw, anchor=center},
  myarrow/.style={single arrow, draw=none},scale=0.6,transform shape]

\node [draw,rectangle,align=left,text width=9cm,above] (mid)
  {\textbf{Static Single assignment:} \\ Each assignment to a temporary is given a unique name \\ All of the uses reached by that assignment are renamed};

\node[font= \huge] at (2.5,-2) {p = a $-$ b};
 
\node[font= \huge] at (2.5,-3) {q = p $*$ c};
\node[font= \huge] at (2.5,-4) {p = u $*$ v};
\node[font= \huge] at (2.5,-5) {q = p $+$ q};
\draw[-,color= cyan,thick] (1,-2) -- (0.5,-2) -- (0.5,-4) -- (1,-4);

\draw[-,color= cyan,thick] (1,-3) -- (0,-3) -- (0,-5) -- (1,-5);
\draw[-,color= cyan,thick] (5.3,-1)--(5.3,-6);
\node[font= \huge] at (8,-2) { = a - b};
\node[font=\Large,circle, inner sep=0pt, minimum size=1cm, color=white, fill=cyan] at (6.3,-2){$p_1$};
\node[font=\Large,circle, inner sep=0pt, minimum size=1cm, color=white, fill=cyan] at (8,-3){$p_1$};
\node[font=\Large,circle, inner sep=0pt, minimum size=1cm, color=white, fill=cyan] at (6.3,-3){$q_1$};
\node[font=\Large,circle, inner sep=0pt, minimum size=1cm, color=white, fill=cyan] at (6.3,-4){$p_2$};
\node[font=\Large,circle, inner sep=0pt, minimum size=1cm, color=white, fill=cyan] at (8,-5){$p_2$};
\node[font=\Large,circle, inner sep=0pt, minimum size=1cm, color=white, fill=cyan] at (9.6,-5){$q_1$};
\node[font= \huge] at (7.1,-3) {=};
\node[font= \huge] at (8,-4) { = u $*$ v};
\node[font= \huge] at (7.1,-5) {= };
\node[font= \huge] at (8.7,-5) {$+$};
\node[font= \huge] at (9.1,-3) {$*$ c};
\node[font= \huge] at (6.3,-5) {$q_2$};
\draw[->](6.6,-2)--(7.5,-2.7);
\draw[->](6.3,-3)--(9,-4.6);
\draw[->](6.6,-4)--(7.5,-4.6);

\end{tikzpicture}

%https://gateoverflow.in/313521/gate2017-ec-1-ga-7?show=313839#a313839
\begin{tikzpicture}[scale=0.5]
\node[font = \huge,draw,circle] at (-4,0) {W} ;
\node[font = \huge,draw,circle] at (4,0) {U} ;
\node[font = \huge,draw,circle] at (0,4) {X} ;
\node[font = \huge,draw,circle] at (0,-4) {T} ;

\node[font = \huge,draw,circle] at (2.82,2.82) {S} ;
\node[font = \huge,draw,circle] at (2.82,-2.82) {Y} ;
\node[font = \huge,draw,circle] at (-2.82,2.82) {Z} ;
\node[font = \huge,draw,circle] at (-2.82,-2.82) {V} ;
\end{tikzpicture}
%%%%%%%%https://gateoverflow.in/313658/gate2017-me-1-ga-3
\usetikzlibrary{arrows}



%\newcommand{\AxisRotator}[1][rotate=0]{%
 %   \tikz [x=0.25cm,y=0.60cm,line width=0.7mm,-stealth,#1] \draw (0,0) arc (-150:150:1 and 1);%
%}

\begin{tikzpicture}[scale=0.8]
\draw[line width = 0.5mm] (10,0.7)node[above] at (10,0) {P}  -- (10,6) node [midway] {\AxisRotator[rotate=-90]?};
\draw[thick,line width=0.7mm] (-1,0) -- (14,0)node [above]{Ground};

\draw [-,line width= 0.7mm,fill=green!5] (2,0)--(10,0) --(4.1,5.22);

\draw[rotate=-24,line width= 0.5mm,fill = black!10 ] (1.6,3.63) ellipse (40pt and 80pt);

\draw [-,line width= 0.5mm,dashed](10,0) -- node [midway,sloped,above,font=\scriptsize]{\textbf{h = 12 cm}} (2.7,2.7) -- node [midway,sloped,above,font=\scriptsize]{\textbf{r = 5 cm}}(4.2,5.3);

 \draw[->,>=stealth',semithick] (3,5.6)node[above=0.4cm, right] {360 deg.} arc (130:40:1cm);
     \draw [fill=red, draw=black] (2.1,0) circle (1.8pt);
 \node[above] at (2.2,0){Q};
\end{tikzpicture}

%%%%%%%%https://gateoverflow.in/313480/gate2017-ce-1-ga-10  
\begin{tikzpicture}

\draw[-] (0,0) -- (0,11);
\draw[-] (-0.2,0.5)node[left]{0} -- (11,0.5);
\draw[-] (-0.2,1.5)node[left]{2} -- (11,1.5);
\draw[-] (-0.2,2.5)node[left]{4}  -- (11,2.5);
\draw[-] (-0.2,3.5) node[left]{6} -- (11,3.5);
\draw[-] (-0.2,4.5)node[left]{8}  -- (11,4.5);
\draw[-] (-0.2,5.5)node[left]{10}  -- (11,5.5);
\draw[-] (-0.2,6.5)node[left]{12}  -- (11,6.5);
\draw[-] (-0.2,7.5)node[left]{14}  -- (11,7.5);
\draw[-] (-0.2,8.5)node[left]{16}  -- (11,8.5);
\draw[-] (-0.2,9.5) node[left]{18} -- (11,9.5);
\draw[-] (-0.2,10.5) node[left]{20} -- (11,10.5);

\draw[-,fill=pink!60] (0.5,0.5) -- (0.5,1.5) -- (1.7,1.5)-- (1.7,0.5) -- cycle;
\draw[-,fill=yellow!60] (0.5,1.5) -- (0.5,5) -- (1.7,5)-- (1.7,1.5) -- cycle;
\draw[-,fill=cyan!60] (0.5,5) -- (0.5,6.5) -- (1.7,6.5)-- (1.7,5) -- cycle;

\draw[-,fill=pink!60] (4.5,0.5) -- (4.5,3) -- (5.7,3)-- (5.7,0.5) -- cycle;
\draw[-,fill=yellow!60] (4.5,3) -- (4.5,7) -- (5.7,7)-- (5.7,3) -- cycle;
\draw[-,fill=cyan!60] (4.5,7) -- (4.5,8) -- (5.7,8)-- (5.7,7) -- cycle;

\draw[-,fill=pink!60] (2.5,0.5) -- (2.5,5.5) -- (3.7,5.5)-- (3.7,0.5) -- cycle;
\draw[-,fill=yellow!60] (2.5,5.5) -- (2.5,6.5) -- (3.7,6.5)-- (3.7,5.5) -- cycle;
\draw[-,fill=cyan!60] (2.5,6.5) -- (2.5,10.5) -- (3.7,10.5)-- (3.7,6.5) -- cycle;

\draw[-,fill=pink!60] (6.5,0.5) -- (6.5,1.5) -- (7.7,1.5)-- (7.7,0.5) -- cycle;
\draw[-,fill=yellow!60] (6.5,1.5) -- (6.5,3) -- (7.7,3)-- (7.7,1.5) -- cycle;
\draw[-,fill=cyan!60] (6.5,3) -- (6.5,4.5) -- (7.7,4.5)-- (7.7,3) -- cycle;

\draw[fill=pink!90,postaction={
        pattern=north west lines
    }] (8.5,0.5) -- (8.5,2.5) -- (9.7,2.5)-- (9.7,0.5) -- cycle;
\draw[-,fill=yellow!60] (8.5,2.5) -- (8.5,7.5) -- (9.7,7.5)-- (9.7,2.5) -- cycle;
\draw[-,fill=cyan!60] (8.5,7.5) -- (8.5,8) -- (9.7,8)-- (9.7,7.5) -- cycle;

\node[below,font=\huge] at (1.1,0.5){$c_1$};
\node[below,font=\huge] at (3.1,0.5){$c_2$};
\node[below,font=\huge] at (5.1,0.5){$c_3$};
\node[below,font=\huge] at (7.1,0.5){$c_4$};
\node[below,font=\huge] at (9.1,0.5){$c_5$};

\filldraw [fill=cyan!60] (9,8.3) rectangle (10.2,8.8)node[right]{bed};
\filldraw [fill=yellow!60, draw=black] (9,8.8) rectangle (10.2,9.3)node[right]{chair};
\filldraw [fill=pink!60, draw=black] (9,9.3) rectangle (10.2,9.8)node[right]{table};
\node at(4.5,-0.5) {Carpenter(C)};
\node[rotate=90] at(-1,4.5) {number of furniture Items};

\end{tikzpicture}

\begin{tikzpicture}
\begin{axis}[
    ybar stacked,
	bar width=25pt,
	ytick={0,2,4,6,8,10,12,14,16,18,20},
	grid = both,
	        ymajorgrids=true,
	        xmajorgrids = false,
	        xtick = \empty,
    %minor tick num=3
    %grid style={line width=.1pt, draw=gray!10},
   % enlargelimits=0.15,
    %legend style={at={(0.5,-0.20)},
     % anchor=north,legend columns=-1},
      % legend style={
    %  anchor=north},
    ylabel={Number of furniture items},
    xlabel={Carpenter(C)},
    symbolic x coords={C1, C2, C3, C4, C5},
    xtick=data,
    reverse legend
    ]
   
\addplot+[ybar,fill=pink,postaction={
        pattern=north west lines
    },draw=black] plot coordinates {(C1,2) (C2,10) (C3,5) (C4,2) (C5,4)
  };
\addplot+[ybar,fill=yellow,postaction={ pattern=north east lines}] plot coordinates {(C1,7) (C2,2) (C3,8) (C4,3) (C5,10)};
\addplot+[ybar,fill=cyan,postaction={
        pattern=crosshatch dots
    },draw=black] plot coordinates {(C1,3) (C2,8) (C3,2) (C4,3) (C5,1)};

%\legend{\strut Bed, \strut Table, \strut Chair}
\legend{ Chair,  Table, Bed}

\end{axis}
\end{tikzpicture}


%https://gateoverflow.in/118286/gate2017-1-6?show=208895#a208895
% first diagram
\begin{tikzpicture}[iv/.style={circle,draw,minimum size = 8pt,inner
sep=1pt},emph/.style={edge from parent/.style={densely dashed,draw}},level 1/.style={sibling distance=24mm},level 2/.style={sibling distance=22mm},level 3/.style={sibling distance=20mm}, scale = 0.4]
\node[iv]{}
    child[missing]
    child {node[iv]{}
          child[missing]
          child[] {node[iv]{}
                child[missing]
                child[emph]{node[iv]{}}}};

\end{tikzpicture}


%https://gateoverflow.in/118286/gate2017-1-6?show=208895#a208895
% second diagram
\begin{tikzpicture}[every node/.style={circle,draw,minimum size = 10pt,inner sep=1pt},level 1/.style={sibling distance=80mm},level 2/.style={sibling distance=40mm},level 3/.style={sibling distance=25mm}, scale = 0.4]
\node{}
    child {node{}
          child {node{}
                child {node{}}
                child {node{}}}
          child {node{}
                child {node{}}
                child {node{}}}}
    child {node{}
          child {node{}
                child {node{}}
                child {node{}}}
          child {node{}
                child {node{}}
                child {node{}}}};

\end{tikzpicture}

%%%%%%%%%%%https://gateoverflow.in/118395/gate2017-2-50?show=123981#a123981

\usetikzlibrary{shapes}

\begin{tikzpicture}[scale=0.5,transform shape, iv/.style={draw,circle,thick,minimum size=50pt,inner
sep=3pt,text=black},ev/.style={draw,rectangle,minimum
size=30pt,inner sep=3pt,text=black},scale = 1, auto,level distance=2.5cm,level 1/.style={sibling distance=40mm},level 2/.style={sibling distance=30mm},level 3/.style={sibling distance=25mm},font=\Huge]


\node[iv] at (-2,-7)  {0.41}
  child {node[ev]{0.19 } node[below,yshift=-0.5cm]{S}}
  child {node[ev]{0.22}node[below,yshift=-0.5cm]{P}};

    \draw [-,dashed] (1.5,-6) -- (1.5,-13);
   
    \node[iv] at (6,-7)  {0.59}
  child {node[iv]{0.25 }
  child{node[ev]{0.08}node[below,yshift=-0.5cm]{T}}
  child{node[ev]{0.17}node[below,yshift=-0.5cm]{R}}}
  child {node[ev]{0.34}node[below,yshift=-0.5cm]{Q}};

  
  %%
   \node[iv] at (6,-15){1}
  child {node[iv] at (-1,0) {0.41} 
  child{node [ev]{0.19} node[below,yshift=-0.5cm]{S}
  edge from parent node [left,yshift=0.2cm]{0}
  }
   %edge from parent[left]node []{0}
  child{node[ev]{0.22}  node[below,yshift=-0.5cm]{P}
  edge from parent node [right,yshift=0.2cm]{1}
  }
  edge from parent node [left,yshift=0.2cm]{0}
  }
  child{node[iv] at (1,0){0.59}
  child{node[iv]{0.25}
  child{node [ev]{0.08} node[below,yshift=-0.5cm]{T}edge from parent node[left,yshift=0.2cm]{0}}
  child{node[ev]{0.17} node[below,yshift=-0.5cm]{R}edge from parent node[right,yshift=0.2cm]{1}}
  edge from parent node[left,yshift=0.2cm]{0}
  }
  child{node[ev]{0.34} node[below,yshift=-0.5cm]{Q}edge from parent node[right,yshift=0.2cm]{1}}
  edge from parent node[right,yshift=0.2cm]{1}
  };

\node[font = \huge] at (0,0) {P: 0.22};
\node[font = \huge] at (3,0) {Q: 0.34};
\node[font = \huge] at (6,0)(r) {R: 0.17};
\node[font = \huge] at (9,0) {S: 0.19};
\node[font = \huge] at (12,0)(t) {T: 0.08};

\node[font = \huge] at (1.5,-2)(p) {P: 0.22};
\node[font = \huge] at (4,-2) {Q: 0.34};
\node[font = \huge] at (6.5,-2) (s){S: 0.19};
\node[font = \huge] at (8.5,-2)(1) { 0.25};

\draw [-] (r) -- (1) -- (t);

\node[font = \huge] at (1.5,-5)(2) { 0.41};
\node[font = \huge] at (4,-5) {0.25};
\node[font = \huge] at (6.5,-5)(1) { Q:0.34};
\draw[-] (p) -- (2) -- (s);

\node [font = \huge,text width=1.5cm] at (-3,-18){P:01 \\ Q:11 \\R:101\\S:00\\T:100};
% \node [font = \huge] at (-3,-13){Q:11};
% \node [font = \huge] at (-3,-13.5){R:101};
% \node [font = \huge] at (-3,-14){S:00};
% \node [font = \huge] at (-3,-14.5){T:100};

%\node [font = \huge] at (-3,-13){Q};
%\node [font = \huge] at (-3,-13.5){R};
%\node [font = \huge] at (-3,-14){S};
%\node [font = \huge] at (-3,-14.5){T};




\end{tikzpicture}



%%%%%%%%%%%https://gateoverflow.in/118395/gate2017-2-50?show=123981#a123981

\usetikzlibrary{shapes}

\begin{tikzpicture}[iv/.style={draw,fill=red!50,circle,minimum size=40pt,inner
sep=0pt,text=black},ev/.style={draw,fill=yellow,rectangle,minimum
size=40pt,inner sep=0pt,text=black},scale = 1, auto]


\node[iv] at (-2,-7)  {0.41}
  child {node[ev]{0.19 }}
  child {node[ev]{0.22}};
  \node at (-2.7,-9.2) {S};
    \node at (-1.6,-9.2) {P};
    \draw [-,dashed] (1.5,-6) -- (1.5,-10);
   
    \node[iv] at (6,-7)  {0.59}
  child {node[iv]{0.25 }
  child{node[ev]{0.08}}
  child{node[ev]{0.17}}}
  child {node[ev]{0.34}};
  \node at (6.5,-9.2) {Q};
   \node at (4.5,-10.7) {T};
    \node at (6,-10.7) {R};
  
  %%
   \node[iv] at (6,-12.5){1}
  child {node[iv] at (-1,0){0.41}
  child{node [ev]{0.19}}
  child{node[ev]{0.22}}}
  child{node[iv] at (1,0){0.59}
  child{node[iv]{0.25}
  child{node [ev]{0.08}}
  child{node[ev]{0.17}} }
  child{node[ev]{0.34}}};

 \node at (4.7,-13.3) {0};
   \node at (7.2,-13.3) {1};
    \node at (3.7,-14.5) {0};
      \node at (4.9,-14.5) {1};
      
       \node at (7.2,-14.5) {0};
      \node at (8.3,-14.5) {1};
      
       \node at (6.3,-16.3) {0};
      \node at (7.7,-16.3) {1};
      
       \node at (3.5,-16.2) {S};
      \node at (5,-16.2) {P};
       \node at (8.6,-16.2) {Q};
      \node at (7.7,-17.7) {R};
       \node at (6.3,-17.7) {T};
      

  
  
\node[font = \huge] at (0,0) {P: 0.22};
\node[font = \huge] at (3,0) {Q: 0.34};
\node[font = \huge] at (6,0)(r) {R: 0.17};
\node[font = \huge] at (9,0) {S: 0.19};
\node[font = \huge] at (12,0)(t) {T: 0.08};

\node[font = \huge] at (1.5,-2)(p) {P: 0.22};
\node[font = \huge] at (4,-2) {Q: 0.34};
\node[font = \huge] at (6.5,-2) (s){S: 0.19};
\node[font = \huge] at (8.5,-2)(1) { 0.25};

\draw [-] (r) -- (1) -- (t);

\node[font = \huge] at (1.5,-5)(2) { 0.41};
\node[font = \huge] at (4,-5) {0.25};
\node[font = \huge] at (6.5,-5)(1) { Q:0.34};
\draw[-] (p) -- (2) -- (s);

\node [font = \huge] at (-3,-12.5){P:01};
\node [font = \huge] at (-3,-13){Q:11};
\node [font = \huge] at (-3,-13.5){R:101};
\node [font = \huge] at (-3,-14){S:00};
\node [font = \huge] at (-3,-14.5){T:100};

%\node [font = \huge] at (-3,-13){Q};
%\node [font = \huge] at (-3,-13.5){R};
%\node [font = \huge] at (-3,-14){S};
%\node [font = \huge] at (-3,-14.5){T};




\end{tikzpicture}

%%%%%%%%%%%%%%%%https://gateoverflow.in/118331/gate2017-1-48?show=119365#a119365

%\usetikzlibrary{calc,shapes.multipart,chains,arrows}


\begin{tikzpicture}[transform shape,scale = .5,font=\Large]
\node[draw,ellipse,text=red] at (0,0){\bf{.}};
\node[draw,circle,fill=red,minimum size=0.2cm,inner sep=0]{};
\node[draw,circle,text=red] at (3,0){\bf{.}};
\node[draw,circle,fill=red,minimum size=0.2cm,inner sep=0]at (3,0){};
\node[draw,circle,text=red] at (6,0){\bf{.}};
\node[draw,circle,fill=red,minimum size=0.2cm,inner sep=0]at (6,0){};
\draw[|-|,line width = 0.5mm](-3,0) -- (8,0);
\node[above,color=red] at (-0.4,0.2) {miss};
\node[above] at (3.6,0.2) {hit};

\draw[->,line width = 0.5mm,color=cyan] (7,3) node [above, font = \small,color=black] {neighbourhood of probe} -- (6.2,0.3);

%\draw[->,line width = 0.5mm] (1.3,2) node [above, font = \small] {1st probe} -- (2.8,0.3);
\draw[bend left =45,->,line width = 0.5mm ,color=cyan](1,2)node [above,color=black,yshift=0.2cm]{1st probe} to  (6,0.2);
\node[yshift = -0.4cm,,xshift=0cm,right,color = red] at (6.4,0.2) {miss};
\draw[bend left =90,->,line width = 0.5mm,color=cyan ](0,0) to node [above,color=black]{3rd probe} (3,0.2);

\draw[bend left =95,->,line width = 0.5mm,color=cyan ](6,0) to node [below,color=black]{2nd probe} (0,-0.2);

  
\end{tikzpicture}

%https://gateoverflow.in/118305/gate2017-1-25#119604
\tikzstyle{decision} = [diamond, draw, fill=blue!20, 
    text width=4.5em, text badly centered, node distance=3cm, inner sep=0pt]
\tikzstyle{block} = [rectangle, draw, fill=blue!20, 
    text width=5em, text centered, rounded corners, minimum height=4em]
\tikzstyle{line} = [draw, -latex']
\tikzstyle{cloud} = [draw, ellipse,fill=red!20, node distance=3cm,
    minimum height=2em]
    
\begin{tikzpicture}[node distance = 2cm, auto]
    % Place nodes
    \node [block] (L1) {L1};
    \node [block, left of= L1, above of = L1] (cpu) {CPU};
  
    \node [block, below of=L1] (L2) {L2};
   \node (r4) [right of = L1,above of = L2,node distance=1cm,text width=2cm] { 140 mem ref};
   \node (r5) [below of = L1,,node distance=4cm,text width=2cm] {};
\node (A) at (1.2,-3) {7 mem ref miss};
\node (B) at (1.2,-4.5) {L2 miss rate = $\frac{7}{140}$ = 0.05};
\node (C) at (1.9,2) {1400 memory references };
\node (D) at (-.3,3.5) {1000 Instructions};
    % Draw edges
   \path [line] (L1) -- (L2);
   \path [line] (L2) -- (r5);
    \path [line] (cpu) -| (L1);
    \draw[] (L1.east) node[right] {Miss Rate = 0.1} (L2.north); 
   %  \draw[] (L1.south) node[right] {Miss Rate = 0.1} (L2.east
  
\end{tikzpicture}


%https://gateoverflow.in/118410/gate2017-1-ga-7?show=120331#a120331
%case 1
          \begin{tikzpicture}[node distance=2cm,auto,>=Latex]
          
             \tikzstyle{nodep}=[circle,scale=1,text=white,draw= blue!80,fill=blue!30]
             \node [circle,draw= cyan!80,fill= cyan!25,scale= 8] (a) at (0,0) { }; 
             \node[nodep] (b) [above of = a] {M};
              \node[nodep] (c) [above left of = a] {M};
              \node[nodep] (d) [left of = a,label={[green] left:{\checkmark}}] {W};
              \node[nodep] (e) [below left of = a,label={[green] below left:{\checkmark}}] {W};
              \node[nodep] (f) [below of = a,label={[green] below:{\checkmark}}] {W};
              \node [rectangle,minimum height=.6cm,minimum width=.6cm,text=white,draw= blue!80,fill=blue!30] (g) [below right of = a,label={[red] below right:{\ding{55}}}] { };
              \draw[->] (d) to [bend right=90, looseness=2] (e);
              \draw[->] (e) to [bend right=90, looseness=2] (f);
              \draw[->] (f) to [bend right=90, looseness=2] (g);
              \draw[->] (2,-.2) -- (1.5,-1);
              
          \end{tikzpicture}
          
 %case 2         
           \begin{tikzpicture}[node distance=2cm,auto,>=Latex]
          
             \tikzstyle{nodep}=[circle,scale=1,text=white,draw= blue!80,fill=blue!30]
             \node [circle,draw= cyan!80,fill= cyan!25,scale= 8] (a) at (0,0) { }; 
             \node[nodep] (b) [above of = a,xshift=-.8cm,yshift= -.2cm] {M};
              \node[nodep] (c) [above right of = a,xshift=-.6cm,yshift= .4cm] {M};
              \node [nodep] (g) [right  of = a,label={below right:{Right Handed Man}}] {M};
               \node[nodep] (d) [left of = a,label={[green] left:{\checkmark}}] {W};
               \node[nodep] (f) [below of = a,xshift=.8cm,yshift= .2cm,label={[green] below right:{\checkmark}}] {M};
              \node[nodep] (e) [below left of = a,xshift=.7cm,yshift= -.4cm,label={[green] 240:{\checkmark}}] {W};
              \draw[->] (d) to [bend right=90, looseness=2] (e);
              \draw[->] (e) to [bend right=90, looseness=2] (f);
             \node (h) at (3.5,-3) {Left handed Man};
              \draw[->] (h) -- (f);
              
          \end{tikzpicture}
         

            
%https://gateoverflow.in/118701/gate2017-1-02?show=123153#a123153

            \begin{tikzpicture}
                \coordinate [label={left:{\tiny $\exists x \forall y$}}] (H) at (-1,1); 
                \coordinate [label={right:{\tiny $\exists y \forall x$}}] (C) at (1,1); 
                \coordinate [label={left:{\tiny $\forall y \exists x$}}] (G) at (-1,-1); 
                \coordinate [label={right:{\tiny $\forall x \exists y$}}] (D) at (1,-1); 
                \coordinate [label={left:{\tiny $\forall x \forall y$}}] (A) at (-.5,1.5); 
                \coordinate [label={right:{\tiny $\forall y \forall x$}}] (B) at (.5,1.5); 
                \coordinate [label={left:{\tiny $\exists x \exists y$}}] (F) at (-.5,-1.5); 
                \coordinate [label={right:{\tiny $\exists y \exists x$}}] (E) at (.5,-1.5); 
                \coordinate [label={center:{\tiny $P(x,y)$}}] (P) at (0,0); 
                \draw[Latex-Latex] (A)--(B);
                \draw[-Latex] (B)--(C);
                \draw[-Latex] (C)--(D);
                \draw[-Latex] (D)--(E);
                \draw[-Latex] (A)--(H);
                \draw[-Latex] (H)--(G);
                \draw[-Latex] (G)--(F);
                \draw[Latex-Latex] (F)--(E);
                
            \end{tikzpicture}
            
            
            %https://gateoverflow.in/313487/gate2017-ce-1-ga-3?show=313973#a313973
            \def\firstcircle{(0,0) circle (1.5cm)}
 \def\secondcircle{(0:2cm) circle (1.5cm)}

  \colorlet{circle edge}{blue!50}
  \colorlet{circle area}{blue!20}

  \tikzset{filled/.style={fill=circle area, draw=circle edge, thick}, outline/.style={draw=circle edge, thick}}

  \setlength{\parskip}{5mm}

\centering
% Set A and B

 \begin{tikzpicture}
  \begin{scope}
    \clip \secondcircle;
    \draw[filled, even odd rule] \firstcircle
                                 \secondcircle node {Lamp};
  \end{scope}
     \draw[outline] \firstcircle node {Bulb}
               \secondcircle;
   \end{tikzpicture}
   
   
   %%%%%%%%%%%%%%%%%%%%%%%%%%%%%%%
   
 

\usetikzlibrary{decorations.text,calc,arrows.meta}

\begin{tikzpicture}
\coordinate (O) at (0,0);


\draw (O) circle (1.5);
\draw (O) circle (1);


\draw[decoration={text along path,reverse path,text align={align=center},text={Bench}},decorate] (1.6,-1) arc (0:180:1.6);
\draw[decoration={text along path,reverse path,text align={align=center},text={Bed}},decorate] (2.6,-1.5) arc (0:180:2.6);



\end{tikzpicture}


  
  %%https://gateoverflow.in/313523/gate2017-ec-1-ga-5?show=313869#a313869

  \usetikzlibrary{decorations.text,calc,arrows.meta}

\begin{tikzpicture}
\coordinate (O) at (0,0);
\coordinate (O) at (0,0);
\coordinate (O1) at (-1.6,0);

\coordinate (O2) at (-3,0);

\draw (O) circle (1.3);
\draw (O) circle (0.9);
\draw (O1) circle (1);
\draw (O2) circle (0.7);

\draw[decoration={text along path,reverse path,text align={align=center},text={C}},decorate] (1.6,-1.5) arc (0:180:1.6);
\draw[decoration={text along path,reverse path,text align={align=center},text={B}},decorate] (2.6,-1.65) arc (0:180:2.6);
\node at (-1.6,0.7){S};
\node at (-3,0.3){T};

\node at (-1,-2.6)[text width=2cm,align=left]{T = Tables\\S = Shelves\\B = Benches\\C = Chairs};
%\node at (0,-2.5){S = shelves};
%\node at (0,-3){B = benches};
%\node at (0,-3.5){C = Chairs};





\end{tikzpicture}

%https://gateoverflow.in/313508/gate2017-ec-2-ga-4?show=313929#a313929
%Subarna
\begin{tikzpicture}[>=stealth]
 \begin{scope}[ decoration={
    markings,
    mark=at position 0.65 with {\arrow{>}}}
    ] 
    \draw[postaction={decorate}] (2.1,1.02) node[below,xshift=1pt, yshift=1pt, scale=0.8] {\tiny P} -- node[xshift=-4pt, scale=0.8]{\tiny $3$}(2.1,1.6) node[yshift=2pt, xshift=-1pt, scale=0.8] {\tiny X};
    \draw[postaction={decorate}] (2.1,1.6) -- node[yshift=4pt, scale=0.8] {\tiny $4$}(3,1.6) node[below,xshift=2pt, yshift=6pt, scale=0.8] {\tiny Q};
    \draw[postaction={decorate}](3,1.6) -- node[xshift=2pt,yshift= -4pt,rotate= 25, scale=0.6, blue] {\tiny $\sqrt{4^{2} + 3^{2}} = 5$} (2.1,1);
    \draw[postaction={decorate}] (2.1,1) -- node[xshift = 6pt,rotate=17,scale=0.8] {\tiny $10$}(0,-0.1);
    \draw[postaction={decorate}] node[xshift=-4pt,scale=0.8]{\tiny Y}(0,-0.1) -- node[xshift=-4pt,scale=0.8] {\tiny $6$}(0,1) node[xshift=-2pt,yshift=2pt, scale=0.8]{\tiny Z};
    \draw[postaction={decorate}] (0,1) -- node[yshift=4pt,scale=0.8,thick] {\tiny $?$}(2,1);
    
    \draw[|<->|] (0.1,-0.5) -- node[below,rotate=19, scale=0.8] {\tiny $15$} (3.2,1.1);
    
    \draw (2.5,-0.05)node[xshift=-3pt, scale=0.8] {\tiny W} -- (3.5,-0.05)node[xshift=3pt, scale=0.8] {\tiny E};
    \draw (3,0.5)node[yshift=3pt, scale=0.8] {\tiny N} -- (3,-0.5)node[yshift=-3pt, scale=0.8] {\tiny S};
 \end{scope}
\end{tikzpicture}


%%%%%%%%%%%%%
%https://gateoverflow.in/313508/gate2017-ec-2-ga-4?show=313929#a313929

\thispagestyle{empty}

\usetikzlibrary{arrows}
%\newcommand{\midarrow}{\tikz \draw[-triangle 90] (0,0) -- +(.1,0);}


\begin{tikzpicture}[scale=0.5]
\begin{scope}[very thick, every node/.style={sloped,allow upside down}]
  \draw (9,4.5)-- node {\midarrow} (6,3) -- node {\midarrow} (0,0);
   \draw[|<->|] (0.5,-1.2)-- node[below] {15} (9.5,3.3);
 \draw (0,0)-- node {\midarrow} (0,3);
  \draw (0,3)-- node {\midarrow} (6,3);
  \draw (6,3)-- node {\midarrow} (6,4.5);
    \draw (6,4.5)-- node {\midarrow} (9,4.5);
    
    \node[left] at (0,0) {$\textbf{Y}$};
      \node[below] at (6,3) {$\textbf{P}$};
        \node[right] at (9,4.5) {$\textbf{Q}$};
          \node[above] at (6,4.5) {$\textbf{X}$};
           \node[above] at (0,3) {$\textbf{Z}$};
           
           \node[left] at (-.2,1.5){6};
           \node[left] at (5.8,3.8){3};
           \node[above] at (7.5,4.7){4};
            \node[above] at (3,3.2){?};
           \node[below] at (3,1.3){10};
             \node[below,rotate=25] at (7.5,3.8){\tiny $\sqrt{3^2 + 4^2}=5$};
             
           
\end{scope}
  \draw[-,thick] (9,-6) -- (9,-2);
             \draw[-,thick] (7,-4) -- (11,-4);
             \node[below] at (9,-6){$\bf{S}$};
               \node[left] at (7,-4){$\bf{W}$};
                 \node[right] at (11,-4){$\bf{E}$};
                   \node[above] at (9,-2){$\bf{N}$};
\end{tikzpicture}



  %https://gateoverflow.in/118423/gate2017-2-ga-9    
  \begin{tikzpicture}[scale=1.5]
      \draw[->] (-1.5,0) -- (1.2,0) node[right] {$x$};
     % \draw[->] (0,-1) -- (0,1.5) node[above] {$y$};
      \filldraw (-0.24,0) circle[radius=1pt];
      \draw[-,dotted,thick] (-0.24,-0.63) -- (-0.24,0) node[above] {$k$};
      \draw[scale=0.5,domain=-2.5:1.2,smooth,variable=\x,blue]
      plot (\x,{(exp(\x))+0.5*pow(\x,2)-2});% node [below] {f};
      \node at (0,-1) {Plot of $f(x)$};
    \end{tikzpicture}
    
     %https://gateoverflow.in/118423/gate2017-2-ga-9    
  \begin{tikzpicture}[scale=1.5]
      \draw[->] (-1.5,0) -- (1.2,0) node[right] {$x$};
      %\draw[->] (0,-1) -- (0,1.5) node[above] {$y$};
       \filldraw (-0.24,0) circle[radius=1pt];
      \draw[-,dotted,thick] (-0.24,-0.63) -- (-0.24,0) node[above] {$k$};
      \filldraw (-1.3,0) circle[radius=1pt];
       \node at (-1.3,0)[below] {-5};
        \filldraw (1,0) circle[radius=1pt];
       \node at (1,0)[below] {5};
      \draw[scale=0.5,domain=-2.5:1.2,smooth,variable=\x,blue]
      plot (\x,{(exp(\x))+0.5*pow(\x,2)-2});% node [below] {f};
      \node at (0,-1) {Image 1};
    \end{tikzpicture}
    
     %https://gateoverflow.in/118423/gate2017-2-ga-9    
  \begin{tikzpicture}[scale=1.5]
      \draw[->] (-1.5,0) -- (1,0) node[right] {$x$};
     % \draw[->] (0,-1) -- (0,1.5) node[above] {$y$};
       \filldraw (-0.24,0) circle[radius=1pt];
      \draw[-,dotted,thick] (-0.24,-0.63) -- (-0.24,0) node[above] {$k$};
      \filldraw (-0.8,0) circle[radius=1pt];
       \node at (-0.8,0)[below] {-5};
        \filldraw (0.8,0) circle[radius=1pt];
       \node at (0.8,0)[below] {5};
      \draw[scale=0.5,domain=-2.5:1.2,smooth,variable=\x,blue]
      plot (\x,{(exp(\x))+0.5*pow(\x,2)-2});% node [below] {f};
      \node at (0,-1) {Image 2};
    \end{tikzpicture}
    
     %https://gateoverflow.in/118423/gate2017-2-ga-9    
  \begin{tikzpicture}[scale=1.5]
      \draw[->] (-1.5,0) -- (1,0) node[right] {$x$};
     % \draw[->] (0,-1) -- (0,1.5) node[above] {$y$};
       \filldraw (-0.24,0) circle[radius=1pt];
      \draw[-,dotted,thick] (-0.24,-0.63) -- (-0.24,0) node[above] {$k$};
      \filldraw (-1.2,0) circle[radius=1pt];
       \node at (-1.2,0)[below] {-5};
        \filldraw (0.2,0) circle[radius=1pt];
       \node at (0.2,0)[below] {5};
      \draw[scale=0.5,domain=-2.5:1.2,smooth,variable=\x,blue]
      plot (\x,{(exp(\x))+0.5*pow(\x,2)-2});% node [below] {f};
      \node at (0,-1) {Image 3};
    \end{tikzpicture}
% \usetikzlibrary{plotmarks}


% \begin{tikzpicture}[dot/.style={draw,rectangle,minimum size=4mm,inner sep=0pt,outer sep=0pt,thick},scale=0.7]
%   \tkzInit[xmax=15,ymax=11,xmin=0,ymin=0]
%   \tkzGrid
%  %  \tkzAxeY
 
%  \foreach \x/\xtext in {0/0,1/2, 2/4, 3/6, 4/8, 5/10,6/12,7/14, 8/16, 9/18, 10/20, 11/22, 12/24, 13/26,14/28,15/30}
% {\draw (\x cm,1pt ) -- (\x cm,-1pt ) node[anchor=north] {$\xtext$};}



% \draw [black] plot [only marks, mark=square] coordinates {(0,0) (1,.3) (2,.6)(3,0.9)(4,1.5)(5,3)(6,4.5)(7,6)(8,6.7)(9,7.1)(10,7.5)(11,7.7)(12,7.9)(13,8)(14,8)(15,8)(2,10)};

% \draw[-] (0,0) -- (1,.3)--(2,.6) -- (3,0.9) -- (4,1.5)-- (5,3)--(6,4.5)--(7,6)--(8,6.7)--(9,7.1)--(10,7.5)--(11,7.7)--(12,7.9)--(13,8)--(14,8)--(15,8);


% \foreach \point in {(0,1.5),(1,1.8),(2,1.9),(3,2.1),(4,2.5),(5,3),(6,4),(7,6),(8,8),(9,9),(10,9.5),(11,10),(12,10.2),(13,10.4),(14,10.4),(15,10.4),(2,9)}{% points
%       \draw[color=red] \point circle (3pt);
% }
% \draw[-] (0,1.5)--(1,1.8)--(2,1.9)--(3,2)--(4,2.5)--(5,3)--(6,4)--(7,6)--(8,8)--(9,9)--(10,9.5)--(11,10)--(12,10.2)--(13,10.4)--(14,10.4)--(15,10.4);

% \node[font = \huge] at (8,-1.5) {Day of the Month};
% \node[font = \huge,rotate= 90] at (-1.5,5) {Pollutant Concentration(ppm)};

% \node[] at (3.5,10){Winter};
% \node[] at (3.5,9.5){Summer};

% \draw (1,10.5) -- (5,10.5) -- (5,8.5) -- (1,8.5) -- cycle;


% \end{tikzpicture}

%https://gateoverflow.in/118301/gate2017-1-21

\begin{karnaugh-map}[4][4][1][][]
        \minterms{4,12,14}
       
      
            \terms{1,3,9,11,6}{X}
        \node at (-0.25,4.7) {ba};           % row name
        \node at (-0.6,4.15) {dc};            % column name
        \draw[ultra thin] (0,4) -- (-1,4.75);  % diagonal line
    \end{karnaugh-map}
    
  %https://gateoverflow.in/118370/gate2017-2-28
\begin{karnaugh-map}[4][4][1][][]
        \minterms{0,4,8,2,10,13,12}
        \maxterms{1,3,7,6,11}
        \implicant{7}{14}
        \implicantedge{1}{3}{9}{11}
       
      
            \terms{5,9,14,15}{d}
        \node at (-0.25,4.7) {WX};           % row name
        \node at (-0.6,4.15) {YZ};            % column name
        \draw[ultra thin] (0,4) -- (-1,4.75);  % diagonal line
    \end{karnaugh-map}
    %
         \usetikzlibrary{positioning}
\tikzset{
  gray box/.style={
    fill=white!20,
    draw=black,
    minimum width={0.5*#1ex},
    minimum height={4em},
  },
  annotation/.style={
    anchor=north,
  }
}

\begin{tikzpicture}[node distance=-0.5pt,font=\Large]
  \node [gray box=2] (p1) {\(P_1\)};

  \node [gray box=3, right=of p1] (p2) {\(P_4\)};
  \node [gray box=28, right=of p2] (p3) {\(P_2\)};
   \node [gray box=7, right=of p3] (p4) {\(P_4\)};
    \node [gray box=9, right=of p4] (p5) {\(P_1\)};
  \node [gray box=2, right=of p5] (p6) {\(P_3\)};
   \node [gray box=16, right=of p6] (p7) {\(P_5\)};
  

 \node [annotation] at (p1.south west) {0};
  \node [annotation] at (p1.south east) {2};
 \node [annotation] at (p2.south east) {5};
  \node [annotation] at (p3.south east) {33};
   \node [annotation] at (p4.south east) {40};
   \node [annotation] at (p5.south east) {49};
  \node [annotation] at (p6.south east) {51};
   \node [annotation] at (p7.south east) {67};
 
\end{tikzpicture}

%https://gateoverflow.in/118304/gate2017-1-24?show=118840#a118840
\begin{tikzpicture}[node distance=-0.5pt]
  \node [gray box=3] (p1) {\(P_1\)};
  \node [gray box=3, right=of p1] (p2) {\(P_2\)};
  \node [gray box=2, right=of p2] (p3) {\(P_4\)};
   \node [gray box=4, right=of p3] (p4) {\(P_1\)};
    \node [gray box=5, right=of p4] (p5) {\(P_3\)};
  

 \node [annotation] at (p1.south west) {0};
  \node [annotation] at (p1.south east) {3};
 \node [annotation] at (p2.south east) {6};
  \node [annotation] at (p3.south east) {8};
   \node [annotation] at (p4.south east) {12};
   \node [annotation] at (p5.south east) {17};
 
 
\end{tikzpicture}

%https://gateoverflow.in/118160/gate-cse-2017-set-2-question-25#118586
%https://gateoverflow.in/118160/gate2017-2-25#118586
\begin{tikzpicture}[>=stealth',shorten >=1pt,auto,node distance=1.5cm]
  \node[state,initial, initial text=]     (1)   {1};
  \node[state] (2)           [right of= 1] {2};
  \node[state] (3)           [right of= 2] {3};
  \node[state] (4)           [above right of= 3] {4};
  \node[state] (5)           [right of= 4] {5};
  \node[state] (6)           [right of= 5] {6};
  \node[state,accepting] (7)           [right of= 6] {7};
  \node[state] (8)           [below right of= 3] {8};
  
  \path[->]
  (1) edge                      node {a,b} (2)
  (2) edge                      node {a,b} (3)
  (3) edge                      node {a}  (4)
      edge                      node {b} (8)
  (4) edge                      node {a,b} (5)
  (5) edge                      node {a,b} (6)
  (6) edge                      node {a,b} (7)
  (7) edge [loop above]         node {a,b} ()
  (8) edge [loop right]         node {a,b} ();
  
\end{tikzpicture}
%https://gateoverflow.in/118302/gate-cse-2017-set-1-question-22#118855
\begin{tikzpicture}[>=stealth',shorten >=1pt,auto,node distance=2cm]
  \node[state]     (A)   {A};
  \node[state] (B) [right of=A] {B};
  \node[state,accepting] (C) [right of=B] {C};
   
   \path[->]
   (A) edge [loop above]         node {a,b} ()
       edge                      node {b} (B)
   (B) edge                      node {a,b} (C);
     
\end{tikzpicture}

%https://gateoverflow.in/118384/gate2017-2-39?show=118678#a118678
\begin{tikzpicture}[>=stealth',shorten >=1pt,node distance=4cm,on grid,auto]
   \node[state,initial,accepting,initial text=] (q_0)     {$q_0$};
   \node[state] (q_2) [right of = q_0] {$q_2$};
   \node[state] (q_3) [below right of = q_2] {$q_3$};
   \node[state] (q_1) [below left of = q_2] {$q_1$};
   
   \path[->]
   (q_0) edge [loop above]      node {b} ()
         edge [bend left=20]    node {null} (q_2)
         edge                   node {a} (q_1)
   (q_1) edge                   node {null,a} (q_2)
         edge                   node {b} (q_3)
   (q_2) edge                   node {null} (q_0)
   (q_3) edge                   node {b} (q_2);
         
\end{tikzpicture}




%https://gateoverflow.in/118278/gate2017-2-21
\begin{tikzpicture}
\tikzstyle{every node}=[circle,draw,scale=0.5]     

\node [label={$\mathbf{e}$}] (a) at (0,1) {};
\node [label=left:{$\mathbf{c}$}](b) at (-1,0) {};
\node [label=right:{$\mathbf{d}$}](c) at (1,-0.5) {};
\node [label=left:{$\mathbf{b}$}](d) at (-1,-1.5) {};
\node [label=below:{$\mathbf{a}$}](f) at (0.2,-2.5) {};


\draw[->,>=stealth',thick] (b) -- (a);
\draw[->,>=stealth',thick] (c) -- (a);
\draw[->,>=stealth',thick] (d) -- (b);
\draw[->,>=stealth',thick] (f) -- (d);
\draw[->,>=stealth',thick] (f) -- (c);



\end{tikzpicture} 

%https://gateoverflow.in/118196/gate-cse-2017-set-2-question-15

\begin{tikzpicture}[scale = 1,font = \sffamily, transform shape, main/.style = {draw, circle}, node distance=2cm and 4cm]
\node[main] (1) {M};
\node[main] (2) [right of=1] {N};
\node[main] (3) [right of=2] {O};
\node[main] (4) [below of=1] {R};
\node[main] (5) [right of=4] {Q};
\node[main] (6) [right of=5] {P};

\draw (1) -- (2);
\draw (2) -- (3);
\draw (1) -- (4);
\draw (1) -- (5);
\draw (2) -- (5);
\draw (3) -- (5);
\draw (3) -- (6);
\draw (5) -- (6);
\end{tikzpicture}


%%%%https://gateoverflow.in/118427/gate-cse-2017-set-2-question-20?show=118481#a118481


\begin{tikzpicture}[line width=.7pt]
 
 \node [fill=pink!50,draw,rectangle,minimum width=2cm,minimum height=1cm,label={$\text{8 bits}$}](11) at(0,0) {};
 \node [fill=gray!20,draw,rectangle,minimum width=2cm,minimum height=1cm,label={$\text{8\ bits}$}] at(2,0) {};
 \node [draw,rectangle,minimum width=8cm,minimum height=1cm,label={$\text{Variable Length}$}] at(7,0) {};
 
 \node [fill=pink!50,draw,rectangle,minimum width=1cm,minimum height=1cm,label=center:$\text{1}$] at(-0.5,-2)(21) {};
 \node [fill=pink!50,draw,rectangle,minimum width=1.5cm,minimum height=1cm,label=center:$\text{2}$] at(0.75,-2)(22) {};
 \node [fill=pink!50,draw,rectangle,minimum width=2.5cm,minimum height=1cm,,label=center:$\text{5}$] at(2.75,-2)(23) {};
 
 \draw[-](11.south west) --(21.north west);
 \draw[-](11.south east) --(23.north east);
 
 \path (0,0)node{$\text{Type}$} (2,0)node{$\text{Length}$} (7,0)node{$\text{Value}$} (4.5,-2)node{$\text{bits}$} ;
 
  \node[draw,label=above:{$\text{Copy}$},align=left,
    rounded corners=0.2cm,
    minimum width=3cm,
    minimum height=1cm] at (1.5,-6)(copy)
    {0 Copy only in first fragment\\ 1 Copy into all fragments};
    
    
 \node[draw,label=above:{$\text{Class}$},align=left,
    rounded corners=0.2cm,
    minimum width=3cm,
    minimum height=1cm] at (6.7,-6)(class)
    {00 Datagram control\\ 01 Reserved\\ 10 Debugging and management\\ 11 Reserved};
    
    
 \node[draw,label=above:{$\text{ Number}$},align=left,
    rounded corners=0.2cm,
    minimum width=3cm,
    minimum height=1cm] at (11.5,-6)(number)
    {00000 End of operation\\ 00001  No operation\\ 00011  Loose source route\\ 00100  Timestamp\\ 00111  Record route\\ 01001  Strict source route};
    
 \draw[->](21.south) --(copy.north west);
  \draw[->](22.south) --(class.north west);
   \draw[->](23.south) --(number.north west);
 
\end{tikzpicture}




% https://gateoverflow.in/118332/gate-cse-2017-set-1-question-49?show=123015#a123015

\begin{tikzpicture}
  \node[draw,black] (A) at (1.2,1) {
    \begin{tabular}{|c|l|l|}
      \hline
      {\footnotesize{2000}} & {\footnotesize{i}} & {\footnotesize{add R2, R3, R4}} \\
      \hline
      {\footnotesize{2004}} & {\footnotesize{i+1}} & {\footnotesize{sub R5, R6, R7}} \\
      \hline
      {\footnotesize{2008}} & {\footnotesize{i+2}} & {\footnotesize{cmp R1, R9, R10}} \\
      \hline
      {\footnotesize{2012}} & {\footnotesize{i+3}} & {\footnotesize{beq R1, Offset}} \\
      \hline
      {\footnotesize{2016}} & \multicolumn{2}{|c|}{\footnotesize{Next instruction}} \\
      \hline
    \end{tabular}
  };
  
\node[align=left,font=\footnotesize](pc) at (-3,0.33){Program Counter\\ is pointing here} ; 
\draw[-Triangle,thick] (pc.east)--(-1,0.33);
  
\node[font=\scriptsize] at (-0.5,2.4){\textbf{Address}} ;
\end{tikzpicture}

% https://gateoverflow.in/118313/gate-cse-2017-set-1-question-32#119014

\begin{tikzpicture}
\node[inner sep=0,outer sep=10] at (-0.4,0.05) {\normalsize\textbf{1 0 1 1}};
\draw[-,thick] (0.3,-0.7)|-(4.8,0.4);
\node[] at (2.35,0) {\normalsize\textbf{0 1 0 1 1 0 1 1 \uline{0} \uline{0} \uline{0}}};
\node[] at (1.15,-0.4) {\normalsize\textbf{0 0 0 0}};

\draw[-,thick] (0.9,-0.7)--(2.15,-0.7);

\node[] at (1.5,-1) {\normalsize\textbf{1 0 1 1}};
\node[] at (1.5,-1.4) {\normalsize\textbf{1 0 1 1}};

\draw[-,thick] (1.95,-1.7)--(3.85,-1.7);

\node[] at (2.9,-2) {\normalsize\textbf{0 0 1 1 0 0}};
\node[] at (3.25,-2.4) {\normalsize\textbf{1 0 1 1}};


\draw[-,thick] (3,-2.7)--(4.2,-2.7);

\node[] at (3.6,-3) {\normalsize\textbf{1 1 1 0}};
\node[] at (3.6,-3.4) {\normalsize\textbf{1 0 1 1}};

\draw[-,thick] (3.3,-3.7)--(4.2,-3.7);

\node[] at (3.75,-4) {\normalsize\textbf{1 0 1}};

\draw[-Triangle,thick,blue] (2,-0.2)--(2,-0.6);

\draw[-Triangle,thick,blue] (2.3,0.-0.2)--(2.3,-1.6);
\draw[-Triangle,thick,blue] (2.65,-0.2)--(2.65,-1.6);
\draw[-Triangle,thick,blue] (3,0.-0.2)--(3,-1.6);
\draw[-Triangle,thick,blue] (3.35,-0.2)--(3.35,-1.6);
\draw[-Triangle,thick,blue] (3.7,-0.2)--(3.7,-1.6);

\draw[-Triangle,thick,blue] (4.05,-0.3)--(4.05,-2.6);

\node[] at (-1,-4) {\normalsize\textbf{M'=0 1 0 1 1 0 1 1 \uline{\textcolor{red}{1 0 1}}}};

\end{tikzpicture}

\tikzstyle{branch}=[>=stealth',shorten >=1pt,node distance=2cm,on grid,auto]
\begin{tikzpicture}[>=stealth,shorten >=1pt,node distance=2cm,on grid,auto]
  \node[initial,state,accepting] (C)                {$C$};
  \node[state,accepting]         (D) [right of=C]   {$D$};
  \node[state]                   (E) [right of=D]  {$E$};   


  \path[->] (C) edge [loop above ]     node {a} (C)
                edge [below]           node {b} (D)
            (D) edge [loop above]      node {b} (D)
                edge [below]           node {a} (E)
            (E) edge [loop above]     node {a} (E)
                edge [loop below]    node {b} (E);
                 
            
\end{tikzpicture}
\begin{tikzpicture}[>=stealth',shorten >=1pt,node distance=4cm,on grid,auto]
  \node[state,accepting] (A)                {$A$};
  \node[state]                   (B) [below left of=A]  {$B$}; 
  \node[state]                   (C) [below right of=A]  {$C$};   



  \path[->] (A) edge [bend left]   node {1/01} (B)
                edge[loop above]    node{0/00} (A)
            (B) edge []  node {1/10}    (C)
                edge [bend left] node {0/01}    (A)
            (C) edge   node {0/01}  (A)
                edge [loop above] node {1/10}    (C);
            
\end{tikzpicture}


\begin{tikzpicture}[>=stealth',shorten >=1pt,node distance=2cm,on grid,auto]
  \node[state,initial]          (A)                {};
  \node[state]                   (B) [right of=A]  {}; 
  \node[state]                   (C) [right of=B]  {};   
  \node[state,accepting]        (D) [right of= C] {};
  
 \path[->] (A) edge [bend left]     node {0} (B)
                edge [loop above]      node{1} (A)
            (B) edge      node{0} (C)
                edge [bend left]      node{1} (A)
            (C) edge            node{1} (D)
                edge [loop above]     node{0} (C)
            (D) edge [loop above]     node{0,1} (D);
\end{tikzpicture}

\begin{tikzpicture}[>=stealth',shorten >=1pt,node distance=3cm,on grid,auto]
  \node[state,initial]          (A)                {};
  \node[state]                   (B) [right of=A]  {}; 
  \node[state,accepting]        (C) [below of =A]  {};   
  
  
 \path[->] (A) edge [bend left]     node {0,1} (B)
                edge      node{1} (C)
            (B) edge      node{0} (C)
                edge [bend left]      node{0} (A)
                edge [loop above]  node{0,1} (B);
          
           
\end{tikzpicture}

\begin{tikzpicture}[shorten >=1pt,node distance=4cm,on grid,auto]
  

  \node[state,]  (Y)                 {$Y$};
  \node[state,initial]  (X) [below left of=Y]  {$X$};
  \node[state]                   (Z)[below of=Y]{$Z$};

  \path[->]
    (X) edge    [loop below]    node{1} (X)
        edge    [bend left]     node{0} (Y)
        edge    [bend left]     node{0} (Z)
    (Y) edge    [loop above]    node{0} (Y)
        edge    [bend left]     node {0} (Z)
    (Z) edge    [bend left]     node {1} (Y)
         edge    [loop below]    node{1} (Z)
         edge    [bend left]     node {0} (X);
    
    
\end{tikzpicture}

\begin{tikzpicture}[shorten >=1pt,node distance=2cm,on grid,auto]
  

  \node[state,initial]  (Q_0)                 {$Q_0$};
  \node[state]          (Q_1) [right of=Q_0]  {$Q_1$};

  \path[->]
      (Q_0)  edge      node {1/1}  (Q_1)
            edge [loop above] node {0/0} (Q_0)
      (Q_1) edge  [loop above]  node {0/1}  (Q_1)
        edge [loop below]   node{1/0}(Q_1);
\end{tikzpicture}

\begin{tikzpicture}[shorten >=1pt,node distance=2cm,on grid,auto]
  

  \node[state]          (S)                {$S=1$};
  \node[state]                 (t) [right of=S]  {$S=0$}; 
  
  
   \path[->]
   (S)  edge    [bend left]    node{x=0} (t)
        edge    [loop above]     node{x=1} (S)
    (t) edge    [bend left]     node{y=0} (S)
       edge    [loop above]    node{y=1} (t);
   
\end{tikzpicture}


\begin{tikzpicture}[shorten >=1pt,node distance=2cm,on grid,auto]
\tikzstyle{every state}=[]
  \node[initial,state] (s)                {$s$};
  \node[state,accepting]         (t) [right of=s]   {$t$};
  \node[state]                   (E) [below of=s]  {};   


  \path[->] (s) edge [bend left]     node {b} (t)
                edge [bend left]    node{a} (E)
            (t) edge [loop above]   node {a} (t)
                edge [bend left]    node {b} (s)
            (E) edge [bend left]    node {a} (s)
                edge [loop below]   node {b} (E);
                
\end{tikzpicture}

\begin{tikzpicture}[>=stealth',shorten >=1pt,node distance=2cm,on grid,auto]
  \node[state,initial]      (q_0)                {$q_0$};
  \node[state,accepting]     (q_1) [right of=A]  {$q_1$}; 
  \node[state,accepting]    (q_2) [right of=B]  {$q_2$};   
  \node[state]              (q_3) [above of=q_1] {$q_3$};
  
 \path[->] (q_0) edge [bend left]     node {a} (q_1)
                edge [loop above]      node{b} (q_0)
            (q_1) edge [bend left]     node{a} (q_2)
                edge [loop below]      node{b} (q_1)
            (q_2) edge [bend left]     node{a} (q_1)
                edge [loop below]     node{b} (q_2)
            (q_3) edge    node{a} (q_1)
                edge  node{b} (q_2);
\end{tikzpicture}



\begin{tikzpicture}[>=stealth',shorten >=1pt,node distance=3cm,on grid,auto]
    \node[state,accepting] (S_1)    {$S_1$};
    \node[state,initial]    (S_0)[below left of=S_1] {$S_0$};
    \node[state]           (S_2)[below right of=S_1] {$S_2$};
    \node[state,accepting] (S_3) [below right of=S_0] {$S_3$};
    
     \path[->]
    (S_0) edge [bend left]     node {x} (S_1)
          edge [bend left]     node {y} (S_3)
    (S_1) edge [bend left]     node {x} (S_0)
          edge [bend left]     node {y} (S_2)
    (S_2) edge [bend left]     node {x} (S_3)
          edge [bend left]     node {y} (S_1)
    (S_3) edge [bend left]     node {x} (S_2)
          edge [bend left]     node {y} (S_0);
            
    
\end{tikzpicture}


\begin{tikzpicture}[->,>=stealth',shorten >=1pt,auto,node distance=2.8cm,
                    semithick]
  \tikzstyle{every state}=[fill=blue,draw=none,text=white]

  \node[state]         (A)                    {$q_a$};
  \node[state]         (B) [above right of=A] {$q_b$};
  \node[state]         (D) [below right of=A] {$q_d$};
  \node[state]         (C) [below right of=B] {$q_c$};
  \node[state]         (E) [below of=D]       {$q_e$};

  \path (A) edge              node {0,1,L} (B)
            edge              node {1,1,R} (C)
        (B) edge [loop above] node {1,1,L} (B)
            edge              node {0,1,L} (C)
        (C) edge              node {0,1,L} (D)
            edge [bend left]  node {1,0,R} (E)
        (D) edge [loop below] node {1,1,R} (D)
            edge              node {0,1,R} (A)
        (E) edge [bend left]  node {1,0,R} (A);
\end{tikzpicture}




%https://gateoverflow.in/118748/gate-cse-2017-set-1-question-54?show=118959#a118959
\begin{tikzpicture}[line width=.75pt]

\begin{scope}[scale=0.7]
  \draw[fill=gray!20] (0,0)node[xshift=0.7cm,yshift=1.5cm]{\text{10 bits}} rectangle(2,1);
  \draw[fill=gray!20] (2,0)node[xshift=2.5cm,yshift=1cm]{\textbf{\text{Direct Mapped}}}rectangle(8,1);
  \draw[fill=gray!20] (8,0)node[below]{}rectangle (11,1);
 
  \path (1,.5)node{$\text{TAG}$};
  \path (5,.5)node{$\text{LINE NUMBER}$};
  \path (9.5,.5)node{$\text{OFFSET}$};
  \draw [decorate,decoration={brace,amplitude=10pt,mirror,raise=4pt},xshift=2pt,rotate=180]
(-1.9,-1) -- (0,-1) ;
\end{scope}

\begin{scope}[yshift=-2.2cm,scale=0.75]
  \draw[fill=gray!20] (0,0)node[xshift=1.1cm,yshift=-0.4cm, text width=2cm,scale=0.8]{$\text{10 bits + 4 bits}$  $(\log 16 = 4)$} rectangle(2,1);
  \draw[fill=gray!20] (2,0)node[below]{} rectangle(3,1);
  \draw[fill=gray!20] (3,0)node[xshift=2cm,yshift=1cm]{\textbf{\text{Set Associative}}}rectangle(7.7,1);
  \draw[fill=gray!20] (7.7,0)node[below]{}rectangle (10.4,1);
 
  \path (1,.5)node{$\text{TAG}$};
  \path (5.4,.5)node{$\text{SET NUMBER}$};
  \path (9.1,.5)node{$\text{OFFSET}$};
  \node[xshift=1.1cm,yshift=-1.2cm]{TAG:14 bits};
  
  \draw [decorate,decoration={brace,amplitude=10pt,mirror,raise=4pt},xshift=0pt]
(-0.1,-0.6) -- (3.15,-0.6) ;
\end{scope}

\end{tikzpicture}


%%https://ec.gateoverflow.in/1525/gate2020-ec-ga-8
\begin{tikzpicture}[scale = 0.8]
\draw[fill=teal!65!white, very thick](0,0) circle (4);
\draw[fill = white, very thick] (-3.20,-2.35) rectangle (3.20,2.35);
\node[above] at (0,0.2) {\huge{$\text{O}$}};
\node[left] at (-3.15,2.5) {\huge{$\text{P}$}};
\node[left] at (-3.25,-2.5) {\huge{$\text{S}$}};
\node[right] at (3.15,2.5) {\huge{$\text{Q}$}};
\node[right] at (3.25,-2.5) {\huge{$\text{R}$}};
\draw[teal!65!white,very thick] (-0.01,0.15) -- (-0.01,-0.20);
\draw[teal!65!white,very thick] (-0.15,0) -- (0.08,0);
\node[] at (-1.75,-0.75) {\huge{$a$}};
\draw[very thick](-3.20,-2.35) -- (0,0);
\end{tikzpicture}


 %%%%%%%https://gateoverflow.in/313674/gate2017-me-2-ga-10
  \begin{tikzpicture}[scale=0.6]
\begin{axis}[ylabel=Pollutant Concentration (ppm),xlabel=Day of the Month,
    xtick={0,2,4,6,8,10,12,14,16,18,20,22,24,26,28,30},
    ytick={0,1,2,3,4,5,6,7,8,9,10,11},
    minor tick num=1,
   xmin = -1,xmax=31,ymin=-.5,
    width=0.6\textwidth,
    height=\axisdefaultheight,
    legend pos=north west,
    grid=both,
    grid style={dotted},
    major grid style={dashed},
]
\addplot[black] plot [mark=square] coordinates { (0,0)(2,.3)(4,.6) (6,0.9) (8,1.5)(10,3)(12,4.5)(14,6)(16,6.7)(18,7.1)(20,7.5)(22,7.7)(24,7.9)(26,8)(28,8)(30,8)};
\addplot [red] plot [mark=o] coordinates {(0,1.5)(2,1.8)(4,1.9)(6,2)(8,2.5)(10,3)(12,4)(14,6)(16,8)(18,9)(20,9.5)(22,10)(24,10.2)(26,10.4)(28,10.4)(30,10.4)};
\addlegendentry{Winter}
\addlegendentry{Summer}
\end{axis}
\end{tikzpicture}  


%https://gateoverflow.in/313662/gate2017-me-1-ga-10
       \begin{tikzpicture}[scale=.5]
       
           \begin{axis}[
           ylabel=Population density,
           xlabel=Time (min),
           xtick={0,20,40,60,80,100,120,140,160,180,200},
           ytick={0.0,0.1,0.2,0.3,0.4,0.5,0.6,0.7,0.8,0.9,1.0},
           minor tick num=1,
           xmin = -10,xmax=210,ymin=-0.07,ymax=1.1,
           width=0.8\textwidth,
           height=\axisdefaultheight,
           legend pos=north west,
           grid=both,
           grid style={dotted},
           major grid style={dashed},
        ]
        \addplot[black] plot [mark=square] coordinates {
        (0,0.0) (10,0.02) (20,0.04) (30,0.06) (40,0.11) (50,0.21) (60,0.4) (70,0.59) (80,0.75) (90,0.82) (100,0.88) (110,0.92) (120,0.95) (130,0.97) (140,0.99) (150,1) (160,1) (170,1) (180,1) (190,1) (200,1)};
        \addplot [red] plot [mark=o] coordinates {
        (0,0.0) (10,0.01) (20,0.02) (30,0.03) (40,0.04) (50,0.05) (60,0.08) (70,0.13) (80,0.2) (90,0.3) (100,0.5) (110,0.67) (120,0.77) (130,0.83) (140,0.88) (150,0.92) (160,0.965) (170,0.98) (180,0.995) (190,1.0) (200,1.0)};
        \addlegendentry{37$^{\circ}$ C}
        \addlegendentry{25$^{\circ}$ C}
        \end{axis} 
           
                   
       \end{tikzpicture}


%%https://me.gateoverflow.in/1813/gate-mechanical-2020-set-2-ga-question-10
\begin{tikzpicture}
\begin{scope}
\filldraw[draw=black,fill=white] (0,0) -- (0:3cm) arc[start angle=0,end angle=54,radius=3cm] -- cycle;
\filldraw[draw=black,fill=white] (0,0) -- (54:3cm) arc[start angle=54,end angle=126,radius=3cm] -- cycle;
\filldraw[draw=black,fill=white] (0,0) -- (126:3cm) arc[start angle=126,end angle=234,radius=3cm] -- cycle;
\filldraw[draw=black,fill=white] (0,0) -- (234:3cm) arc[start angle=234,end angle=288,radius=3cm] -- cycle;
\filldraw[draw=black,fill=white] (0,0) -- (288:3cm) arc[start angle=288,end angle=360,radius=3cm] -- cycle;

\node[] at (10:1.8cm) {\textbf{15\%}};
\node[] at (20:1.88cm) {\textbf{Management}};

\node[] at (90:1.75cm) {\textbf{20\%}};
\node[] at (90:2.15cm) {\textbf{Arts}};

\node[] at (170:1.65cm) {\textbf{30\%}};
\node[] at (160:1.75cm) {\textbf{Engineering}};

\node[] at (260:2.35cm) {\textbf{15\%}};
\node[] at (260:2cm) {\textbf{Commerce}};

\node[] at (320:2cm) {\textbf{20\%}};
\node[] at (330:1.85cm) {\textbf{Science}};

\node[above] at (0,3.4){\textbf{Percentage of students enrolled in different}};
\node[above] at (0,3) {\textbf{stream in a University}};
\end{scope}

\begin{scope}[xshift = 8.5cm]
\filldraw[draw=black,fill=white] (0,0) -- (0:3cm) arc[start angle=0,end angle=54,radius=3cm] -- cycle;
\filldraw[draw=black,fill=white] (0,0) -- (54:3cm) arc[start angle=54,end angle=162,radius=3cm] -- cycle;
\filldraw[draw=black,fill=white] (0,0) -- (162:3cm) arc[start angle=162,end angle=198,radius=3cm] -- cycle;
\filldraw[draw=black,fill=white] (0,0) -- (198:3cm) arc[start angle=198,end angle=270,radius=3cm] -- cycle;
\filldraw[draw=black,fill=white] (0,0) -- (270:3cm) arc[start angle=270,end angle=360,radius=3cm] -- cycle;

\node[] at (10:1.8cm) {\textbf{15\%}};
\node[] at (20:1.88cm) {\textbf{Management}};

\node[] at (100:1.35cm) {\textbf{30\%}};
\node[] at (100:1.75cm) {\textbf{Arts}};

\node[] at (190:1.65cm) {\textbf{10\%}};
\node[] at (180:1.75cm) {\textbf{Engineering}};

\node[] at (230:1.75cm) {\textbf{20\%}};
\node[] at (220:1.5cm) {\textbf{Commerce}};

\node[] at (320:2cm) {\textbf{25\%}};
\node[] at (330:1.85cm) {\textbf{Science}};

\node[above] at (0,3.4){\textbf{Percentage of girls enrolled in different}};
\node[above] at (0,3) {\textbf{stream}};
\end{scope}
\end{tikzpicture}

%%https://ch.gateoverflow.in/1156/gate-chemical-2020-ga-question-10
%%https://gateoverflow.in/359808/gate-chemical-2020-ga-question-10
\begin{tikzpicture}[scale = 0.8,font = \bfseries]
\begin{axis}[
ybar=0pt, %space between adjacent bars is 0
enlarge x limits=0.20,
enlarge y limits=false,
legend style={at={(0.20,0.95),ybar=area},
anchor=north,legend columns=1.5},
ylabel= Profit Percentage,
xlabel= Year,
ymajorgrids = true,
xmajorgrids = true,
symbolic x coords={2013,2014,2015,2016,2017,2018},
xtick=data,
ytick={0,10,20,...,80},
bar width=0.45cm,
%style={font=\boldmath},
nodes near coords align={vertical},
%every x tick label/.append style={font=\bfseries},
every y tick label/.append style={font=\boldmath},
%tick label style={font=\boldmath}, 
label style={color=black,font=\bfseries},
width=12cm,
height=10cm,
ymax=70,
ymin=0
]
\addplot
[fill=red!40!orange,draw=black,mark=none]
coordinates {(2013,10) (2014,20) (2015,40) (2016,40) (2017,50) (2018,40)};
\addplot
[fill=blue!20!gray,draw=black]
coordinates {(2013,20) (2014,30) (2015,30) (2016,50) (2017,60) (2018,60)};
\legend{\;Company P\;,Company Q}

\end{axis}
\end{tikzpicture}




%%https://gateoverflow.in/359518/gate-mechanical-2020-set-1-ga-question-10
%%https://me.gateoverflow.in/1711/gate-mechanical-2020-set-1-ga-question-10

\begin{tikzpicture}[scale = 0.8,font = \bfseries]
\begin{axis}[
ybar=0pt, %space between adjacent bars is 0
enlarge x limits=0.18,
enlarge y limits=false,
title={Performance of Schools P, Q, R and S},
legend style={at={(0.18,0.97),ybar=area},
anchor=north,legend columns=-1},
ylabel= {Number of students},
symbolic x coords={School P, School Q, School R, School S},
xtick=data,
ytick={0,100,200,...,800,900,1000},
nodes near coords,
bar width=0.75cm,
ymajorgrids = true,
every node near coord/.append style={font=\boldmath},
nodes near coords align={vertical},
every x tick label/.append style={font=\bfseries},
every y tick label/.append style={font=\boldmath},
%tick label style={font=\boldmath}, 
label style={color=black,font=\bfseries},
width=14cm,
height=10cm,
ymax=800,
ymin=0
]
\addplot
[fill=magenta!40!white,postaction={
        pattern=north west lines
    },draw=black]
coordinates {(School P,500) (School Q,600) (School R,700) (School S,400)};
\addplot
[fill=blue!40!white,postaction={
        pattern=north east lines
    },draw=black]
coordinates {(School P,280) (School Q,330) (School R,455) (School S,240)};
\legend{Appeared\;\;,Passed}
\end{axis}
\end{tikzpicture}


%%https://gateoverflow.in/359759/gate-ece-2020-ga-question-10
%%https://ec.gateoverflow.in/1523/gate-ece-2020-ga-question-10

\begin{tikzpicture}[scale = 0.8,font = \bfseries]
\begin{axis}[
ybar=0pt, %space between adjacent bars is 0
enlarge x limits=0.20,
enlarge y limits=false,
legend style={at={(0.18,0.97),ybar=area},
anchor=north,legend columns=1},
ylabel= Number of students (in thousands),
xlabel= Year,
ymajorgrids = true,
xmajorgrids = true,
symbolic x coords={2014,2015,2016,2017,2018},
xtick=data,
ytick={0,1,2,...,8,9,10},
bar width=0.5cm,
every node near coord/.append style={font=\boldmath},
nodes near coords align={vertical},
every x tick label/.append style={font=\bfseries},
every y tick label/.append style={font=\boldmath},
%tick label style={font=\boldmath}, 
label style={color=black,font=\bfseries},
width=12cm,
height=10cm,
ymax=9,
ymin=0
]
\addplot
[fill=red!40!orange,postaction={
        pattern=north east lines
    },draw=black]
coordinates {(2014,3) (2015,5) (2016,5) (2017,6) (2018,4)};
\addplot
[fill=blue!20!gray,postaction={
        pattern= vertical lines
    }, draw=black]
coordinates {(2014,4) (2015,7) (2016,8) (2017,7) (2018,5)};
\legend{\;School P\;,School Q}

\end{axis}
\end{tikzpicture}



%%https://gateoverflow.in/359710/gate-electrical-2020-ga-question-10
%%https://ee.gateoverflow.in/1483/gate-electrical-2020-ga-question-10

\begin{tikzpicture}[scale = 0.8,font = \bfseries]
\begin{axis}[
ybar=0pt, %space between adjacent bars is 0
enlarge x limits=0.20,
enlarge y limits=false,
title={Revenue and Expenditure (in million rupees) of four  \\ companies P, Q, R and S in 2015},
title style={yshift=1mm,align=center},
legend style={at={(0.5,0.97),ybar=area},
anchor=north,legend columns=-1},
ylabel= Revenue/Expenditure (in million rupees),
symbolic x coords={Company P, Company Q, Company R, Company S},
xtick=data,
ytick={0,5,10,...,50,55,60},
bar width=0.60cm,
ymajorgrids = true,
every node near coord/.append style={font=\boldmath},
nodes near coords align={vertical},
every x tick label/.append style={font=\bfseries},
every y tick label/.append style={font=\boldmath},
%tick label style={font=\boldmath}, 
label style={color=black,font=\bfseries},
width=12cm,
height=10cm,
ymax=55,
ymin=0
]
\addplot
[fill=magenta!40!white,postaction={
        pattern=north west lines
    },draw=black]
coordinates {(Company P,35) (Company Q,45) (Company R,30) (Company S,40)};
\addplot
[fill=blue!40!white,postaction={
        pattern=north east lines
    },draw=black]
coordinates {(Company P,25) (Company Q,35) (Company R,40) (Company S,50)};
\legend{Revenue\;\;,Expenditure}
\end{axis}
\end{tikzpicture}

%https://gateoverflow.in/313419/gate2017-ce-2-ga-10
\begin{tikzpicture}
            \tikzstyle{nodep}=[circle,draw,scale=.5,fill=red!50]
            
            \draw[-] (0,6)--(12.5,6)--(12.5,0)--(0,0)--(0,6); 
    
             \draw (0,-.2) node {0};
             \draw (1,-.2) node {5};
             \draw (2,-.2) node {10};
             \draw (3,-.2) node {15};
             \draw (4,-.2) node {20};
             \draw (5,-.2) node {25};
             \draw (6,-.2) node {30};
             \draw (7,-.2) node {35};
             \draw (8,-.2) node {40};
             \draw (9,-.2) node {45};
             \draw (10,-.2) node {50};
             \draw (11,-.2) node {55};
             \draw (12,-.2) node {60};
             
            \foreach \x in {1,2,...,12}
            {
                \draw[dotted] (\x,0)--(\x,6);
            }
            \foreach \x in {.5,1.5,...,11.5}
            {
                \draw[-] (\x,0)--(\x,.1);
            }
            \foreach \x in {.5,1.5,...,11.5}
            {
                \draw[-] (\x,6)--(\x,5.9);
            
            }
            \draw (-.2,.5) node {0};
            \draw (-.2,1.5) node {1};
            \draw (-.2,2.5) node {2};
            \draw (-.2,3.5) node {3};
            \draw (-.2,4.5) node {4};
            \draw (-.2,5.5) node {5};
            
             \foreach \y in {.5,...,5.5}
            {
                \draw[-] (0,\y)--(12.5,\y);
            }
             \foreach \y in {1,...,5}
            {
                \draw[-] (0,\y)--(.1,\y);
            }
             \foreach \y in {1,...,5}
            {
                \draw[-] (12.5,\y)--(12.4,\y);
            }
            
            %ground_floor
            \node[nodep] at (.2,.5) {};
            \node[nodep] at (.84,.5) {};
            \node[nodep] at (1,.5) {};
            \node[nodep] at (1.84,.5) {};
            \node[nodep] at (2.2,.5) {};
            \node[nodep] at (3,.5) {};
            \node[nodep] at (4,.5) {};
            \node[nodep] at (4.44,.5) {};
            \node[nodep] at (4.6,.5) {};
            \node[nodep] at (5.4,.5) {};
            \node[nodep] at (6,.5) {};
            \node[nodep] at (6.16,.5) {};
            \node[nodep] at (6.84,.5) {};
            \node[nodep] at (7.4,.5) {};
            \node[nodep] at (8.4,.5) {};
            \node[nodep] at (8.8,.5) {};
            \node[nodep] at (9.63,.5) {};
            \node[nodep] at (9.8,.5) {};
            \node[nodep] at (10.4,.5) {};
            \node[nodep] at (11,.5) {};
            \node[nodep] at (11.63,.5) {};
            
            %first_floor
            \node[nodep] at (11.2,1.5) {};
            \node[nodep] at (11.37,1.5) {};
            
            %second_floor
            \node[nodep] at (4.2,2.5) {};
            \node[nodep] at (5.6,2.5) {};
            \node[nodep] at (7.63,2.5) {};
            \node[nodep] at (7.8,2.5) {};
            
            %third_floor
            \node[nodep] at (2.46,3.5) {};
            \node[nodep] at (2.63,3.5) {};
            \node[nodep] at (2.8,3.5) {};
            \node[nodep] at (8.6,3.5) {};
            \node[nodep] at (10,3.5) {};
            \node[nodep] at (10.16,3.5) {};
            
            %fourth_floor
            \node[nodep] at (1.26,4.5) {};
            \node[nodep] at (1.43,4.5) {};
            \node[nodep] at (1.6,4.5) {};
            \node[nodep] at (3.26,4.5) {};
            \node[nodep] at (3.43,4.5) {};
            \node[nodep] at (3.6,4.5) {};
            \node[nodep] at (3.78,4.5) {};
            \node[nodep] at (4.83,4.5) {};
            \node[nodep] at (5,4.5) {};
            \node[nodep] at (5.17,4.5) {};
            \node[nodep] at (7,4.5) {};
            \node[nodep] at (7.17,4.5) {};
            \node[nodep] at (9,4.5) {};
            \node[nodep] at (9.17,4.5) {};
            \node[nodep] at (9.34,4.5) {};
            \node[nodep] at (10.6,4.5) {};
            \node[nodep] at (10.77,4.5) {};
            \node[nodep] at (11.83,4.5) {};
            \node[nodep] at (12,4.5) {};
            
            %fifth_floor
            \node[nodep] at (.43,5.5) {};
            \node[nodep] at (.6,5.5) {};
            \node[nodep] at (2,5.5) {};
            \node[nodep] at (5.77,5.5) {};
            \node[nodep] at (6.43,5.5) {};
            \node[nodep] at (6.6,5.5) {};
            \node[nodep] at (8,5.5) {};
            \node[nodep] at (8.17,5.5) {};
            
            \draw (-1,3) node[rotate=90] {Floor Number};
            \draw (5.5,-1) node {Time (min)};
            
       \end{tikzpicture}







%% https://me.gateoverflow.in/1818/gate2020-me-2-ga-5

\begin{tikzpicture}[transform shape,scale = 0.8]
\draw[thick,rounded corners] (0,0) rectangle (2,1);
\draw[thick,rounded corners] (3,0) rectangle (5,1);
\draw[thick,rounded corners] (6,0) rectangle (8,1);
\draw[thick,rounded corners] (9,0) rectangle (11,1);
\draw[thick,rounded corners] (12,0) rectangle (14,1);
\draw[thick,rounded corners] (15,0) rectangle (17,1);
\node[left] at (1.25,0.5){$\text{P}$};
\node[left] at (4.25,0.5){$\text{Q}$};
\node[left] at (7.25,0.5){$\text{R}$};
\node[left] at (10.25,0.5){$\text{S}$};
\node[left] at (13.25,0.5){$\text{T}$};
\node[left] at (16.90,0.5){$\text{Customers}$};
\draw[-{Triangle[width=14pt,length=6pt,]}, line width=6pt,gray!65!white](2.20,0.5) -- (2.80, 0.5);
\draw[-{Triangle[width=14pt,length=6pt,]}, line width=6pt,gray!65!white](5.20,0.5) -- (5.80, 0.5);
\draw[-{Triangle[width=14pt,length=6pt,]}, line width=6pt,gray!65!white](8.20,0.5) -- (8.80, 0.5);
\draw[-{Triangle[width=14pt,length=6pt,]}, line width=6pt,gray!65!white](11.20,0.5) -- (11.80, 0.5);
\draw[-{Triangle[width=14pt,length=6pt,]}, line width=6pt,gray!65!white](14.20,0.5) -- (14.80, 0.5);
\end{tikzpicture}


%%https://gateoverflow.in/333232/gate-cse-2020-question-ga-9

\begin{tikzpicture}[scale = 0.8]
\draw[help lines, color=gray!40, dashed] (-3,-3) grid (10,7);
\draw[->,very thick] (-2,0)--(8,0) node[right]{$\text{X-axis}$};
\draw[->,very thick] (0,-2)--(0,6) node[above]{$\text{Y-axis}$};
\draw [] (1,-1) -- (5,5);
\draw [] (6,-1) -- (2,5);
\path[right](2.10,0.60)  edge  [bend left]  node {$\alpha$} (2.15,0);
\path[left](4.90,0.65)  edge  [bend right]  node {$\beta$} (4.80,0);
\draw[] (3.25,2.40) -- (3.5,2.20) -- (3.75,2.40);
\node[right] at (4.5,4) {$L_{1}$};
\node[left] at (2.5,4) {$L_{2}$};
\end{tikzpicture}



%https://ee.gateoverflow.in/1484/gate2020-ee-ga-9

\begin{tikzpicture}
\node at (0.0,1.7) {C};
\node at (0.0,-1.4) {O};
\node at (-2.5,-0.9) {A};
\node at (2.5,-0.9) {B};
\node [draw , semicircle, minimum size= 2.2cm, line width=0.5mm] at (0.0,0.0) {};
\draw[-] (0.0,-0.9) -- (0.0,1.3);
\draw[-, color =blue, line width=0.5mm] (-2.2,-0.9) -- (0.0,1.3);
\draw[-, color =blue, line width=0.5mm] (2.2,-0.9) -- (0.0,1.3);
\draw[-] (0.2,-0.9) -- (0.2,-0.7);
\draw[-] (0.0,-0.7) -- (0.2,-0.7);
\end{tikzpicture}



%https://me.gateoverflow.in/1814/gate2020-me-2-ga-9

\begin{tikzpicture}[>=stealth',shorten >=1pt,auto,thick,node distance=2.8cm]
\node at (-0.1,1.9) {\text{\small{(X,Y)}}};
\node at (2.4,-0.3) {\text{\small{Z}}};
\node at (0.1,-0.2) {\text{\small{O}}};

\draw[->,line width=0.4mm] (0.0,0.0) -- (2.5,0.0);
\draw[->,line width=0.4mm] (0.4,-0.4) -- (0.4,2.5);
\draw[-, color=blue,line width=0.4mm] (0.4,0.0) -- (1.2,2.1);
\end{tikzpicture}



%%https://gateoverflow.in/333233/gate-cse-2020-question-ga-8

\begin{tikzpicture}[scale = 0.8,transform shape]
\draw[thick](0,0) circle (5);
\draw[thick](0,0) circle (3.75);
\draw[fill=teal!70!white,draw=teal,thick](0,-4.36) circle (0.60);
\draw[fill=teal!70!white,draw=teal,thick](1.20,-4.20) circle (0.60);
\draw[fill=teal!70!white,draw=teal,thick](2.32,-3.70) circle (0.60);
\draw[fill=teal!70!white,draw=teal,thick](3.25,-2.93) circle (0.60);
\draw[fill=teal!70!white,draw=teal,thick](3.92,-1.91) circle (0.60);
\draw[fill=teal!70!white,draw=teal,thick](4.30,-0.77) circle (0.60);
\draw[fill=teal!70!white,draw=teal,thick](4.35,0.42) circle (0.60);
\draw[fill=teal!70!white,draw=teal,thick](4.08,1.60) circle (0.60);
\draw[fill=teal!70!white,draw=teal,thick](3.48,2.66) circle (0.60);

\draw[draw=teal, ultra thick,line width=3pt] (2.10,3.82) -- (2.40,3.63);
\draw[draw=teal, ultra thick,line width=3pt] (1.40,4.12) -- (1.70,4);
\draw[draw=teal, ultra thick, line width=3pt] (0.65,4.37) -- (0.95,4.28);

\draw[->,>=latex, ultra thick,teal!80!white] (0,0) -- (-4.20,2.75);
\draw[->,>=latex,ultra thick,teal!80!white] (0,0) -- (-2.85,-2.45);

\node[above] at (-1.5,1.20) {\Huge{$b$}};
\node[left] at (-1.5,-0.90) {\Huge{$a$}};
\end{tikzpicture}







%%https://gateoverflow.in/333231/gate-cse-2020-question-ga-10
\definecolor{mintbg}{rgb}{.63,.79,.95}

    \begin{tikzpicture}[scale= 0.5,transform shape]
        \begin{axis}[
        ybar stacked,
        ymin=0,
        ymax=900,
        ytick={0,100,...,900},
            symbolic x coords={2014,2015,2016,2017,2018},
           xtick=data,
             ylabel near ticks,
    xlabel near ticks,
              xlabel={Year},
              ymajorgrids = true,
   ylabel={Revenue (In million rupees)} ]
            \addplot[ybar,fill=mintbg,bar width=15pt] coordinates {
                (2014, 500)
                (2015, 700)
                (2016, 800)
                (2017, 600)
                (2018, 400)
               
            };
        \end{axis}
    \end{tikzpicture}








%%https://me.gateoverflow.in/1712/gate-mechanical-2020-set-1-ga-question-9
%answer image
\begin{tikzpicture}[>=latex,shorten >=1pt,auto, semithick,node distance=2.8cm,scale = 1]
\node[] at (-0.2,3.4) {$y$};
\node[left] at (0,2.5) {$1$};
\node[left] at (0,-0.2) {$0$};
\node[below] at (2.0,0) {$1$};
\node[] at (3.1,-0.2) {$x$};
\node[] at (1.65,0.6) {$x^{m}$};
\node[] at (1.1,1.6) {$x^{1/m}$};
\draw[->,line width=0.5mm] (0.0,-0.3) -- (0.0,3.5);
\draw[->,line width=0.5mm] (-0.3,0.0) -- (3.0,0.0);
\draw[-,line width=0.45mm,dashed](2.0,0.0) -- (2.0,2.5) -- (0.0,2.5);
\path[-,line width=0.45mm]
[teal] (0.0,0.0) edge [bend left = 35]  node  {} (2.0,2.52)
[teal] (0.0,0.0) edge [bend right = 35]  node  {} (1.98,2.53);
\end{tikzpicture}

%%https://ch.gateoverflow.in/1159/gate-chemical-2020-ga-question-7
%answer image 1
\begin{tikzpicture}[scale = 0.6]
    % equidistant points and arc
    \foreach \x [count=\p] in {0,...,7} {
        \node[shape=circle,fill=black, scale=0.4] (\p) at (-\x*45:2) {};};
    \draw (1) arc (0:360:2);

\node[above] at (0,2.15) {$\textbf{P}$};    
\node[below] at (0,-2.15) {$\textbf{R}$}; 
\node[right] at (1.55,1.5) {$\textbf{W}$};    
\node[left] at (-1.55,1.5) {$\textbf{T}$}; 
\node[right] at (2.12,0.08) {$\textbf{U}$};   
\node[left] at (-2.12,0.08) {$\textbf{V}$};   
\node[right] at (1.68,-1.5) {$\textbf{Q}$};  
\node[left] at (-1.68,-1.5) {$\textbf{S}$}; 

\end{tikzpicture}
%answer image 2
\begin{tikzpicture}[scale = 0.6]
    % equidistant points and arc
    \foreach \x [count=\p] in {0,...,7} {
        \node[shape=circle,fill=black, scale=0.4] (\p) at (-\x*45:2) {};};
    \draw (1) arc (0:360:2);

\node[above] at (0,2.15) {$\textbf{Q}$};    
\node[below] at (0,-2.15) {$\textbf{R}$}; 
\node[right] at (1.55,1.5) {$\textbf{W}$};    
\node[left] at (-1.55,1.5) {$\textbf{T}$}; 
\node[right] at (2.12,0.08) {$\textbf{U}$};   
\node[left] at (-2.12,0.08) {$\textbf{V}$};   
\node[right] at (1.68,-1.5) {$\textbf{P}$};  
\node[left] at (-1.68,-1.5) {$\textbf{S}$}; 
\end{tikzpicture}


%%https://me.gateoverflow.in/1816/gate2020-me-2-ga-7
%question-image
\begin{tikzpicture}[scale = 0.6]
\draw[] (0,0) -- (8,0) -- (4.20,-4.5) -- (0,0);
\draw[] (0,-2.75) -- (8,-2.75) -- (4.20,1.5) -- (0,-2.75);
\node[] at (4.1,0.5){\large{$5$}};
\node[] at (4.1,-3.25){\large{$9$}};
\node[] at (1.25,-0.5){\large{$t$}};
\node[] at (1.25,-2.25){\large{$?$}};
\node[] at (6.70,-0.5){\large{$h$}};
\node[] at (6.70,-2.25){$x$};
\node[] at (4.1,-1.45){\large{$n$}};
\end{tikzpicture}
%%https://me.gateoverflow.in/1816/gate2020-me-2-ga-7
%answer-image1
\begin{tikzpicture}[scale = 0.6]
\draw[] (0,0) -- (8,0) -- (4.20,-4.5) -- (0,0);
\draw[] (0,-2.75) -- (8,-2.75) -- (4.20,1.5) -- (0,-2.75);
\node[] at (4.1,0.5){\large{$5$}};
\node[] at (4.1,-3.25){\large{$9$}};
\node[] at (1.25,-0.5){\large{$20$}};
\node[] at (1.25,-2.25){\large{$?$}};
\node[] at (6.70,-0.5){\large{$8$}};
\node[] at (6.70,-2.25){$24$};
\node[] at (4.1,-1.45){\large{$14$}};
\end{tikzpicture}
%%https://me.gateoverflow.in/1816/gate2020-me-2-ga-7
%answer-image2
\begin{tikzpicture}[node distance=25mm,auto,scale = 0.8]
\node[state,minimum size= 0.5mm] (q0) {$14$};
\node[state,minimum size= 0.5mm] (q1)[above of  = q0] {$5$};
\node[state,minimum size= 0.5mm] (q2)[below of  = q0] {$9$};
\node[state,minimum size= 0.5mm] at (-3,1.25) (q3) {$20$};
\node[state,minimum size= 0.5mm] at (-3,-1.25)(q4) {$4$};
\node[state,minimum size= 0.5mm] at (3,1.25)(q5) {$8$};
\node[state,minimum size= 0.5mm] at (3,-1.25)(q6) {$24$};
\path[] (q0) edge[above left,pos = 0.90,sloped] node {$14-9$} (q1);
\path[] (q0) edge[below right,pos = 0.90,sloped,rotate=180] node {$14-5$} (q2);
\path[] (q0) edge[above right,pos = 0.85,sloped] node {$14+6$} (q3);
\path[] (q0) edge[above right,pos = 0.85,sloped] node {$14-10$} (q4);
\path[] (q0) edge[above right,pos = 0.25,sloped] node {$14-6$} (q5);
\path[] (q0) edge[above right,pos = 0.25,sloped] node {$14+10$} (q6);
\end{tikzpicture}



%%https://civil.gateoverflow.in/1668/gate2020-ce-1-ga-10
%question image
\begin{tikzpicture}
\tikzset{
     lines/.style={draw=none},
}
\pie[text = legend,style={lines},color={blue!40!cyan,red!60!white, yellow!60!green, violet!50!white,cyan!70!white,orange!70!white,{rgb:red,1;green,2;blue,3}}]
{5/Health (5\%),
 10/Transport (10\%),
 15/Household Items (15\%),
 15/Education (15\%),
 10/Leisure (10\%),
 20/House rent (20\%),
 25/Others (25\%)
 }
\filldraw[draw=none,fill=yellow!60!green] (0,0) -- (-27:3cm) arc[start angle=-27,end angle=27,radius=3cm] -- cycle;
\filldraw[draw=none,fill=red!60!white] (0,0) -- (27:3cm) arc[start angle=27,end angle=63,radius=3cm] -- cycle;
\filldraw[draw=none,fill=blue!40!cyan] (0,0) -- (63:3cm) arc[start angle=63,end angle=81,radius=3cm] -- cycle;
\filldraw[draw=none,fill={rgb:red,1;green,2;blue,3}] (0,0) -- (81:3cm) arc[start angle=81,end angle=171,radius=3cm] -- cycle;
\filldraw[draw=none,fill=orange!70!white] (0,0) -- (171:3cm) arc[start angle=171,end angle=243,radius=3cm] -- cycle;
\filldraw[draw=none,fill=cyan!70!white] (0,0) -- (243.2:3cm) arc[start angle=243,end angle=279,radius=3cm] -- cycle;
\filldraw[draw=none,fill=violet!50!white] (0,0) -- (279:3cm) arc[start angle=279,end angle=333,radius=3cm] -- cycle;

\node[right] at (6:1.85cm) {\text{15\%}};

\node[above] at (40:2cm) {\text{10\%}};

\node[above] at (70:2cm) {\text{5\%}};

\node[left] at (120:2cm) {\text{25\%}};

\node[left] at (200:2cm) {\text{20\%}};

\node[below] at (260:2cm) {\text{10\%}};

\node[below] at (310:2cm) {\text{15\%}};


\end{tikzpicture}


%https://me.gateoverflow.in/1712/gate2020-me-1-ga-9
%answer image 1
 \begin{tikzpicture}[>=latex,shorten >=1pt,auto, semithick,node distance=2.8cm,scale = 1]
\node[] at (-0.2,3.4) {$y$};
\node[left] at (0,2.5) {$1$};
\node[left] at (0,-0.2) {$0$};
\node[below] at (2.0,0) {$1$};
\node[] at (3.1,-0.2) {$x$};
\node[] at (1.65,0.6) {$x^{m}$};
\node[] at (1.1,1.6) {$x^{1/m}$};
\draw[->,line width=0.5mm] (0.0,-0.3) -- (0.0,3.5);
\draw[->,line width=0.5mm] (-0.3,0.0) -- (3.0,0.0);
\draw[-,line width=0.45mm,dashed](2.0,0.0) -- (2.0,2.5) -- (0.0,2.5);
\path[-,line width=0.45mm]
[teal] (0.0,0.0) edge [bend left = 35]  node  {} (2.0,2.52)
[teal] (0.0,0.0) edge [bend right = 35]  node  {} (1.98,2.53);
\end{tikzpicture}


%answer image 2
 \begin{tikzpicture}[>=latex,shorten >=1pt,auto, semithick,node distance=2.8cm,scale = 1]
\node[] at (-0.2,3.4) {$y$};
\node[left] at (0,2.5) {$1$};
\node[left] at (0,-0.2) {$0$};
\node[below] at (2.0,0) {$1$};
\node[] at (3.1,-0.2) {$x$};
\node[] at (1.60,0.3) {$x^{1/m}$};
\node[] at (1.1,1.6) {$x^{m}$};
\draw[->,line width=0.5mm] (0.0,-0.3) -- (0.0,3.5);
\draw[->,line width=0.5mm] (-0.3,0.0) -- (3.0,0.0);
\draw[-,line width=0.45mm,dashed](2.0,0.0) -- (2.0,2.5) -- (0.0,2.5);
\path[-,line width=0.45mm]
[teal] (0.0,0.0) edge [bend left = 35]  node  {} (2.0,2.52)
[teal] (0.0,0.0) edge [bend right = 35]  node  {} (1.98,2.53);
\end{tikzpicture}

%answer image 3
\begin{tikzpicture}[>=stealth',shorten >=1pt,auto, semithick,node distance=2.8cm]
\node at (-0.2,3.4) {y};
\node at (-0.2,2.5) {1};
\node at (-0.2,-0.2) {0};
\node at (2.0,-0.2) {1};
\node at (3.1,-0.2) {x};
\node at (1.5,0.3) {$x^{1/m}$};
\node at (0.6,1.0) {$x^{m}$};

\draw[->,line width=0.5mm] (0.0,-0.3) -- (0.0,3.5);
\draw[->,line width=0.5mm] (-0.3,0.0) -- (3.0,0.0);
\draw[-,line width=0.5mm,dashed](2.0,0.0) -- (2.0,2.5);
\draw[-,line width=0.5mm,dashed](0.0,2.5) --(2.0,2.5);

\path [-,line width=0.5mm]
[teal] (0.0,0.0) edge [bend right = 15]  node  {} (2.0,2.5)
[teal] (0.0,0.0) edge [bend right = 35]  node  {} (1.98,2.53);
\end{tikzpicture}

%answer image 4
\begin{tikzpicture}[>=stealth',shorten >=1pt,auto, semithick,node distance=2.8cm]
\node at (-0.2,3.4) {y};
\node at (-0.2,2.5) {1};
\node at (-0.2,-0.2) {0};
\node at (2.0,-0.2) {1};
\node at (3.1,-0.2) {x};
\node at (1.2,1.2) {$x^{m}$};
\node at (0.5,2.2) {$x^{1/m}$};

\draw[->,line width=0.5mm] (0.0,-0.3) -- (0.0,3.5);
\draw[->,line width=0.5mm] (-0.3,0.0) -- (3.0,0.0);
\draw[-,line width=0.5mm,dashed](2.0,0.0) -- (2.0,2.5);
\draw[-,line width=0.5mm,dashed](0.0,2.5) --(2.0,2.5);

\path[-,line width=0.5mm]
[teal] (0.0,0.0) edge [bend left = 15]  node  {} (2.0,2.5)
[teal] (0.0,0.0) edge [bend left = 35]  node  {} (2.0,2.52);
\end{tikzpicture}












%https://gateoverflow.in/118343/gate-cse-2017-set-2-question-6?show=118224
\begin{tikzpicture}[scale = 0.4,transform shape]

\draw[very thick] (0,0) ellipse (12cm and 6cm);

\draw[rounded corners,very thick] (-8,-4) rectangle (8,4);

\draw[rounded corners,very thick] (-6,-2.75) rectangle (6,2.75);

\draw[rounded corners,very thick] (-4.5,-1.5) rectangle (4.5,1.5);
\draw[very thick] (0,0) ellipse (3cm and 0.60cm);

\node[] at (0,0) {\huge{$\textbf{LR(0)}$}};
\node[right] at (-4.25,1) {\huge{$\textbf{SLR(1)}$}};
\node[right] at (-5.75,2.20) {\huge{$\textbf{LALR(1)}$}};
\node[right] at (-7.75,3.45) {\huge{$\textbf{CLR(1)}$}};
\node[above] at (-0.5,4.25) {\huge{$\textbf{Unambiguous CFG}$}};
\node[above] at (-0.5,6.5) {\Huge{$\textbf{LR Grammar Hierarchy}$}};
\end{tikzpicture}
    %https://gateoverflow.in/333196/gate-cse-2020-question-35?show=333279

\begin{tikzpicture}[scale = 0.4,font = \bfseries, transform shape]
\Large
\draw (-12,-2.3) -- (-12,1.8);
\draw (10,-2.3) -- (10,1.8);
\draw (0,1.385) -- (1.5,1.295);
\draw (1.5,1.295) -- (2.25,1.25);
\draw (2.25,1.25) -- (-2.25,0.5);
\draw (-2.25,0.5) -- (-10.5,-0.875);
\draw (-10.5,-0.875) -- (8.25,-2);

\filldraw [red] (0,1.385) circle (2pt);
\filldraw [red] (1.5,1.295) circle (2pt);
\filldraw [red] (2.25,1.25) circle (2pt);
\filldraw [red] (-2.25,0.5) circle (2pt);
\filldraw [red] (-10.5,-0.875) circle (2pt);
\filldraw [red] (8.25,-2) circle (2pt);

\node at (-11,-2.5) {Lower Cylinder Number};
\node at (9,-2.5) {Higher Cylinder Number};
\node at (-10.5,2) {(S,30)};
\node at (-2.25,2) {(Q,85)};
\node at (1.20,2) {(R,110)};
\node at (3.05,2) {(T,115)};
\node at (8.25,2) {(P,155)};
\node[text=blue] at (-0.5,3) {Current Position};
\node[text=blue] at (-0.5,2) {100};

\end{tikzpicture}
%%%%https://gateoverflow.in/333185/gate-cse-2020-question-46#349100


\begin{tikzpicture}[level 1/.style={sibling distance=30mm},level 2/.style={sibling distance=48mm},level 3/.style={sibling distance=53mm},level 4/.style={sibling distance=33mm},level 5/.style={sibling distance=28mm},level distance = 30mm, scale = 0.8]

\node[]{f2(5)}
         child {node[anchor=south,draw,
    rounded corners=0.3cm,
    minimum height=0.5cm,
    text width=2cm,
    align =center](5){$\text{i=0+f1(5)}$}edge from parent[thick,solid]}
         child {node[]{$\text{f2(4)}$}edge from parent[thick,solid]
               child {node[anchor=west,draw,solid,
    rounded corners=0.3cm,
    minimum height=0.5cm,
    text width=2cm,
    align =center](4){$\text{i=i+f1(4)}$}edge from parent[thick,solid]}
               child {node {$\text{f2(3)}$}
                     child {node[anchor=west,draw,solid,
    rounded corners=0.3cm,
    minimum height=0.5cm,
    text width=2cm,
    align =center](3) {$\text{i=i+f1(3)}$}edge from parent[thick,solid]}
                     child {node {$\text{f2(2)}$}
                           child {node[anchor=south,draw,solid,
    rounded corners=0.3cm,
    minimum height=0.5cm,
    text width=2cm,
    align =center](2) {$\text{i=i+f1(2)}$}edge from parent[thick,solid]}
                           child {node {$\text{f2(1)}$}
                                 child {node[anchor=south,draw,solid,
    rounded corners=0.3cm,
    minimum height=0.5cm,
    text width=2cm,
    align =center](1) {$\text{i=i+f1(1)}$}edge from parent[thick,solid]
                                 child{node[blue]{\boxed{i=55}}edge from parent[->,thick,dashed,blue]edge from parent[thick]}}
                                 child {node {$\text{f2(0)}$}
                                       child{node[align=center]{ if condition \\fails}edge from parent[thick,solid]}}}}}};
\draw [->,thick,blue,dashed] (5.south) ++(0,0) |- (4.west) node[pos=0.25,fill=white,inner sep=0] {\boxed{{i=5}}};
\draw [->,thick,blue,dashed] (4.south) ++(0,0) |- (3.west) node[pos=0.25,fill=white,inner sep=0,align=left] {\boxed{{i=14}}};
\draw [->,thick,blue,dashed] (3.south) ++(0,0) |- (2.west) node[pos=0.25,fill=white,inner sep=0,align=left] {\boxed{{i=26}}};
\draw [->,thick,blue,dashed] (2.south) ++(0,0) |- (1.west) node[pos=0.25,fill=white,inner sep=0,align=left] {\boxed{{i=40}}};

\end{tikzpicture}


%%%https://gateoverflow.in/118194/gate-cse-2017-set-1-question-14?show=120967#a120967
\begin{tikzpicture}

%%Big colorful rectangle

\node[draw=violet!10,thick,fill=violet!10,
    minimum width=5cm,
    minimum height=8cm] at (0,0){};
\node[draw=blue!11,thick,fill=blue!11,
    minimum width=5cm,
    minimum height=8cm] at (6,0){};
\node[draw=orange!15,thick,fill=orange!15,
    minimum width=5cm,
    minimum height=8cm] at (0,-8.5){};
\node[draw=pink!40,thick,fill=pink!40,
    minimum width=5cm,
    minimum height=7cm] at (6,-8){};
\node[draw=yellow!15,thick,fill=yellow!15,
    minimum width=3.3cm,
    minimum height=5cm] at (2.8,-9.5){};
\node[draw=cyan!15,thick,fill=cyan!15,
    minimum width=5cm,
    minimum height=1.4cm] at (3.2,-0.8){};

%%%%small rectangle    

\node[rectangle,draw,very thick,
    fill =gray!20,
    minimum width = 2.2cm, 
    minimum height = 0.7cm] (closed) at (3,4) {CLOSED};
\node[rectangle,draw,very thick,
    fill =violet!20,
    minimum width = 2.2cm, 
    minimum height = 0.7cm] (listen) at (-0.5,1.5) {LISTEN};
\node[rectangle,draw,very thick,
    fill =violet!20,
    minimum width = 2.2cm, 
    minimum height = 0.7cm] (synr) at (-0.5,-1) {SYN-RECEIVED};
\node[rectangle,draw,very thick,
    fill =blue!30,
    minimum width = 2.2cm, 
    minimum height = 0.7cm] (syns) at (6.5,-1) {SYN-SENT};
\node[rectangle,draw,very thick,
    fill=green!30,
    minimum width = 2.2cm, 
    minimum height = 0.7cm] (estb) at (3,-4.2) {ESTABLISHED};
    
\node[rectangle,draw,very thick,
    fill=orange!30,
    minimum width = 2.2cm, 
    minimum height = 0.7cm] (finw1) at (0.5,-7) {FIN-WAIT-1};
\node[rectangle,draw,very thick,
    fill=orange!30,
    minimum width = 2.2cm, 
    minimum height = 0.7cm] (finw2) at (-1,-9.5) {FIN-WAIT-2};
\node[rectangle,draw,very thick,
    fill=yellow!40,
    minimum width = 2.2cm, 
    minimum height = 0.7cm] (closing) at (2.8,-9.5) {CLOSING};
\node[rectangle,draw,very thick,
    fill=yellow!40,
    minimum width = 2.2cm, 
    minimum height = 0.7cm] (twait) at (2.8,-12) {TIME-WAIT};
    
\node[rectangle,draw,very thick,
    fill=pink!90,
    minimum width = 2.2cm, 
    minimum height = 0.7cm] (cwait) at (6.2,-7.3) {CLOSE-WAIT};
\node[rectangle,draw,very thick,
    fill=pink!90,
    minimum width = 2.2cm, 
    minimum height = 0.7cm] (lack) at (6.2,-9.7) {LAST-ACK};
    
\node[](left)at(-0.5,-3.8) {\footnotesize{Open-Responder Sequence}};
\node[](left)at(6.6,-3.8) {\footnotesize{Open-Initiator Sequence}};
\node[](left)at(-0.6,-4.8) {\footnotesize{Close-Initiator Sequence}};
\node[](left)at(6.4,-4.8) {\footnotesize{Close-Responder Sequence}};
\node[](left)at(3.1,-7.2) {\footnotesize{Simultaneous Close}};

%%%connection between rectangle

\draw [-Triangle,thick,align=center]
(closed.south) - ++(0,-0.8) -| (listen.north) node[pos=0.25,inner sep=0,fill=violet!10] {Passive Open\\ Set UP TCB};
\draw [-Triangle,thick,align=center] (closed.east) -| (syns.north) node[pos=0.7,inner sep=0,fill=blue!11] {Active Open\\ Set UP TCB\\ Send SYN};
\draw [-Triangle,thick,align=center] (listen.south) -- (synr.north) node[pos=0.5,inner sep=0,fill=violet!10] {Receive SYN\\ Send SYN+ACK};
\draw [-Triangle,thick,align=center] (syns.west) -- (synr.east) node[pos=0.5,inner sep=0,fill=cyan!15,label=above:\footnotesize{Simultaneous Open}] {Receive SYN\\ Send ACK};
\draw [-Triangle,thick,align=center] (synr.south) - ++(0,-1.5) -| (estb.north west) node[pos=0.05,inner sep=0,fill=violet!10,yshift=0.6cm] {Receive ACK};
\draw [-Triangle,thick,align=center] (syns.south) - ++(0,-1.5) -| (estb.north east) node[pos=0,inner sep=0,fill=blue!11,yshift=0.6cm] {Receive SYN+ACK\\ Send ACK};

\draw [-Triangle,thick,align=center]
(estb.south) - ++(0,-0.9) -| (finw1.north) node[pos=0.7,inner sep=0,fill=orange!15,yshift=-0.6] {Close, Send FIN};
\draw [-Triangle,thick,align=center] (estb.south east) - ++(0,-0.9) -| (cwait.north) node[pos=0.5,inner sep=0,fill=pink!40,yshift=-0.6cm] {Receive Fin\\ Send ACK};

\draw [-Triangle,thick,align=center]
(finw1.south) - ++(0,-0.4) -| (finw2.north) node[pos=0.7,inner sep=0,fill=orange!15,yshift=-0.4] {Receive ACK\\ for FIN};
\draw [-Triangle,thick,align=center] (finw1.south east) - ++(0,-0.4) -| (closing.north) node[pos=0.7,inner sep=0,fill=yellow!15,yshift=-0.1cm] {Receive Fin\\ Send ACK};

\draw [-Triangle,thick,align=center] (finw2.south) |- (twait.west) node[pos=0.3,inner sep=0,fill=orange!15] {Receive FIN\\ Send ACK};
\draw [-Triangle,thick,align=center] (closing.south) -- (twait.north) node[pos=0.5,inner sep=0,fill=yellow!15] {Receive ACK\\ for FIN};

\draw [-Triangle,thick,align=center] (cwait.south) -- (lack.north) node[pos=0.5,inner sep=0,fill=pink!40] {Wait for Application\\ Close, Send FIN};

\draw [-Triangle,thick,align=center] (lack.south) -- (6.2,-11) -| (8.4,-11)--(8.4,5)-| (closed.north) node[pos=0,inner sep=0, fill=pink!40, xshift=-2.1cm,yshift=-15.45cm] {Receive ACK for FIN};


\draw [-Triangle,thick,align=center]
(twait.east) -| (8.4,-11) node[pos=0.3,inner sep=0,fill=white,xshift=-0.4] {Timer Expiration};
   
\end{tikzpicture}
\input{function-plots}
%https://tikz.dev/tikz-graphs

%https://gateoverflow.in/166239/ugc-net-cse-november-2017-part-2-question-5

\begin{tikzpicture}[scale = 0.7,transform shape,minimum size=1pt]
\tikzstyle{mynode}=[circle,fill,inner sep=0pt,minimum size=1pt]

\node[mynode] (e) at (0.8,3)[label={left:E}]  {};
\node[mynode] (a) at (3,4.3)[label={above:A}] {};
\node[mynode] (d) at (3,1.7) [label={below:D}]{};
\node[mynode] (c) at (7,4.3)[label={above:C}] {};
\node[mynode] (b) at (7,1.7) [label={below:B}]{};
\node[mynode] (f) at (9.3,3)[label={right:F}] {};
\node[mynode,minimum size=5pt] (g) at (5,3)[label={below:G}] {};

\draw [-] (e) edge node[above] {$4$} (a);
\draw [-] (a) edge node[left] {$2$} (d);
\draw [-] (e) edge node[below] {$5$} (d);
\draw [-] (a) edge node[above] {$6$} (c);

\draw [-] (a) edge node[above] {$3$} (g);
\draw [-] (d) edge node[above] {$4$} (g);
\draw [-] (g) edge node[above] {$3$} (b);
\draw [-] (d) edge node[above] {} (b);

\draw [-] (c) edge node[above] {$5$} (f);
\draw [-] (b) edge node[below] {$4$} (f);
\draw [-] (c) edge node[right] {$4$} (b);
\draw [-] (c) edge node[above] {$2$} (g);


\end{tikzpicture}




%https://gateoverflow.in/333236/gate-cse-2020-question-ga-5
\begin{tikzpicture}[>=latex,shorten >=0.1pt,auto, thick,node distance=2.0cm]
\node (a) [draw , circle, minimum size= 0.5cm, line width=0.2mm] at (0.0,0.3) {1};
\node (b) [draw , circle, minimum size= 0.6cm, line width=0.2mm] at (2.0,1.7) {a};
\node (c) [draw , circle, minimum size= 0.6cm, line width=0.2mm] at (5.0,1.7) {c};
\node (d) [draw , circle, minimum size= 0.5cm, line width=0.2mm] at (3.5,0.3) {b};
\node (e) [draw , circle, minimum size= 0.5cm, line width=0.2mm] at (6.7,0.3) {2};
\node (f) [draw , circle, minimum size= 0.6cm, line width=0.2mm] at (3.5,-2.2) {f};
\node (g) [draw , circle, minimum size= 0.6cm, line width=0.2mm] at (6.7,-1.3) {e};
\node (h) [draw , circle, minimum size= 0.5cm, line width=0.2mm] at (5.2,-0.5) {d};
\path[->,line width=0.5mm]
(a) edge              node [sloped,above] {200} (b)
(b) edge              node {100} (c)
(c) edge              node [above,sloped] {100} (e)
(b) edge              node [below,sloped] {200} (e)
(a) edge              node {300} (d)
(d) edge              node {200} (e)
(d) edge              node [below ,sloped] {0} (h)
(h) edge              node [below,sloped] {100} (g)
(g) edge              node [right]{200} (e)
(a) edge              node [below, sloped] {100} (f)
(f) edge              node [below,sloped] {100} (g)
(f) edge              node [left] {0} (d);
\end{tikzpicture}



%https://gateoverflow.in/204122/gate2018-47
\begin{tikzpicture}
\tikzstyle{every node}=[scale=2,auto]

\node[shape=circle,draw=black] (A) at (-2,0) {};
\node[shape=circle,draw=black] (B) at (0,2) {};
\node[shape=circle,draw=black] (C) at (2.5,0) {};
\node[shape=circle,draw=black] (D) at (-1,-2) {};
\node[shape=circle,draw=black] (E) at (1,-2) {};
   
\path [-] (A) edge node[above] {$4$} (B);
\path [-] (B) edge node[right] {$x$} (C);
\path [-] (C) edge node[right] {$5$} (E);
\path [-] (E) edge node[right] {$3$} (B);
\path [-] (E) edge node[below left,pos = 0.2] {$4$} (D);
\path [-] (D) edge node[left] {$4$} (A);
\path [-] (D) edge node[left] {$1$} (B);
   
\end{tikzpicture}










%https://gateoverflow.in/333203/gate-cse-2020-question-28?show=333313#a333313

    \begin{karnaugh-map}[4][2][1][][]
        \minterms{1,4,5,6,7}
        \maxterms{0,2,3}
        \implicant{4}{6}
        \implicant{1}{5}
        \draw[color=black, ultra thin] (0, 2) --
    node [pos=0.7, above right, anchor=south west] {$bc$} 
    node [pos=0.7, below left, anchor=north east] {$a$} 
    ++(135:1);
    \end{karnaugh-map}
    
    %https://gateoverflow.in/1814/gate2006-38
\begin{center}

 \begin{karnaugh-map}[4][4][1][$yz$][$wx$]
       \minterms{5,7,12,13,15,11}
        \autoterms[0]
       %\indeterminats{2,5}
       \implicant{5}{15}
       \implicant{12}{13}
       \implicant {15}{11}
    \end{karnaugh-map}
    
    \begin{karnaugh-map}[4][4][1][$yz$][$wx$]
       \minterms{5,7,12,13,15,11}
        \autoterms[0]
       %\indeterminats{2,5}
       \implicant{5}{7}
       \implicant{5}{13}
       \implicant{12}{13}
       \implicant {15}{11}
       \implicant{7}{15}
    \end{karnaugh-map}
    
      \begin{karnaugh-map}[4][4][1][$yz$][$wx$]
       \minterms{5,7,12,13,15,11}
        \autoterms[0]
       %\indeterminats{2,5}
       \implicant{12}{12}
       \implicant{5}{15}
       \implicant{11}{11}
    \end{karnaugh-map}
    
      \begin{karnaugh-map}[4][4][1][$yz$][$wx$]
       \minterms{5,7,12,13,15,11}
        \autoterms[0]
       %\indeterminats{2,5}
       \implicant{13}{15}
       \implicant{5}{7}
       \implicant{5}{13}
       \implicant{12}{13}
       \implicant {15}{11}
       \implicant{7}{15}
    \end{karnaugh-map}
    
    \end{center}
%https://gateoverflow.in/333219/gate-cse-2020-question-12?show=333265#a333265
\begin{tikzpicture}[->,>=stealth',shorten >=1pt,auto,node distance=2.8cm,semithick,scale=0.7,transform shape]
  \tikzstyle{elli}=[draw,ellipse,fill=green]

\node [elli] (A) {waiting};
\node [elli] (B) [above right of = A] {running};
\node [elli] (C) [above left of = A] {ready};
\node [elli] (D) [above left of = C] {new};
\node [elli] (E) [above right of = B] {terminated};
\path[->]
(D) edge       [bend left]       node {admitted} (C)
(B) edge       [bend left]       node {exit} (E)
(B) edge       [bend right,looseness=0.9]       node [above]{interrupt} (C)
(C) edge       [bend right,looseness=0.9]       node [below]{scheduler dispatch} (B)
(A) edge       [bend left]       node {I/O or event completion} (C)
(B) edge       [bend left]       node {I/O or event wait} (A);
\end{tikzpicture}
%https://gateoverflow.in/359840/gate-civil-2020-set-2-ga-question-4
%https://civil.gateoverflow.in/1517/gate-civil-2020-set-2-ga-question-4
\begin{tikzpicture}
\node[scale=0.9, rotate=-90] at (-2.0,-0.5) {\textbf{ROAD}};
\node[] at (0.8,2.0) {\textbf{PLAYGROUND}};
\node[] at (-0.9,1.1) {\textbf{HoD}};
\node[] at (0.2,1.1) {\textbf{Q}};
\node[] at (1.0,1.1) {\textbf{R}};
\node[] at (1.6,1.1) {\textbf{S}};
\node[] at (-1.1,-0.9) {\textbf{P}};
\node[] at (1.9,-0.9) {\textbf{LIFT}};
\node[] at (1.0,-1.7) {\textbf{LAB}};

\draw[-](-2.5,2.5) -- (-2.5,-2.1);
\draw[-](-2.5,2.5) -- (2.5,2.5);
\draw[-](2.5,2.5) -- (2.5,-2.1);
\draw[-](2.5,-2.1) -- (-2.5,-2.1);

\draw[-] (-1.5,-2.1) -- (-1.5,2.5);
\draw[-] (-2.5,1.6) -- (2.5,1.6);
\draw[-] (-1.5,0.6) -- (2.5,0.6);
\draw[-,dotted] (-1.5,-0.3) -- (2.5,-0.3);
\draw[-] (-1.5,-1.2) -- (2.5,-1.2);

\draw[-] (-0.3,1.6) -- (-0.3,0.6);
\draw[-] (0.6,1.6) -- (0.6,0.6);
\draw[-] (1.3,1.6) -- (1.3,0.6);
\draw[-] (-1.5,-0.3) -- (-0.3,-0.3);
\draw[-] (-0.3,-0.3) -- (-0.3,-1.2);
\draw[-,line width=0.9mm] (2.5,-0.3) -- (1.3,-0.3);
\draw[-,line width=0.4mm] (2.5,-0.5) -- (1.3,-0.5);
\draw[-] (1.3,-0.5) -- (1.3,-1.2);
\end{tikzpicture}

\begin{tikzpicture}\node[scale=0.9, rotate=-90] at (-2.0,-0.5) {\textbf{ROAD}};
\node[] at (0.8,2.0) {\textbf{PLAYGROUND}};
\node[] at (-1.2,1.1) {\textbf{S}};
\node[] at (-0.3,1.1) {\textbf{R}};
\node[] at (0.6,1.1) {\textbf{P}};
\node[] at (1.7,1.1) {\textbf{HoD}};
\node[] at (-1.1,-0.9) {\textbf{Q}};
\node[] at (1.7,-0.9) {\textbf{LIFT}};
\node[] at (1.0,-1.7) {\textbf{LAB}};

\draw[-](-2.5,2.5) -- (-2.5,-2.1);
\draw[-](-2.5,2.5) -- (2.5,2.5);
\draw[-](2.5,2.5) -- (2.5,-2.1);
\draw[-](2.5,-2.1) -- (-2.5,-2.1);

\draw[-] (-1.5,-2.1) -- (-1.5,2.5);
\draw[-] (-2.5,1.6) -- (2.5,1.6);
\draw[-] (-1.5,0.6) -- (2.5,0.6);
\draw[-,dotted] (-1.5,-0.3) -- (2.5,-0.3);
\draw[-] (-1.5,-1.2) -- (2.5,-1.2);

\draw[-] (-0.7,1.6) -- (-0.7,0.6);
\draw[-] (0.2,1.6) -- (0.2,0.6);
\draw[-] (1.1,1.6) -- (1.1,0.6);
\draw[-] (-1.5,-0.3) -- (-0.8,-0.3);
\draw[-] (-0.8,-0.3) -- (-0.8,-1.2);
\draw[-,line width=0.9mm] (2.5,-0.3) -- (1.1,-0.3);
\draw[-,line width=0.4mm] (2.5,-0.5) -- (1.1,-0.5);
\draw[-] (1.1,-0.5) -- (1.1,-1.2);
\end{tikzpicture}

\begin{tikzpicture}\node[scale=0.9, rotate=-90] at (-2.0,-0.5) {\textbf{ROAD}};
\node[] at (0.8,2.0) {\textbf{PLAYGROUND}};
\node[] at (-1.2,1.1) {\textbf{S}};
\node[] at (-0.5,1.1) {\textbf{R}};
\node[] at (0.6,1.1) {\textbf{HoD}};
\node[] at (1.5,1.1) {\textbf{Q}};
\node[] at (-1.1,-0.9) {\textbf{P}};
\node[] at (1.9,-0.9) {\textbf{LIFT}};
\node[] at (1.0,-1.7) {\textbf{LAB}};

\draw[-](-2.5,2.5) -- (-2.5,-2.1);
\draw[-](-2.5,2.5) -- (2.5,2.5);
\draw[-](2.5,2.5) -- (2.5,-2.1);
\draw[-](2.5,-2.1) -- (-2.5,-2.1);

\draw[-] (-1.5,-2.1) -- (-1.5,2.5);
\draw[-] (-2.5,1.6) -- (2.5,1.6);
\draw[-] (-1.5,0.6) -- (2.5,0.6);
\draw[-,dotted] (-1.5,-0.3) -- (2.5,-0.3);
\draw[-] (-1.5,-1.2) -- (2.5,-1.2);

\draw[-] (-0.8,1.6) -- (-0.8,0.6);
\draw[-] (-0.0,1.6) -- (-0.0,0.6);
\draw[-] (1.2,1.6) -- (1.2,0.6);
\draw[-] (-1.5,-0.3) -- (-0.8,-0.3);
\draw[-] (-0.8,-0.3) -- (-0.8,-1.2);
\draw[-,line width=0.9mm] (2.5,-0.3) -- (1.3,-0.3);
\draw[-,line width=0.4mm] (2.5,-0.5) -- (1.3,-0.5);
\draw[-] (1.3,-0.5) -- (1.3,-1.2);
\end{tikzpicture}


\begin{tikzpicture}
\node[scale=0.9, rotate=-90] at (-2.0,-0.5) {\textbf{ROAD}};
\node[] at (0.8,2.0) {\textbf{PLAYGROUND}};
\node[] at (-0.9,1.1) {\textbf{HoD}};
\node[] at (0.2,1.1) {\textbf{S}};
\node[] at (1.0,1.1) {\textbf{R}};
\node[] at (1.6,1.1) {\textbf{Q}};
\node[] at (-1.1,-0.9) {\textbf{P}};
\node[] at (1.9,-0.9) {\textbf{LIFT}};
\node[] at (1.0,-1.7) {\textbf{LAB}};

\draw[-](-2.5,2.5) -- (-2.5,-2.1);
\draw[-](-2.5,2.5) -- (2.5,2.5);
\draw[-](2.5,2.5) -- (2.5,-2.1);
\draw[-](2.5,-2.1) -- (-2.5,-2.1);

\draw[-] (-1.5,-2.1) -- (-1.5,2.5);
\draw[-] (-2.5,1.6) -- (2.5,1.6);
\draw[-] (-1.5,0.6) -- (2.5,0.6);
\draw[-,dotted] (-1.5,-0.3) -- (2.5,-0.3);
\draw[-] (-1.5,-1.2) -- (2.5,-1.2);

\draw[-] (-0.3,1.6) -- (-0.3,0.6);
\draw[-] (0.5,1.6) -- (0.5,0.6);
\draw[-] (1.3,1.6) -- (1.3,0.6);
\draw[-] (-1.5,-0.3) -- (-0.3,-0.3);
\draw[-] (-0.3,-0.3) -- (-0.3,-1.2);
\draw[-,line width=0.9mm] (2.5,-0.3) -- (1.3,-0.3);
\draw[-,line width=0.4mm] (2.5,-0.5) -- (1.3,-0.5);
\draw[-] (1.3,-0.5) -- (1.3,-1.2);
\end{tikzpicture}
\begin{verbatim}
\begin{tikztimingtable}
  Clock 128\,MHz 0\degr    & H   12{2C} G \\ % ends with edge
  Clock 128\,MHz 90\degr   & [C] 12{2C} C \\ % starts with edge
  Clock 128\,MHz 180\degr  & C   12{2C} G \\ % ends with edge
  Clock 128\,MHz 270\degr  &     12{2C} C \\
\end{tikztimingtable}
\end{verbatim}
\begin{tikztimingtable}
  Clock 128\,MHz 0\degr    & H   12{2C} G \\ % ends with edge
  Clock 128\,MHz 90\degr   & [C] 12{2C} C \\ % starts with edge
  Clock 128\,MHz 180\degr  & C   12{2C} G \\ % ends with edge
  Clock 128\,MHz 270\degr  &     12{2C} C \\
\end{tikztimingtable}


\begin{verbatim}\begin{tikztimingtable}
  Coarse Pulse                          & 3L 16H 6L \\
  Coarse Pulse - Delayed 1              & 4L 16H 5L \\
  Coarse Pulse - Delayed 2              & 5L 16H 4L \\
  Coarse Pulse - Delayed 3              & 6L 16H 3L \\
  \\ % Gives vertical space
  Final Pulse Set                       & 3L 16H 6L \\
  Final Pulse $\overline{\mbox{Reset}}$ & 6L 16H 3L \\
  Final Pulse                           & 3L 19H 3L \\
\end{tikztimingtable}
\end{verbatim}
\begin{tikztimingtable}
  Coarse Pulse                          & 3L 16H 6L \\
  Coarse Pulse - Delayed 1              & 4L 16H 5L \\
  Coarse Pulse - Delayed 2              & 5L 16H 4L \\
  Coarse Pulse - Delayed 3              & 6L 16H 3L \\
  \\ % Gives vertical space
  Final Pulse Set                       & 3L 16H 6L \\
  Final Pulse $\overline{\mbox{Reset}}$ & 6L 16H 3L \\
  Final Pulse                           & 3L 19H 3L \\
\end{tikztimingtable}


\begin{verbatim}
  Clock 128\,MHz 0\degr    & H         12{2C}             G \\ % without notes
  Clock 128\,MHz 0\degr    & H 2C N(A1) 8{2C} N(A5) 3{2C} G \\ % with    notes
\end{verbatim}



\begin{verbatim}
  \begin{pgfonlayer}{background}
    \foreach \n in {1,...,8}
      \draw [help lines] (A\n) -- (B\n);
  \end{pgfonlayer}
\end{verbatim}



\begin{verbatim}
\def\degr{${}^\circ$}\begin{tikztimingtable}
  Clock 128\,MHz 0\degr    & H 2C N(A1) 8{2C} N(A5) 3{2C} G\\
  Clock 128\,MHz 90\degr   & [C] 2{2C} N(A2) 8{2C} N(A6) 2{2C} C\\
  Clock 128\,MHz 180\degr  & C 2{2C} N(A3) 8{2C} N(A7) 2{2C} G\\
  Clock 128\,MHz 270\degr  & 3{2C} N(A4) 8{2C} N(A8) 2C C\\
  Coarse Pulse             & 3L 16H 6L \\
  Coarse Pulse - Delayed 1 & 4L N(B2) 16H N(B6) 5L \\
  Coarse Pulse - Delayed 2 & 5L N(B3) 16H N(B7) 4L \\
  Coarse Pulse - Delayed 3 & 6L 16H 3L \\
  \\
  Final Pulse Set          & 3L 16H N(B5) 6L \\
  Final Pulse $\overline{\mbox{Reset}}$ & 6L N(B4) 16H 3L \\
  Final Pulse              & 3L N(B1) 19H N(B8) 3L \\
\extracode
  \tablerules
  \begin{pgfonlayer}{background}
    \foreach \n in {1,...,8}
      \draw [help lines] (A\n) -- (B\n);
  \end{pgfonlayer}
\end{tikztimingtable}
\end{verbatim}

\def\degr{${}^\circ$}
\begin{tikztimingtable}
  Clock 128\,MHz 0\degr    & H 2C N(A1) 8{2C} N(A5) 3{2C} G\\
  Clock 128\,MHz 90\degr   & [C] 2{2C} N(A2) 8{2C} N(A6) 2{2C} C\\
  Clock 128\,MHz 180\degr  & C 2{2C} N(A3) 8{2C} N(A7) 2{2C} G\\
  Clock 128\,MHz 270\degr  & 3{2C} N(A4) 8{2C} N(A8) 2C C\\
  Coarse Pulse             & 3L 16H 6L \\
  Coarse Pulse - Delayed 1 & 4L N(B2) 16H N(B6) 5L \\
  Coarse Pulse - Delayed 2 & 5L N(B3) 16H N(B7) 4L \\
  Coarse Pulse - Delayed 3 & 6L 16H 3L \\
  \\
  Final Pulse Set          & 3L 16H N(B5) 6L \\
  Final Pulse $\overline{\mbox{Reset}}$ & 6L N(B4) 16H 3L \\
  Final Pulse              & 3L N(B1) 19H N(B8) 3L \\
\extracode
  \tablerules
  \begin{pgfonlayer}{background}
    \foreach \n in {1,...,8}
      \draw [help lines] (A\n) -- (B\n);
  \end{pgfonlayer}
\end{tikztimingtable}

\begin{tikzpicture}[every node/.style={circle,draw},level 1/.style={sibling distance=30mm},level 2/.style={sibling distance=30mm},level 3/.style={sibling distance=20mm},level 4/.style={sibling distance=10mm},minimum size=0.50cm,iv/.style={draw,rectangle,minimum size=15pt,inner
sep=0pt,text=black},ev/.style={draw,circle,minimum
size=15pt,inner sep=0pt,text=black}]
\node[ev]{}
  child {node[ev]{}
         child {node[ev]{}
               child {node[ev]{}
               child {node[iv]{$2$}}
               child {node[iv]{$4$}}}
               child{node[ev]{}
                    child {node[iv]{$5$}}
                    child {node[iv]{$7$}}}
               }
         child {node[ev]{}
              child {node[iv]{$9$}}
              child {node[iv]{$10$}}}}
  child {node[iv]{$15$}
        child[missing]};
 
\end{tikzpicture}





%
%%https://ec.gateoverflow.in/1525/gate2020-ec-ga-8
\begin{tikzpicture}[scale = 0.8]
\draw[fill=teal!65!white, very thick](0,0) circle (4);
\draw[fill = white, very thick] (-3.20,-2.35) rectangle (3.20,2.35);
\node[above] at (0,0.2) {\huge{$\text{O}$}};
\node[left] at (-3.15,2.5) {\huge{$\text{P}$}};
\node[left] at (-3.25,-2.5) {\huge{$\text{S}$}};
\node[right] at (3.15,2.5) {\huge{$\text{Q}$}};
\node[right] at (3.25,-2.5) {\huge{$\text{R}$}};
\draw[teal!65!white,very thick] (-0.01,0.15) -- (-0.01,-0.20);
\draw[teal!65!white,very thick] (-0.15,0) -- (0.08,0);
\node[] at (-1.75,-0.75) {\huge{$a$}};
\draw[very thick](-3.20,-2.35) -- (0,0);
\end{tikzpicture}


 %%%%%%%https://gateoverflow.in/313674/gate2017-me-2-ga-10
  \begin{tikzpicture}[scale=0.6]
\begin{axis}[ylabel=Pollutant Concentration (ppm),xlabel=Day of the Month,
    xtick={0,2,4,6,8,10,12,14,16,18,20,22,24,26,28,30},
    ytick={0,1,2,3,4,5,6,7,8,9,10,11},
    minor tick num=1,
   xmin = -1,xmax=31,ymin=-.5,
    width=0.6\textwidth,
    height=\axisdefaultheight,
    legend pos=north west,
    grid=both,
    grid style={dotted},
    major grid style={dashed},
]
\addplot[black] plot [mark=square] coordinates { (0,0)(2,.3)(4,.6) (6,0.9) (8,1.5)(10,3)(12,4.5)(14,6)(16,6.7)(18,7.1)(20,7.5)(22,7.7)(24,7.9)(26,8)(28,8)(30,8)};
\addplot [red] plot [mark=o] coordinates {(0,1.5)(2,1.8)(4,1.9)(6,2)(8,2.5)(10,3)(12,4)(14,6)(16,8)(18,9)(20,9.5)(22,10)(24,10.2)(26,10.4)(28,10.4)(30,10.4)};
\addlegendentry{Winter}
\addlegendentry{Summer}
\end{axis}
\end{tikzpicture}  


%https://gateoverflow.in/313662/gate2017-me-1-ga-10
       \begin{tikzpicture}[scale=.5]
       
           \begin{axis}[
           ylabel=Population density,
           xlabel=Time (min),
           xtick={0,20,40,60,80,100,120,140,160,180,200},
           ytick={0.0,0.1,0.2,0.3,0.4,0.5,0.6,0.7,0.8,0.9,1.0},
           minor tick num=1,
           xmin = -10,xmax=210,ymin=-0.07,ymax=1.1,
           width=0.8\textwidth,
           height=\axisdefaultheight,
           legend pos=north west,
           grid=both,
           grid style={dotted},
           major grid style={dashed},
        ]
        \addplot[black] plot [mark=square] coordinates {
        (0,0.0) (10,0.02) (20,0.04) (30,0.06) (40,0.11) (50,0.21) (60,0.4) (70,0.59) (80,0.75) (90,0.82) (100,0.88) (110,0.92) (120,0.95) (130,0.97) (140,0.99) (150,1) (160,1) (170,1) (180,1) (190,1) (200,1)};
        \addplot [red] plot [mark=o] coordinates {
        (0,0.0) (10,0.01) (20,0.02) (30,0.03) (40,0.04) (50,0.05) (60,0.08) (70,0.13) (80,0.2) (90,0.3) (100,0.5) (110,0.67) (120,0.77) (130,0.83) (140,0.88) (150,0.92) (160,0.965) (170,0.98) (180,0.995) (190,1.0) (200,1.0)};
        \addlegendentry{37$^{\circ}$ C}
        \addlegendentry{25$^{\circ}$ C}
        \end{axis} 
           
                   
       \end{tikzpicture}


%%https://me.gateoverflow.in/1813/gate-mechanical-2020-set-2-ga-question-10
\begin{tikzpicture}
\begin{scope}
\filldraw[draw=black,fill=white] (0,0) -- (0:3cm) arc[start angle=0,end angle=54,radius=3cm] -- cycle;
\filldraw[draw=black,fill=white] (0,0) -- (54:3cm) arc[start angle=54,end angle=126,radius=3cm] -- cycle;
\filldraw[draw=black,fill=white] (0,0) -- (126:3cm) arc[start angle=126,end angle=234,radius=3cm] -- cycle;
\filldraw[draw=black,fill=white] (0,0) -- (234:3cm) arc[start angle=234,end angle=288,radius=3cm] -- cycle;
\filldraw[draw=black,fill=white] (0,0) -- (288:3cm) arc[start angle=288,end angle=360,radius=3cm] -- cycle;

\node[] at (10:1.8cm) {\textbf{15\%}};
\node[] at (20:1.88cm) {\textbf{Management}};

\node[] at (90:1.75cm) {\textbf{20\%}};
\node[] at (90:2.15cm) {\textbf{Arts}};

\node[] at (170:1.65cm) {\textbf{30\%}};
\node[] at (160:1.75cm) {\textbf{Engineering}};

\node[] at (260:2.35cm) {\textbf{15\%}};
\node[] at (260:2cm) {\textbf{Commerce}};

\node[] at (320:2cm) {\textbf{20\%}};
\node[] at (330:1.85cm) {\textbf{Science}};

\node[above] at (0,3.4){\textbf{Percentage of students enrolled in different}};
\node[above] at (0,3) {\textbf{stream in a University}};
\end{scope}

\begin{scope}[xshift = 8.5cm]
\filldraw[draw=black,fill=white] (0,0) -- (0:3cm) arc[start angle=0,end angle=54,radius=3cm] -- cycle;
\filldraw[draw=black,fill=white] (0,0) -- (54:3cm) arc[start angle=54,end angle=162,radius=3cm] -- cycle;
\filldraw[draw=black,fill=white] (0,0) -- (162:3cm) arc[start angle=162,end angle=198,radius=3cm] -- cycle;
\filldraw[draw=black,fill=white] (0,0) -- (198:3cm) arc[start angle=198,end angle=270,radius=3cm] -- cycle;
\filldraw[draw=black,fill=white] (0,0) -- (270:3cm) arc[start angle=270,end angle=360,radius=3cm] -- cycle;

\node[] at (10:1.8cm) {\textbf{15\%}};
\node[] at (20:1.88cm) {\textbf{Management}};

\node[] at (100:1.35cm) {\textbf{30\%}};
\node[] at (100:1.75cm) {\textbf{Arts}};

\node[] at (190:1.65cm) {\textbf{10\%}};
\node[] at (180:1.75cm) {\textbf{Engineering}};

\node[] at (230:1.75cm) {\textbf{20\%}};
\node[] at (220:1.5cm) {\textbf{Commerce}};

\node[] at (320:2cm) {\textbf{25\%}};
\node[] at (330:1.85cm) {\textbf{Science}};

\node[above] at (0,3.4){\textbf{Percentage of girls enrolled in different}};
\node[above] at (0,3) {\textbf{stream}};
\end{scope}
\end{tikzpicture}

%%https://ch.gateoverflow.in/1156/gate-chemical-2020-ga-question-10
%%https://gateoverflow.in/359808/gate-chemical-2020-ga-question-10
\begin{tikzpicture}[scale = 0.8,font = \bfseries]
\begin{axis}[
ybar=0pt, %space between adjacent bars is 0
enlarge x limits=0.20,
enlarge y limits=false,
legend style={at={(0.20,0.95),ybar=area},
anchor=north,legend columns=1.5},
ylabel= Profit Percentage,
xlabel= Year,
ymajorgrids = true,
xmajorgrids = true,
symbolic x coords={2013,2014,2015,2016,2017,2018},
xtick=data,
ytick={0,10,20,...,80},
bar width=0.45cm,
%style={font=\boldmath},
nodes near coords align={vertical},
%every x tick label/.append style={font=\bfseries},
every y tick label/.append style={font=\boldmath},
%tick label style={font=\boldmath}, 
label style={color=black,font=\bfseries},
width=12cm,
height=10cm,
ymax=70,
ymin=0
]
\addplot
[fill=red!40!orange,draw=black,mark=none]
coordinates {(2013,10) (2014,20) (2015,40) (2016,40) (2017,50) (2018,40)};
\addplot
[fill=blue!20!gray,draw=black]
coordinates {(2013,20) (2014,30) (2015,30) (2016,50) (2017,60) (2018,60)};
\legend{\;Company P\;,Company Q}

\end{axis}
\end{tikzpicture}




%%https://gateoverflow.in/359518/gate-mechanical-2020-set-1-ga-question-10
%%https://me.gateoverflow.in/1711/gate-mechanical-2020-set-1-ga-question-10

\begin{tikzpicture}[scale = 0.8,font = \bfseries]
\begin{axis}[
ybar=0pt, %space between adjacent bars is 0
enlarge x limits=0.18,
enlarge y limits=false,
title={Performance of Schools P, Q, R and S},
legend style={at={(0.18,0.97),ybar=area},
anchor=north,legend columns=-1},
ylabel= {Number of students},
symbolic x coords={School P, School Q, School R, School S},
xtick=data,
ytick={0,100,200,...,800,900,1000},
nodes near coords,
bar width=0.75cm,
ymajorgrids = true,
every node near coord/.append style={font=\boldmath},
nodes near coords align={vertical},
every x tick label/.append style={font=\bfseries},
every y tick label/.append style={font=\boldmath},
%tick label style={font=\boldmath}, 
label style={color=black,font=\bfseries},
width=14cm,
height=10cm,
ymax=800,
ymin=0
]
\addplot
[fill=magenta!40!white,postaction={
        pattern=north west lines
    },draw=black]
coordinates {(School P,500) (School Q,600) (School R,700) (School S,400)};
\addplot
[fill=blue!40!white,postaction={
        pattern=north east lines
    },draw=black]
coordinates {(School P,280) (School Q,330) (School R,455) (School S,240)};
\legend{Appeared\;\;,Passed}
\end{axis}
\end{tikzpicture}


%%https://gateoverflow.in/359759/gate-ece-2020-ga-question-10
%%https://ec.gateoverflow.in/1523/gate-ece-2020-ga-question-10

\begin{tikzpicture}[scale = 0.8,font = \bfseries]
\begin{axis}[
ybar=0pt, %space between adjacent bars is 0
enlarge x limits=0.20,
enlarge y limits=false,
legend style={at={(0.18,0.97),ybar=area},
anchor=north,legend columns=1},
ylabel= Number of students (in thousands),
xlabel= Year,
ymajorgrids = true,
xmajorgrids = true,
symbolic x coords={2014,2015,2016,2017,2018},
xtick=data,
ytick={0,1,2,...,8,9,10},
bar width=0.5cm,
every node near coord/.append style={font=\boldmath},
nodes near coords align={vertical},
every x tick label/.append style={font=\bfseries},
every y tick label/.append style={font=\boldmath},
%tick label style={font=\boldmath}, 
label style={color=black,font=\bfseries},
width=12cm,
height=10cm,
ymax=9,
ymin=0
]
\addplot
[fill=red!40!orange,postaction={
        pattern=north east lines
    },draw=black]
coordinates {(2014,3) (2015,5) (2016,5) (2017,6) (2018,4)};
\addplot
[fill=blue!20!gray,postaction={
        pattern= vertical lines
    }, draw=black]
coordinates {(2014,4) (2015,7) (2016,8) (2017,7) (2018,5)};
\legend{\;School P\;,School Q}

\end{axis}
\end{tikzpicture}



%%https://gateoverflow.in/359710/gate-electrical-2020-ga-question-10
%%https://ee.gateoverflow.in/1483/gate-electrical-2020-ga-question-10

\begin{tikzpicture}[scale = 0.8,font = \bfseries]
\begin{axis}[
ybar=0pt, %space between adjacent bars is 0
enlarge x limits=0.20,
enlarge y limits=false,
title={Revenue and Expenditure (in million rupees) of four  \\ companies P, Q, R and S in 2015},
title style={yshift=1mm,align=center},
legend style={at={(0.5,0.97),ybar=area},
anchor=north,legend columns=-1},
ylabel= Revenue/Expenditure (in million rupees),
symbolic x coords={Company P, Company Q, Company R, Company S},
xtick=data,
ytick={0,5,10,...,50,55,60},
bar width=0.60cm,
ymajorgrids = true,
every node near coord/.append style={font=\boldmath},
nodes near coords align={vertical},
every x tick label/.append style={font=\bfseries},
every y tick label/.append style={font=\boldmath},
%tick label style={font=\boldmath}, 
label style={color=black,font=\bfseries},
width=12cm,
height=10cm,
ymax=55,
ymin=0
]
\addplot
[fill=magenta!40!white,postaction={
        pattern=north west lines
    },draw=black]
coordinates {(Company P,35) (Company Q,45) (Company R,30) (Company S,40)};
\addplot
[fill=blue!40!white,postaction={
        pattern=north east lines
    },draw=black]
coordinates {(Company P,25) (Company Q,35) (Company R,40) (Company S,50)};
\legend{Revenue\;\;,Expenditure}
\end{axis}
\end{tikzpicture}

%https://gateoverflow.in/313419/gate2017-ce-2-ga-10
\begin{tikzpicture}
            \tikzstyle{nodep}=[circle,draw,scale=.5,fill=red!50]
            
            \draw[-] (0,6)--(12.5,6)--(12.5,0)--(0,0)--(0,6); 
    
             \draw (0,-.2) node {0};
             \draw (1,-.2) node {5};
             \draw (2,-.2) node {10};
             \draw (3,-.2) node {15};
             \draw (4,-.2) node {20};
             \draw (5,-.2) node {25};
             \draw (6,-.2) node {30};
             \draw (7,-.2) node {35};
             \draw (8,-.2) node {40};
             \draw (9,-.2) node {45};
             \draw (10,-.2) node {50};
             \draw (11,-.2) node {55};
             \draw (12,-.2) node {60};
             
            \foreach \x in {1,2,...,12}
            {
                \draw[dotted] (\x,0)--(\x,6);
            }
            \foreach \x in {.5,1.5,...,11.5}
            {
                \draw[-] (\x,0)--(\x,.1);
            }
            \foreach \x in {.5,1.5,...,11.5}
            {
                \draw[-] (\x,6)--(\x,5.9);
            
            }
            \draw (-.2,.5) node {0};
            \draw (-.2,1.5) node {1};
            \draw (-.2,2.5) node {2};
            \draw (-.2,3.5) node {3};
            \draw (-.2,4.5) node {4};
            \draw (-.2,5.5) node {5};
            
             \foreach \y in {.5,...,5.5}
            {
                \draw[-] (0,\y)--(12.5,\y);
            }
             \foreach \y in {1,...,5}
            {
                \draw[-] (0,\y)--(.1,\y);
            }
             \foreach \y in {1,...,5}
            {
                \draw[-] (12.5,\y)--(12.4,\y);
            }
            
            %ground_floor
            \node[nodep] at (.2,.5) {};
            \node[nodep] at (.84,.5) {};
            \node[nodep] at (1,.5) {};
            \node[nodep] at (1.84,.5) {};
            \node[nodep] at (2.2,.5) {};
            \node[nodep] at (3,.5) {};
            \node[nodep] at (4,.5) {};
            \node[nodep] at (4.44,.5) {};
            \node[nodep] at (4.6,.5) {};
            \node[nodep] at (5.4,.5) {};
            \node[nodep] at (6,.5) {};
            \node[nodep] at (6.16,.5) {};
            \node[nodep] at (6.84,.5) {};
            \node[nodep] at (7.4,.5) {};
            \node[nodep] at (8.4,.5) {};
            \node[nodep] at (8.8,.5) {};
            \node[nodep] at (9.63,.5) {};
            \node[nodep] at (9.8,.5) {};
            \node[nodep] at (10.4,.5) {};
            \node[nodep] at (11,.5) {};
            \node[nodep] at (11.63,.5) {};
            
            %first_floor
            \node[nodep] at (11.2,1.5) {};
            \node[nodep] at (11.37,1.5) {};
            
            %second_floor
            \node[nodep] at (4.2,2.5) {};
            \node[nodep] at (5.6,2.5) {};
            \node[nodep] at (7.63,2.5) {};
            \node[nodep] at (7.8,2.5) {};
            
            %third_floor
            \node[nodep] at (2.46,3.5) {};
            \node[nodep] at (2.63,3.5) {};
            \node[nodep] at (2.8,3.5) {};
            \node[nodep] at (8.6,3.5) {};
            \node[nodep] at (10,3.5) {};
            \node[nodep] at (10.16,3.5) {};
            
            %fourth_floor
            \node[nodep] at (1.26,4.5) {};
            \node[nodep] at (1.43,4.5) {};
            \node[nodep] at (1.6,4.5) {};
            \node[nodep] at (3.26,4.5) {};
            \node[nodep] at (3.43,4.5) {};
            \node[nodep] at (3.6,4.5) {};
            \node[nodep] at (3.78,4.5) {};
            \node[nodep] at (4.83,4.5) {};
            \node[nodep] at (5,4.5) {};
            \node[nodep] at (5.17,4.5) {};
            \node[nodep] at (7,4.5) {};
            \node[nodep] at (7.17,4.5) {};
            \node[nodep] at (9,4.5) {};
            \node[nodep] at (9.17,4.5) {};
            \node[nodep] at (9.34,4.5) {};
            \node[nodep] at (10.6,4.5) {};
            \node[nodep] at (10.77,4.5) {};
            \node[nodep] at (11.83,4.5) {};
            \node[nodep] at (12,4.5) {};
            
            %fifth_floor
            \node[nodep] at (.43,5.5) {};
            \node[nodep] at (.6,5.5) {};
            \node[nodep] at (2,5.5) {};
            \node[nodep] at (5.77,5.5) {};
            \node[nodep] at (6.43,5.5) {};
            \node[nodep] at (6.6,5.5) {};
            \node[nodep] at (8,5.5) {};
            \node[nodep] at (8.17,5.5) {};
            
            \draw (-1,3) node[rotate=90] {Floor Number};
            \draw (5.5,-1) node {Time (min)};
            
       \end{tikzpicture}







%% https://me.gateoverflow.in/1818/gate2020-me-2-ga-5

\begin{tikzpicture}[transform shape,scale = 0.8]
\draw[thick,rounded corners] (0,0) rectangle (2,1);
\draw[thick,rounded corners] (3,0) rectangle (5,1);
\draw[thick,rounded corners] (6,0) rectangle (8,1);
\draw[thick,rounded corners] (9,0) rectangle (11,1);
\draw[thick,rounded corners] (12,0) rectangle (14,1);
\draw[thick,rounded corners] (15,0) rectangle (17,1);
\node[left] at (1.25,0.5){$\text{P}$};
\node[left] at (4.25,0.5){$\text{Q}$};
\node[left] at (7.25,0.5){$\text{R}$};
\node[left] at (10.25,0.5){$\text{S}$};
\node[left] at (13.25,0.5){$\text{T}$};
\node[left] at (16.90,0.5){$\text{Customers}$};
\draw[-{Triangle[width=14pt,length=6pt,]}, line width=6pt,gray!65!white](2.20,0.5) -- (2.80, 0.5);
\draw[-{Triangle[width=14pt,length=6pt,]}, line width=6pt,gray!65!white](5.20,0.5) -- (5.80, 0.5);
\draw[-{Triangle[width=14pt,length=6pt,]}, line width=6pt,gray!65!white](8.20,0.5) -- (8.80, 0.5);
\draw[-{Triangle[width=14pt,length=6pt,]}, line width=6pt,gray!65!white](11.20,0.5) -- (11.80, 0.5);
\draw[-{Triangle[width=14pt,length=6pt,]}, line width=6pt,gray!65!white](14.20,0.5) -- (14.80, 0.5);
\end{tikzpicture}


%%https://gateoverflow.in/333232/gate-cse-2020-question-ga-9

\begin{tikzpicture}[scale = 0.8]
\draw[help lines, color=gray!40, dashed] (-3,-3) grid (10,7);
\draw[->,very thick] (-2,0)--(8,0) node[right]{$\text{X-axis}$};
\draw[->,very thick] (0,-2)--(0,6) node[above]{$\text{Y-axis}$};
\draw [] (1,-1) -- (5,5);
\draw [] (6,-1) -- (2,5);
\path[right](2.10,0.60)  edge  [bend left]  node {$\alpha$} (2.15,0);
\path[left](4.90,0.65)  edge  [bend right]  node {$\beta$} (4.80,0);
\draw[] (3.25,2.40) -- (3.5,2.20) -- (3.75,2.40);
\node[right] at (4.5,4) {$L_{1}$};
\node[left] at (2.5,4) {$L_{2}$};
\end{tikzpicture}



%https://ee.gateoverflow.in/1484/gate2020-ee-ga-9

\begin{tikzpicture}
\node at (0.0,1.7) {C};
\node at (0.0,-1.4) {O};
\node at (-2.5,-0.9) {A};
\node at (2.5,-0.9) {B};
\node [draw , semicircle, minimum size= 2.2cm, line width=0.5mm] at (0.0,0.0) {};
\draw[-] (0.0,-0.9) -- (0.0,1.3);
\draw[-, color =blue, line width=0.5mm] (-2.2,-0.9) -- (0.0,1.3);
\draw[-, color =blue, line width=0.5mm] (2.2,-0.9) -- (0.0,1.3);
\draw[-] (0.2,-0.9) -- (0.2,-0.7);
\draw[-] (0.0,-0.7) -- (0.2,-0.7);
\end{tikzpicture}



%https://me.gateoverflow.in/1814/gate2020-me-2-ga-9

\begin{tikzpicture}[>=stealth',shorten >=1pt,auto,thick,node distance=2.8cm]
\node at (-0.1,1.9) {\text{\small{(X,Y)}}};
\node at (2.4,-0.3) {\text{\small{Z}}};
\node at (0.1,-0.2) {\text{\small{O}}};

\draw[->,line width=0.4mm] (0.0,0.0) -- (2.5,0.0);
\draw[->,line width=0.4mm] (0.4,-0.4) -- (0.4,2.5);
\draw[-, color=blue,line width=0.4mm] (0.4,0.0) -- (1.2,2.1);
\end{tikzpicture}



%%https://gateoverflow.in/333233/gate-cse-2020-question-ga-8

\begin{tikzpicture}[scale = 0.8,transform shape]
\draw[thick](0,0) circle (5);
\draw[thick](0,0) circle (3.75);
\draw[fill=teal!70!white,draw=teal,thick](0,-4.36) circle (0.60);
\draw[fill=teal!70!white,draw=teal,thick](1.20,-4.20) circle (0.60);
\draw[fill=teal!70!white,draw=teal,thick](2.32,-3.70) circle (0.60);
\draw[fill=teal!70!white,draw=teal,thick](3.25,-2.93) circle (0.60);
\draw[fill=teal!70!white,draw=teal,thick](3.92,-1.91) circle (0.60);
\draw[fill=teal!70!white,draw=teal,thick](4.30,-0.77) circle (0.60);
\draw[fill=teal!70!white,draw=teal,thick](4.35,0.42) circle (0.60);
\draw[fill=teal!70!white,draw=teal,thick](4.08,1.60) circle (0.60);
\draw[fill=teal!70!white,draw=teal,thick](3.48,2.66) circle (0.60);

\draw[draw=teal, ultra thick,line width=3pt] (2.10,3.82) -- (2.40,3.63);
\draw[draw=teal, ultra thick,line width=3pt] (1.40,4.12) -- (1.70,4);
\draw[draw=teal, ultra thick, line width=3pt] (0.65,4.37) -- (0.95,4.28);

\draw[->,>=latex, ultra thick,teal!80!white] (0,0) -- (-4.20,2.75);
\draw[->,>=latex,ultra thick,teal!80!white] (0,0) -- (-2.85,-2.45);

\node[above] at (-1.5,1.20) {\Huge{$b$}};
\node[left] at (-1.5,-0.90) {\Huge{$a$}};
\end{tikzpicture}







%%https://gateoverflow.in/333231/gate-cse-2020-question-ga-10
\definecolor{mintbg}{rgb}{.63,.79,.95}

    \begin{tikzpicture}[scale= 0.5,transform shape]
        \begin{axis}[
        ybar stacked,
        ymin=0,
        ymax=900,
        ytick={0,100,...,900},
            symbolic x coords={2014,2015,2016,2017,2018},
           xtick=data,
             ylabel near ticks,
    xlabel near ticks,
              xlabel={Year},
              ymajorgrids = true,
   ylabel={Revenue (In million rupees)} ]
            \addplot[ybar,fill=mintbg,bar width=15pt] coordinates {
                (2014, 500)
                (2015, 700)
                (2016, 800)
                (2017, 600)
                (2018, 400)
               
            };
        \end{axis}
    \end{tikzpicture}








%%https://me.gateoverflow.in/1712/gate-mechanical-2020-set-1-ga-question-9
%answer image
\begin{tikzpicture}[>=latex,shorten >=1pt,auto, semithick,node distance=2.8cm,scale = 1]
\node[] at (-0.2,3.4) {$y$};
\node[left] at (0,2.5) {$1$};
\node[left] at (0,-0.2) {$0$};
\node[below] at (2.0,0) {$1$};
\node[] at (3.1,-0.2) {$x$};
\node[] at (1.65,0.6) {$x^{m}$};
\node[] at (1.1,1.6) {$x^{1/m}$};
\draw[->,line width=0.5mm] (0.0,-0.3) -- (0.0,3.5);
\draw[->,line width=0.5mm] (-0.3,0.0) -- (3.0,0.0);
\draw[-,line width=0.45mm,dashed](2.0,0.0) -- (2.0,2.5) -- (0.0,2.5);
\path[-,line width=0.45mm]
[teal] (0.0,0.0) edge [bend left = 35]  node  {} (2.0,2.52)
[teal] (0.0,0.0) edge [bend right = 35]  node  {} (1.98,2.53);
\end{tikzpicture}

%%https://ch.gateoverflow.in/1159/gate-chemical-2020-ga-question-7
%answer image 1
\begin{tikzpicture}[scale = 0.6]
    % equidistant points and arc
    \foreach \x [count=\p] in {0,...,7} {
        \node[shape=circle,fill=black, scale=0.4] (\p) at (-\x*45:2) {};};
    \draw (1) arc (0:360:2);

\node[above] at (0,2.15) {$\textbf{P}$};    
\node[below] at (0,-2.15) {$\textbf{R}$}; 
\node[right] at (1.55,1.5) {$\textbf{W}$};    
\node[left] at (-1.55,1.5) {$\textbf{T}$}; 
\node[right] at (2.12,0.08) {$\textbf{U}$};   
\node[left] at (-2.12,0.08) {$\textbf{V}$};   
\node[right] at (1.68,-1.5) {$\textbf{Q}$};  
\node[left] at (-1.68,-1.5) {$\textbf{S}$}; 

\end{tikzpicture}
%answer image 2
\begin{tikzpicture}[scale = 0.6]
    % equidistant points and arc
    \foreach \x [count=\p] in {0,...,7} {
        \node[shape=circle,fill=black, scale=0.4] (\p) at (-\x*45:2) {};};
    \draw (1) arc (0:360:2);

\node[above] at (0,2.15) {$\textbf{Q}$};    
\node[below] at (0,-2.15) {$\textbf{R}$}; 
\node[right] at (1.55,1.5) {$\textbf{W}$};    
\node[left] at (-1.55,1.5) {$\textbf{T}$}; 
\node[right] at (2.12,0.08) {$\textbf{U}$};   
\node[left] at (-2.12,0.08) {$\textbf{V}$};   
\node[right] at (1.68,-1.5) {$\textbf{P}$};  
\node[left] at (-1.68,-1.5) {$\textbf{S}$}; 
\end{tikzpicture}


%%https://me.gateoverflow.in/1816/gate2020-me-2-ga-7
%question-image
\begin{tikzpicture}[scale = 0.6]
\draw[] (0,0) -- (8,0) -- (4.20,-4.5) -- (0,0);
\draw[] (0,-2.75) -- (8,-2.75) -- (4.20,1.5) -- (0,-2.75);
\node[] at (4.1,0.5){\large{$5$}};
\node[] at (4.1,-3.25){\large{$9$}};
\node[] at (1.25,-0.5){\large{$t$}};
\node[] at (1.25,-2.25){\large{$?$}};
\node[] at (6.70,-0.5){\large{$h$}};
\node[] at (6.70,-2.25){$x$};
\node[] at (4.1,-1.45){\large{$n$}};
\end{tikzpicture}
%%https://me.gateoverflow.in/1816/gate2020-me-2-ga-7
%answer-image1
\begin{tikzpicture}[scale = 0.6]
\draw[] (0,0) -- (8,0) -- (4.20,-4.5) -- (0,0);
\draw[] (0,-2.75) -- (8,-2.75) -- (4.20,1.5) -- (0,-2.75);
\node[] at (4.1,0.5){\large{$5$}};
\node[] at (4.1,-3.25){\large{$9$}};
\node[] at (1.25,-0.5){\large{$20$}};
\node[] at (1.25,-2.25){\large{$?$}};
\node[] at (6.70,-0.5){\large{$8$}};
\node[] at (6.70,-2.25){$24$};
\node[] at (4.1,-1.45){\large{$14$}};
\end{tikzpicture}
%%https://me.gateoverflow.in/1816/gate2020-me-2-ga-7
%answer-image2
\begin{tikzpicture}[node distance=25mm,auto,scale = 0.8]
\node[state,minimum size= 0.5mm] (q0) {$14$};
\node[state,minimum size= 0.5mm] (q1)[above of  = q0] {$5$};
\node[state,minimum size= 0.5mm] (q2)[below of  = q0] {$9$};
\node[state,minimum size= 0.5mm] at (-3,1.25) (q3) {$20$};
\node[state,minimum size= 0.5mm] at (-3,-1.25)(q4) {$4$};
\node[state,minimum size= 0.5mm] at (3,1.25)(q5) {$8$};
\node[state,minimum size= 0.5mm] at (3,-1.25)(q6) {$24$};
\path[] (q0) edge[above left,pos = 0.90,sloped] node {$14-9$} (q1);
\path[] (q0) edge[below right,pos = 0.90,sloped,rotate=180] node {$14-5$} (q2);
\path[] (q0) edge[above right,pos = 0.85,sloped] node {$14+6$} (q3);
\path[] (q0) edge[above right,pos = 0.85,sloped] node {$14-10$} (q4);
\path[] (q0) edge[above right,pos = 0.25,sloped] node {$14-6$} (q5);
\path[] (q0) edge[above right,pos = 0.25,sloped] node {$14+10$} (q6);
\end{tikzpicture}



%%https://civil.gateoverflow.in/1668/gate2020-ce-1-ga-10
%question image
\begin{tikzpicture}
\tikzset{
     lines/.style={draw=none},
}
\pie[text = legend,style={lines},color={blue!40!cyan,red!60!white, yellow!60!green, violet!50!white,cyan!70!white,orange!70!white,{rgb:red,1;green,2;blue,3}}]
{5/Health (5\%),
 10/Transport (10\%),
 15/Household Items (15\%),
 15/Education (15\%),
 10/Leisure (10\%),
 20/House rent (20\%),
 25/Others (25\%)
 }
\filldraw[draw=none,fill=yellow!60!green] (0,0) -- (-27:3cm) arc[start angle=-27,end angle=27,radius=3cm] -- cycle;
\filldraw[draw=none,fill=red!60!white] (0,0) -- (27:3cm) arc[start angle=27,end angle=63,radius=3cm] -- cycle;
\filldraw[draw=none,fill=blue!40!cyan] (0,0) -- (63:3cm) arc[start angle=63,end angle=81,radius=3cm] -- cycle;
\filldraw[draw=none,fill={rgb:red,1;green,2;blue,3}] (0,0) -- (81:3cm) arc[start angle=81,end angle=171,radius=3cm] -- cycle;
\filldraw[draw=none,fill=orange!70!white] (0,0) -- (171:3cm) arc[start angle=171,end angle=243,radius=3cm] -- cycle;
\filldraw[draw=none,fill=cyan!70!white] (0,0) -- (243.2:3cm) arc[start angle=243,end angle=279,radius=3cm] -- cycle;
\filldraw[draw=none,fill=violet!50!white] (0,0) -- (279:3cm) arc[start angle=279,end angle=333,radius=3cm] -- cycle;

\node[right] at (6:1.85cm) {\text{15\%}};

\node[above] at (40:2cm) {\text{10\%}};

\node[above] at (70:2cm) {\text{5\%}};

\node[left] at (120:2cm) {\text{25\%}};

\node[left] at (200:2cm) {\text{20\%}};

\node[below] at (260:2cm) {\text{10\%}};

\node[below] at (310:2cm) {\text{15\%}};


\end{tikzpicture}


%https://me.gateoverflow.in/1712/gate2020-me-1-ga-9
%answer image 1
 \begin{tikzpicture}[>=latex,shorten >=1pt,auto, semithick,node distance=2.8cm,scale = 1]
\node[] at (-0.2,3.4) {$y$};
\node[left] at (0,2.5) {$1$};
\node[left] at (0,-0.2) {$0$};
\node[below] at (2.0,0) {$1$};
\node[] at (3.1,-0.2) {$x$};
\node[] at (1.65,0.6) {$x^{m}$};
\node[] at (1.1,1.6) {$x^{1/m}$};
\draw[->,line width=0.5mm] (0.0,-0.3) -- (0.0,3.5);
\draw[->,line width=0.5mm] (-0.3,0.0) -- (3.0,0.0);
\draw[-,line width=0.45mm,dashed](2.0,0.0) -- (2.0,2.5) -- (0.0,2.5);
\path[-,line width=0.45mm]
[teal] (0.0,0.0) edge [bend left = 35]  node  {} (2.0,2.52)
[teal] (0.0,0.0) edge [bend right = 35]  node  {} (1.98,2.53);
\end{tikzpicture}


%answer image 2
 \begin{tikzpicture}[>=latex,shorten >=1pt,auto, semithick,node distance=2.8cm,scale = 1]
\node[] at (-0.2,3.4) {$y$};
\node[left] at (0,2.5) {$1$};
\node[left] at (0,-0.2) {$0$};
\node[below] at (2.0,0) {$1$};
\node[] at (3.1,-0.2) {$x$};
\node[] at (1.60,0.3) {$x^{1/m}$};
\node[] at (1.1,1.6) {$x^{m}$};
\draw[->,line width=0.5mm] (0.0,-0.3) -- (0.0,3.5);
\draw[->,line width=0.5mm] (-0.3,0.0) -- (3.0,0.0);
\draw[-,line width=0.45mm,dashed](2.0,0.0) -- (2.0,2.5) -- (0.0,2.5);
\path[-,line width=0.45mm]
[teal] (0.0,0.0) edge [bend left = 35]  node  {} (2.0,2.52)
[teal] (0.0,0.0) edge [bend right = 35]  node  {} (1.98,2.53);
\end{tikzpicture}

%answer image 3
\begin{tikzpicture}[>=stealth',shorten >=1pt,auto, semithick,node distance=2.8cm]
\node at (-0.2,3.4) {y};
\node at (-0.2,2.5) {1};
\node at (-0.2,-0.2) {0};
\node at (2.0,-0.2) {1};
\node at (3.1,-0.2) {x};
\node at (1.5,0.3) {$x^{1/m}$};
\node at (0.6,1.0) {$x^{m}$};

\draw[->,line width=0.5mm] (0.0,-0.3) -- (0.0,3.5);
\draw[->,line width=0.5mm] (-0.3,0.0) -- (3.0,0.0);
\draw[-,line width=0.5mm,dashed](2.0,0.0) -- (2.0,2.5);
\draw[-,line width=0.5mm,dashed](0.0,2.5) --(2.0,2.5);

\path [-,line width=0.5mm]
[teal] (0.0,0.0) edge [bend right = 15]  node  {} (2.0,2.5)
[teal] (0.0,0.0) edge [bend right = 35]  node  {} (1.98,2.53);
\end{tikzpicture}

%answer image 4
\begin{tikzpicture}[>=stealth',shorten >=1pt,auto, semithick,node distance=2.8cm]
\node at (-0.2,3.4) {y};
\node at (-0.2,2.5) {1};
\node at (-0.2,-0.2) {0};
\node at (2.0,-0.2) {1};
\node at (3.1,-0.2) {x};
\node at (1.2,1.2) {$x^{m}$};
\node at (0.5,2.2) {$x^{1/m}$};

\draw[->,line width=0.5mm] (0.0,-0.3) -- (0.0,3.5);
\draw[->,line width=0.5mm] (-0.3,0.0) -- (3.0,0.0);
\draw[-,line width=0.5mm,dashed](2.0,0.0) -- (2.0,2.5);
\draw[-,line width=0.5mm,dashed](0.0,2.5) --(2.0,2.5);

\path[-,line width=0.5mm]
[teal] (0.0,0.0) edge [bend left = 15]  node  {} (2.0,2.5)
[teal] (0.0,0.0) edge [bend left = 35]  node  {} (2.0,2.52);
\end{tikzpicture}












%\input{2018.tex}
%\input{2019.tex}
%\input{2020.tex}

\end{document}
