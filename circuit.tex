%https://tikz.dev/library-circuits
%https://texdoc.org/serve/circuitikzmanual.pdf/0

\tikzset{
  multiplexer1/.pic ={
    \coordinate (-out) at (.8,0);
    \coordinate (-in-up) at (-0.8,.5);
    \coordinate (-in-down) at (-0.8,-.5);
     %\coordinate (-sel-down) at (0,0);
    \draw[pic actions] (-.5,1.25) -- ++(1,-.5)
    coordinate[midway](-sel-up) 
   -- ++(0,-1.5) -- ++(-1,-.5)
  coordinate[midway](-sel-down)
   --cycle;
   \draw[pic actions] (.5,0) -- (-out) (-in-up) -- +(.3,0) (-in-down) -- +(.3,0);% (mysel) node[below](-sel-down){};
   % \draw[pic actions] (mysel) --++(0,-1);
       

    \node{ \tikzpictext};
  }
}

%https://ec.gateoverflow.in/523/gate2014-3-16
\begin{tikzpicture}

\pic["MUX"] (one) {multiplexer1};

\node at (one-in-up) [xshift=0.5cm]{0};
\node at (one-in-down) [xshift=0.5cm]{1};
\draw [-] (one-sel-down) --++ (-0.0,-1) node[below]{$S_1$};

\pic["MUX"] (two) at ($(one-out)-(one-in-up)$)[xshift=2cm] {multiplexer1};
\draw [-] (two-sel-down) --++ (-0.0,-1) node[below]{$S_2$};
\node at (two-in-up) [xshift=0.5cm]{0};
\node at (two-in-down) [xshift=0.5cm]{1};

\draw [-] (one-out) -- (two-in-up);
\draw[-] ($(one-out) + (0.5,0)$) |- ($(two-in-down)-(1,0)$);
\node[anchor=input, not gate US, draw] at  (two-in-down) [xshift=-1cm] (not2){};

\draw[-] (two-in-down) -- (not2.output); 
\draw[-] (one-in-up) --++ (-2,0) node [left] (w){$W$};
\node[anchor=input, not gate US, draw] at  (one-in-down) [xshift=-1cm] (not1){};
\draw[-] ($(w.east)+(0.5,0)$) |- (not1.input);
\draw[-] (one-in-down) -- (not1.output); 
\draw[-] (two-out) --++(0.5,0) node[right]{$F$};

\end{tikzpicture}

%https://ec.gateoverflow.in/369/gate2014-1-40
\begin{tikzpicture}[scale=.5]
\node[xnor gate US, draw]  (xnor1){} node [below,yshift=-0.5cm]{XNOR};
\node[xor gate US, draw, anchor=output] at (xnor1.output) [xshift=0cm,yshift=2cm] (xor1) {};
\node [above,yshift=0.5cm] at (xor1) {XOR};
\node[and gate US, draw, anchor=input 1] at (xnor1)[xshift=1.5cm,yshift=1cm] (and1){};
\node [above]at (and1) [yshift=0.5cm]{AND}{};
\coordinate (A) at (0,0);
\coordinate (B) at ($(and1.input 1)-(xor1.output)$);

\draw (xor1.output) -- ([xshift=0.5cm]xor1.output) [name path=line 1]  |-   ( A |- B) ;
\draw (xnor1.output) -- ([xshift=0.5cm]xnor1.output) |- (and1.input 2);
\draw (and1.output) -- ([xshift=0.5cm]and1.output) node [right] {F};

\draw (xor1.input 1) --++ (-1,0) node [yshift=0.1cm,left] {X};
\draw (xor1.input 2) --++ (-1,0) node [yshift=-0.1cm, left] { Y};
\draw (xnor1.input 2) --++ (-1,0) node [left] {Z};
\draw (xnor1.input 1) --++ (-0.5,0) [name path=line 2] |- (and1.input 1);

\path [name intersections={of = line 1 and line 2}];
\coordinate (S)  at (intersection-1);
\node [fill=black,inner sep=1pt] at (S) {};

\end{tikzpicture}



\tikzstyle{branch}=[fill,shape=circle,minimum size=3pt,inner sep=0pt]
\begin{tikzpicture}[label distance=2mm]

    \node (x3) at (0,0) {$x_3$};
    \node (x2) at (1,0) {$x_2$};
    \node (x1) at (2,0) {$x_1$};
    \node (x0) at (3,0) {$x_0$};

    \node[not gate US, draw, rotate=-90] at ($(x2)+(0.5,-1)$) (Not2) {};
    \node[not gate US, draw, rotate=-90] at ($(x1)+(0.5,-1)$) (Not1) {};
    \node[not gate US, draw, rotate=-90] at ($(x0)+(0.5,-1)$) (Not0) {};

    \node[or gate US, draw, logic gate inputs=nnn] at ($(x0)+(2,-2)$) (Or1) {};
    \node[or gate US, draw, logic gate inputs=nnnn] at ($(Or1)+(0,-1)$) (Or2) {};
    \node[or gate US, draw, logic gate inputs=nnn] at ($(Or2)+(0,-1)$) (Or3) {};
    \node[xor gate US, draw, logic gate inputs=nn] at ($(Or3)+(0,-1)$) (Xor1) {};
    \node[and gate US, draw, logic gate inputs=nn, anchor=input 1] at ($(Or3.output)+(1,0)$) (And1) {};
    \node[nor gate US, draw, logic gate inputs=nn, anchor=input 1] at ($(Or2.output -| And1.output)+(1,0)$) (Nor1) {};
    \node[and gate US, draw, logic gate inputs=nn, anchor=input 1] at ($(Or1.output -| Nor1.output)+(1,0)$) (And2) {};

    \foreach \i in {2,1,0}
    {
        \path (x\i) -- coordinate (punt\i) (x\i |- Not\i.input);
        \draw (punt\i) node[branch] {} -| (Not\i.input);
    }
    \draw (x3) |- (Or2.input 1);
    \draw (x3 |- Or1.input 1) node[branch] {} -- (Or1.input 1);
    \draw (x2) |- (Xor1.input 1);
    \draw (x2 |- Or3.input 1) node[branch] {} -- (Or3.input 1);
    \draw (Not2.output) |- (Or2.input 2);
    \draw (x1) |- (Or3.input 2);
    \draw (x1 |- Or1.input 2) node[branch] {} -- (Or1.input 2);
    \draw (Not1.output) |- (Xor1.input 2);
    \draw (Not1.output |- Or2.input 3) node[branch] {} -- (Or2.input 3);
    \draw (x0) |- (Or2.input 4);
    \draw (Not0.output) |- (Or3.input 3);
    \draw (Not0.output |- Or1.input 3) node[branch] {} -- (Or1.input 3);
    \draw (Or3.output) -- (And1.input 1);
    \draw (Xor1.output) -- ([xshift=0.5cm]Xor1.output) |- (And1.input 2);
    \draw (Or2.output) -- (Nor1.input 1);
    \draw (And1.output) -- ([xshift=0.5cm]And1.output) |- (Nor1.input 2);
    \draw (Or1.output) -- (And2.input 1);
    \draw (Nor1.output) -- ([xshift=0.5cm]Nor1.output) |- (And2.input 2);
    \draw (And2.output) -- ([xshift=0.5cm]And2.output) node[above] {$f_1$};

\end{tikzpicture}


\begin{tikzpicture}
  
  \node (c) at (0,0) {};
  \node (a) at (0,-.6) {\footnotesize $a$};
  \node (b) at (0,-1.6) {\footnotesize $b$};
  \node (d) at (0,-2.2) {};
  
  
  \node[and gate US, draw,logic gate inputs=nn] at ($(c) + (1, 0)$) (anda) {};
  \node[and gate US, draw,logic gate inputs=nn] at ($(a) + (1, -0.5)$) (andb) {};
  \node[and gate US, draw,logic gate inputs=nn] at ($(d) + (1, 0)$) (andc) {};
  \node[or gate US, draw, rotate=0, logic gate inputs=nnn] at ($(a) + (2.5, -.5)$) (oro) {};
  
   \draw (a)   --+(0.4,0) |- (anda.input 2);
  \draw (a) -- +(0.4,-.008) -| +(0.4,0) |- (andb.input 1);
   \draw (b)   -| +(0.4,0) |- (andb.input 2);
  \draw (b) -- +(0.4,-.008) -| +(0.4,0) |- (andc.input 1);
  \draw (c)  (0.4,.1) -- (0.7,.1);
  \draw (c) (0.4,.1) |- (0.4,.5) -- (3.2,.5) -| (3.2,-1.1);
  \draw (d)  (0.4,-2.3) -- (0.7,-2.3);
  \draw (d) (0.4,-2.3) |- (0.4,-2.7) -- (3.2,-2.7) -| (3.2,-1.1);
  
  \draw (anda.output) -- ([xshift=0.2cm,yshift=0cm]anda.output) |- (oro.input 1);
  \draw (andb.output) -- ([xshift=0.2cm,yshift=0cm]andb.output) |- (oro.input 2);
  \draw (andc.output) -- ([xshift=0.2cm,yshift=0cm]andc.output) |- (oro.input 3);

  \draw (oro.output) -- node[at end, above]{\tiny $Q$} ($(oro) + (1.3, 0)$);
  
  \filldraw 
(0.4,-.6008) circle (1pt) node[align=left,   left] {} 
(0.4,-1.6008) circle (1pt) node[align=left,   left] {} 
(3.2,-1.1) circle (1pt) node[align=left,   left] {};
  
\end{tikzpicture}
%https://gateoverflow.in/2750/gate1996-2-21
\tikzstyle{branch}=[fill,shape=circle,minimum size=3pt,inner sep=0pt]
\begin{tikzpicture}[
    fulladder/.style={draw, minimum width=5cm,minimum height = 3cm,
    label={[xshift=8mm]left:$c_\text{out}$},
    label={[xshift=-6mm]right:$c_\text{in}$},
    }]

    \node[fulladder] (a) {};
    \node[align=center,anchor=center] (lab) at (a.center) {$4-bit$\\\text{Binary}\\\text{Adder}};
     \node[and gate US, draw, logic gate inputs=nn,rotate=-180] at (-0.6,2.5) (and1) {};
\draw [->] (and1.output) -| (-1.7,1.5);
    \draw[<-] (-2,1.5) --++(90:0.5cm) node [above] {$0$};
    
     \draw[<-] (-1.4,1.5) --++(90:0.5cm) node [above] {$0$};
     \draw[<-] (-1.1,1.5) --++(90:0.5cm) node [above] {$0$};
    
 
    \draw[<-] (0.5,1.5) --++(90:1.5cm) node [above] {$b_3$};
    \draw[<-] (1,1.5) --++(90:1.5cm) node [above] {$b_2$};
    \draw[<-] (1.5,1.5) --++(90:1.5cm) node [above] {$b_1$};
    \draw[<-] (2,1.5) --++(90:1.5cm) node [above] {$b_0$};
    
    \draw[->] (a.west) ( -2.5,0) -|(-3,-2.5) node [below] {$d_4$};
    \draw[<-] (a.east) --++(0:1cm) node [right] {$0$};
    \draw[->] (-0.5,-1.5) --++(-90:1cm) node [below] {$d_3$};
     \draw[->] (-0,-1.5) --++(-90:1cm) node [below] {$d_2$};
      \draw[->] (0.5,-1.5) --++(-90:1cm) node [below] {$d_1$};
       \draw[->] (1,-1.5) --++(-90:1cm) node [below] {$d_0$};
    
    \draw  (and1.input 1) -- (0.5,2.415)node[branch] {};
    \draw  (and1.input 2) -- (1,2.59)node[branch] {};

\end{tikzpicture}

%%% https://gateoverflow.in/118315/gate-cse-2017-set-1-question-33
\begin{tikzpicture}[]
 \def\radius{1.mm} 
  % place dffs and draw connections
   \renewcommand{\inpa}{}
  \renewcommand{\inpb}{}
  \renewcommand{\outa}{}
  \renewcommand{\outb}{}
    \tikzstyle{branch}=[fill,shape=circle,minimum size=3pt,inner sep=0pt]

\pagestyle{empty}

\makeatletter

% Data Flip Flip (DFF) shape
\pgfdeclareshape{dff}{
  % The 'minimum width' and 'minimum height' keys, not the content, determine
  % the size
  \savedanchor\northeast{%
    \pgfmathsetlength\pgf@x{\pgfshapeminwidth}%
    \pgfmathsetlength\pgf@y{\pgfshapeminheight}%
    \pgf@x=0.25\pgf@x
    \pgf@y=0.25\pgf@y
  }
  % This is redundant, but makes some things easier:
  \savedanchor\southwest{%
    \pgfmathsetlength\pgf@x{\pgfshapeminwidth}%
    \pgfmathsetlength\pgf@y{\pgfshapeminheight}%
    \pgf@x=-0.25\pgf@x
    \pgf@y=-0.25\pgf@y
  }
  % Inherit from rectangle
  \inheritanchorborder[from=rectangle]

  % Define same anchor a normal rectangle has
  \anchor{center}{\pgfpointorigin}
  \anchor{north}{\northeast \pgf@x=0pt}
  \anchor{east}{\northeast \pgf@y=0pt}
  \anchor{south}{\southwest \pgf@x=0pt}
  \anchor{west}{\southwest \pgf@y=0pt}
  \anchor{north east}{\northeast}
  \anchor{north west}{\northeast \pgf@x=-\pgf@x}
  \anchor{south west}{\southwest}
  \anchor{south east}{\southwest \pgf@x=-\pgf@x}
  \anchor{text}{
    \pgfpointorigin
    \advance\pgf@x by -.5\wd\pgfnodeparttextbox%
    \advance\pgf@y by -.5\ht\pgfnodeparttextbox%
    \advance\pgf@y by +.5\dp\pgfnodeparttextbox%
  }

  % Define anchors for signal ports
  \anchor{D}{
    \pgf@process{\northeast}%
    \pgf@x=-1\pgf@x%
    \pgf@y=.5\pgf@y%
  }
  \anchor{K}{
    \pgf@process{\northeast}%
    \pgf@x=-1\pgf@x%
    \pgf@y=-.5\pgf@y%
  }
  \anchor{Q}{
    \pgf@process{\northeast}%
    \pgf@y=.5\pgf@y%
  }
  \anchor{Qn}{
    \pgf@process{\northeast}%
    \pgf@y=-.5\pgf@y%
  }
    \anchor{CLK}{
    \pgf@process{\northeast}%
    \pgf@x=0\pgf@x%
    \pgf@y=-1\pgf@y%
  }
  % Draw the rectangle box and the port labels
  \backgroundpath{
    % Rectangle box
    \pgfpathrectanglecorners{\southwest}{\northeast}
    % Angle (>) for clock input
   \pgf@anchor@dff@CLK
   \pgf@xa=\pgf@x \pgf@ya=\pgf@y
    \pgf@xb=\pgf@x \pgf@yb=\pgf@y
    \pgf@xc=\pgf@x \pgf@yc=\pgf@y
   \pgfmathsetlength\pgf@x{0.8ex} % size depends on font size
   \advance\pgf@xa by \pgf@x
   % \advance\pgf@xb by -\pgf@x
     \advance\pgf@xc by -\pgf@x
    \advance\pgf@yb by \pgf@x
    \pgfpathmoveto{\pgfpoint{\pgf@xa}{\pgf@ya}}
  \pgfpathlineto{\pgfpoint{\pgf@xb}{\pgf@yb}}
  \pgfpathlineto{\pgfpoint{\pgf@xc}{\pgf@yc}}
   \pgfclosepath
    
    %\pgfpathrectanglecorners{\southwest}{\northeast}
    % Angle (>) for clock input
%   \pgf@anchor@dff@CLK
%   \pgf@xa=\pgf@x \pgf@ya=\pgf@y
%     \pgf@xb=\pgf@x \pgf@yb=\pgf@y
%     \pgf@xc=\pgf@x \pgf@yc=\pgf@y
%   \pgfmathsetlength\pgf@x{0.8ex} % size depends on font size
%   \advance\pgf@ya by \pgf@x
%     \advance\pgf@xb by \pgf@x
%     \advance\pgf@yc by -\pgf@x
%     \pgfpathmoveto{\pgfpoint{\pgf@xa}{\pgf@ya}}
%   \pgfpathlineto{\pgfpoint{\pgf@xb}{\pgf@yb}}
%   \pgfpathlineto{\pgfpoint{\pgf@xc}{\pgf@yc}}
%   \pgfclosepath

    % Draw port labels
    \begingroup
    \tikzset{flip flop/port labels} % Use font from this style
    \tikz@textfont

    \pgf@anchor@dff@D
    \pgftext[left,base,at={\pgfpoint{\pgf@x}{\pgf@y}},x=\pgfshapeinnerxsep]{\raisebox{-0.75ex}{\inpa}}
    
    \pgf@anchor@dff@K
    \pgftext[left,base,at={\pgfpoint{\pgf@x}{\pgf@y}},x=\pgfshapeinnerxsep]{\raisebox{-0.65ex}{\inpb}}

    \pgf@anchor@dff@Q
    \pgftext[right,base,at={\pgfpoint{\pgf@x}{\pgf@y}},x=-\pgfshapeinnerxsep]{\raisebox{-.75ex}{\outa}}
   \pgf@anchor@dff@Q
 
    \pgf@anchor@dff@Qn
    \pgftext[right,base,at={\pgfpoint{\pgf@x}{\pgf@y}},x=-\pgfshapeinnerxsep]{\raisebox{-.75ex}{\outb}}

    \endgroup
  }
}

% Key to add font macros to the current font
\tikzset{add font/.code={\expandafter\def\expandafter\tikz@textfont\expandafter{\tikz@textfont#1}}}

% Define default style for this node
\tikzset{flip flop/port labels/.style={font=\sffamily\scriptsize}}
\tikzset{every dff node/.style={draw,minimum width=2.5cm,minimum
    height=3.328427125cm,very thick,inner sep=1mm,outer sep=0pt,cap=round,add
    font=\sffamily}}

\makeatother


     \tikzstyle{branch}=[fill,shape=circle,minimum size=3pt,inner sep=0pt]

\pagestyle{empty}

\makeatletter

% Data Flip Flip (DFF) shape
\pgfdeclareshape{dff}{
  % The 'minimum width' and 'minimum height' keys, not the content, determine
  % the size
  \savedanchor\northeast{%
    \pgfmathsetlength\pgf@x{\pgfshapeminwidth}%
    \pgfmathsetlength\pgf@y{\pgfshapeminheight}%
    \pgf@x=0.25\pgf@x
    \pgf@y=0.25\pgf@y
  }
  % This is redundant, but makes some things easier:
  \savedanchor\southwest{%
    \pgfmathsetlength\pgf@x{\pgfshapeminwidth}%
    \pgfmathsetlength\pgf@y{\pgfshapeminheight}%
    \pgf@x=-0.25\pgf@x
    \pgf@y=-0.25\pgf@y
  }
  % Inherit from rectangle
  \inheritanchorborder[from=rectangle]

  % Define same anchor a normal rectangle has
  \anchor{center}{\pgfpointorigin}
  \anchor{north}{\northeast \pgf@x=0pt}
  \anchor{east}{\northeast \pgf@y=0pt}
  \anchor{south}{\southwest \pgf@x=0pt}
  \anchor{west}{\southwest \pgf@y=0pt}
  \anchor{north east}{\northeast}
  \anchor{north west}{\northeast \pgf@x=-\pgf@x}
  \anchor{south west}{\southwest}
  \anchor{south east}{\southwest \pgf@x=-\pgf@x}
  \anchor{text}{
    \pgfpointorigin
    \advance\pgf@x by -.5\wd\pgfnodeparttextbox%
    \advance\pgf@y by -.5\ht\pgfnodeparttextbox%
    \advance\pgf@y by +.5\dp\pgfnodeparttextbox%
  }

  % Define anchors for signal ports
  \anchor{D}{
    \pgf@process{\northeast}%
    \pgf@x=-1\pgf@x%
    \pgf@y=.5\pgf@y%
  }
  \anchor{K}{
    \pgf@process{\northeast}%
    \pgf@x=-1\pgf@x%
    \pgf@y=-.5\pgf@y%
  }
  \anchor{Q}{
    \pgf@process{\northeast}%
    \pgf@y=.5\pgf@y%
  }
  \anchor{Qn}{
    \pgf@process{\northeast}%
    \pgf@y=-.5\pgf@y%
  }
    \anchor{CLK}{
    \pgf@process{\northeast}%
    \pgf@x=-1\pgf@x%
    \pgf@y=0\pgf@y%
  }
  % Draw the rectangle box and the port labels
  \backgroundpath{
    % Rectangle box
    \pgfpathrectanglecorners{\southwest}{\northeast}
    % Angle (>) for clock input
 % \pgf@anchor@dff@CLK
  %\pgf@xa=\pgf@x \pgf@ya=\pgf@y
   %\pgf@xb=\pgf@x \pgf@yb=\pgf@y
  % \pgf@xc=\pgf@x \pgf@yc=\pgf@y
   %\pgfmathsetlength\pgf@x{0.8ex} % size depends on font size
  %\advance\pgf@xa by \pgf@x
  %%\advance\pgf@xb by -\pgf@x
   %  \advance\pgf@xc by -\pgf@x
    %\advance\pgf@yb by \pgf@x
    %\pgfpathmoveto{\pgfpoint{\pgf@xa}{\pgf@ya}}
 %\pgfpathlineto{\pgfpoint{\pgf@xb}{\pgf@yb}}
 %\pgfpathlineto{\pgfpoint{\pgf@xc}{\pgf@yc}}
  % \pgfclosepath
    
    %\pgfpathrectanglecorners{\southwest}{\northeast}
    % Angle (>) for clock input
  \pgf@anchor@dff@CLK
  \pgf@xa=\pgf@x \pgf@ya=\pgf@y
    \pgf@xb=\pgf@x \pgf@yb=\pgf@y
    \pgf@xc=\pgf@x \pgf@yc=\pgf@y
  \pgfmathsetlength\pgf@x{0.8ex} % size depends on font size
  \advance\pgf@ya by \pgf@x
 \advance\pgf@xb by \pgf@x
    \advance\pgf@yc by -\pgf@x
   \pgfpathmoveto{\pgfpoint{\pgf@xa}{\pgf@ya}}
  \pgfpathlineto{\pgfpoint{\pgf@xb}{\pgf@yb}}
  \pgfpathlineto{\pgfpoint{\pgf@xc}{\pgf@yc}}
  \pgfclosepath

    % Draw port labels
    \begingroup
    \tikzset{flip flop/port labels} % Use font from this style
    \tikz@textfont

    \pgf@anchor@dff@D
    \pgftext[left,base,at={\pgfpoint{\pgf@x}{\pgf@y}},x=\pgfshapeinnerxsep]{\raisebox{-0.75ex}{\inpa}}
    
    \pgf@anchor@dff@K
    \pgftext[left,base,at={\pgfpoint{\pgf@x}{\pgf@y}},x=\pgfshapeinnerxsep]{\raisebox{-0.65ex}{\inpb}}

    \pgf@anchor@dff@Q
    \pgftext[right,base,at={\pgfpoint{\pgf@x}{\pgf@y}},x=-\pgfshapeinnerxsep]{\raisebox{-.75ex}{\outa}}
   \pgf@anchor@dff@Q
 
    \pgf@anchor@dff@Qn
    \pgftext[right,base,at={\pgfpoint{\pgf@x}{\pgf@y}},x=-\pgfshapeinnerxsep]{\raisebox{-.75ex}{\outb}}

    \endgroup
  }
}

% Key to add font macros to the current font
\tikzset{add font/.code={\expandafter\def\expandafter\tikz@textfont\expandafter{\tikz@textfont#1}}}

% Define default style for this node
\tikzset{flip flop/port labels/.style={font=\sffamily\scriptsize}}
\tikzset{every dff node/.style={draw,minimum width=2.5cm,minimum
    height=3.328427125cm,very thick,inner sep=1mm,outer sep=0pt,cap=round,add
    font=\sffamily}}

\makeatother

 
  \node [shape=dff] (DFF0) at (0,0) {};
    \renewcommand{\inpa}{}
  \renewcommand{\inpb}{}
  \renewcommand{\outa}{}
  \renewcommand{\outb}{}
  \tikzstyle{branch}=[fill,shape=circle,minimum size=3pt,inner sep=0pt]

\pagestyle{empty}

\makeatletter

% Data Flip Flip (DFF) shape
\pgfdeclareshape{dff}{
  % The 'minimum width' and 'minimum height' keys, not the content, determine
  % the size
  \savedanchor\northeast{%
    \pgfmathsetlength\pgf@x{\pgfshapeminwidth}%
    \pgfmathsetlength\pgf@y{\pgfshapeminheight}%
    \pgf@x=0.25\pgf@x
    \pgf@y=0.25\pgf@y
  }
  % This is redundant, but makes some things easier:
  \savedanchor\southwest{%
    \pgfmathsetlength\pgf@x{\pgfshapeminwidth}%
    \pgfmathsetlength\pgf@y{\pgfshapeminheight}%
    \pgf@x=-0.25\pgf@x
    \pgf@y=-0.25\pgf@y
  }
  % Inherit from rectangle
  \inheritanchorborder[from=rectangle]

  % Define same anchor a normal rectangle has
  \anchor{center}{\pgfpointorigin}
  \anchor{north}{\northeast \pgf@x=0pt}
  \anchor{east}{\northeast \pgf@y=0pt}
  \anchor{south}{\southwest \pgf@x=0pt}
  \anchor{west}{\southwest \pgf@y=0pt}
  \anchor{north east}{\northeast}
  \anchor{north west}{\northeast \pgf@x=-\pgf@x}
  \anchor{south west}{\southwest}
  \anchor{south east}{\southwest \pgf@x=-\pgf@x}
  \anchor{text}{
    \pgfpointorigin
    \advance\pgf@x by -.5\wd\pgfnodeparttextbox%
    \advance\pgf@y by -.5\ht\pgfnodeparttextbox%
    \advance\pgf@y by +.5\dp\pgfnodeparttextbox%
  }

  % Define anchors for signal ports
  \anchor{D}{
    \pgf@process{\northeast}%
    \pgf@x=-1\pgf@x%
    \pgf@y=.5\pgf@y%
  }
  \anchor{K}{
    \pgf@process{\northeast}%
    \pgf@x=-1\pgf@x%
    \pgf@y=-.5\pgf@y%
  }
  \anchor{Q}{
    \pgf@process{\northeast}%
    \pgf@y=.5\pgf@y%
  }
  \anchor{Qn}{
    \pgf@process{\northeast}%
    \pgf@y=-.5\pgf@y%
  }
    \anchor{CLK}{
    \pgf@process{\northeast}%
    \pgf@x=0\pgf@x%
    \pgf@y=-1\pgf@y%
  }
  % Draw the rectangle box and the port labels
  \backgroundpath{
    % Rectangle box
    \pgfpathrectanglecorners{\southwest}{\northeast}
    % Angle (>) for clock input
   \pgf@anchor@dff@CLK
   \pgf@xa=\pgf@x \pgf@ya=\pgf@y
    \pgf@xb=\pgf@x \pgf@yb=\pgf@y
    \pgf@xc=\pgf@x \pgf@yc=\pgf@y
   \pgfmathsetlength\pgf@x{0.8ex} % size depends on font size
   \advance\pgf@xa by \pgf@x
   % \advance\pgf@xb by -\pgf@x
     \advance\pgf@xc by -\pgf@x
    \advance\pgf@yb by \pgf@x
    \pgfpathmoveto{\pgfpoint{\pgf@xa}{\pgf@ya}}
  \pgfpathlineto{\pgfpoint{\pgf@xb}{\pgf@yb}}
  \pgfpathlineto{\pgfpoint{\pgf@xc}{\pgf@yc}}
   \pgfclosepath
    
    %\pgfpathrectanglecorners{\southwest}{\northeast}
    % Angle (>) for clock input
%   \pgf@anchor@dff@CLK
%   \pgf@xa=\pgf@x \pgf@ya=\pgf@y
%     \pgf@xb=\pgf@x \pgf@yb=\pgf@y
%     \pgf@xc=\pgf@x \pgf@yc=\pgf@y
%   \pgfmathsetlength\pgf@x{0.8ex} % size depends on font size
%   \advance\pgf@ya by \pgf@x
%     \advance\pgf@xb by \pgf@x
%     \advance\pgf@yc by -\pgf@x
%     \pgfpathmoveto{\pgfpoint{\pgf@xa}{\pgf@ya}}
%   \pgfpathlineto{\pgfpoint{\pgf@xb}{\pgf@yb}}
%   \pgfpathlineto{\pgfpoint{\pgf@xc}{\pgf@yc}}
%   \pgfclosepath

    % Draw port labels
    \begingroup
    \tikzset{flip flop/port labels} % Use font from this style
    \tikz@textfont

    \pgf@anchor@dff@D
    \pgftext[left,base,at={\pgfpoint{\pgf@x}{\pgf@y}},x=\pgfshapeinnerxsep]{\raisebox{-0.75ex}{\inpa}}
    
    \pgf@anchor@dff@K
    \pgftext[left,base,at={\pgfpoint{\pgf@x}{\pgf@y}},x=\pgfshapeinnerxsep]{\raisebox{-0.65ex}{\inpb}}

    \pgf@anchor@dff@Q
    \pgftext[right,base,at={\pgfpoint{\pgf@x}{\pgf@y}},x=-\pgfshapeinnerxsep]{\raisebox{-.75ex}{\outa}}
   \pgf@anchor@dff@Q
 
    \pgf@anchor@dff@Qn
    \pgftext[right,base,at={\pgfpoint{\pgf@x}{\pgf@y}},x=-\pgfshapeinnerxsep]{\raisebox{-.75ex}{\outb}}

    \endgroup
  }
}

% Key to add font macros to the current font
\tikzset{add font/.code={\expandafter\def\expandafter\tikz@textfont\expandafter{\tikz@textfont#1}}}

% Define default style for this node
\tikzset{flip flop/port labels/.style={font=\sffamily\scriptsize}}
\tikzset{every dff node/.style={draw,minimum width=2.5cm,minimum
    height=3.328427125cm,very thick,inner sep=1mm,outer sep=0pt,cap=round,add
    font=\sffamily}}

\makeatother


   \tikzstyle{branch}=[fill,shape=circle,minimum size=3pt,inner sep=0pt]

\pagestyle{empty}

\makeatletter

% Data Flip Flip (DFF) shape
\pgfdeclareshape{dff}{
  % The 'minimum width' and 'minimum height' keys, not the content, determine
  % the size
  \savedanchor\northeast{%
    \pgfmathsetlength\pgf@x{\pgfshapeminwidth}%
    \pgfmathsetlength\pgf@y{\pgfshapeminheight}%
    \pgf@x=0.25\pgf@x
    \pgf@y=0.25\pgf@y
  }
  % This is redundant, but makes some things easier:
  \savedanchor\southwest{%
    \pgfmathsetlength\pgf@x{\pgfshapeminwidth}%
    \pgfmathsetlength\pgf@y{\pgfshapeminheight}%
    \pgf@x=-0.25\pgf@x
    \pgf@y=-0.25\pgf@y
  }
  % Inherit from rectangle
  \inheritanchorborder[from=rectangle]

  % Define same anchor a normal rectangle has
  \anchor{center}{\pgfpointorigin}
  \anchor{north}{\northeast \pgf@x=0pt}
  \anchor{east}{\northeast \pgf@y=0pt}
  \anchor{south}{\southwest \pgf@x=0pt}
  \anchor{west}{\southwest \pgf@y=0pt}
  \anchor{north east}{\northeast}
  \anchor{north west}{\northeast \pgf@x=-\pgf@x}
  \anchor{south west}{\southwest}
  \anchor{south east}{\southwest \pgf@x=-\pgf@x}
  \anchor{text}{
    \pgfpointorigin
    \advance\pgf@x by -.5\wd\pgfnodeparttextbox%
    \advance\pgf@y by -.5\ht\pgfnodeparttextbox%
    \advance\pgf@y by +.5\dp\pgfnodeparttextbox%
  }

  % Define anchors for signal ports
  \anchor{D}{
    \pgf@process{\northeast}%
    \pgf@x=-1\pgf@x%
    \pgf@y=.5\pgf@y%
  }
  \anchor{K}{
    \pgf@process{\northeast}%
    \pgf@x=-1\pgf@x%
    \pgf@y=-.5\pgf@y%
  }
  \anchor{Q}{
    \pgf@process{\northeast}%
    \pgf@y=.5\pgf@y%
  }
  \anchor{Qn}{
    \pgf@process{\northeast}%
    \pgf@y=-.5\pgf@y%
  }
    \anchor{CLK}{
    \pgf@process{\northeast}%
    \pgf@x=-1\pgf@x%
    \pgf@y=0\pgf@y%
  }
  % Draw the rectangle box and the port labels
  \backgroundpath{
    % Rectangle box
    \pgfpathrectanglecorners{\southwest}{\northeast}
    % Angle (>) for clock input
 % \pgf@anchor@dff@CLK
  %\pgf@xa=\pgf@x \pgf@ya=\pgf@y
   %\pgf@xb=\pgf@x \pgf@yb=\pgf@y
  % \pgf@xc=\pgf@x \pgf@yc=\pgf@y
   %\pgfmathsetlength\pgf@x{0.8ex} % size depends on font size
  %\advance\pgf@xa by \pgf@x
  %%\advance\pgf@xb by -\pgf@x
   %  \advance\pgf@xc by -\pgf@x
    %\advance\pgf@yb by \pgf@x
    %\pgfpathmoveto{\pgfpoint{\pgf@xa}{\pgf@ya}}
 %\pgfpathlineto{\pgfpoint{\pgf@xb}{\pgf@yb}}
 %\pgfpathlineto{\pgfpoint{\pgf@xc}{\pgf@yc}}
  % \pgfclosepath
    
    %\pgfpathrectanglecorners{\southwest}{\northeast}
    % Angle (>) for clock input
  \pgf@anchor@dff@CLK
  \pgf@xa=\pgf@x \pgf@ya=\pgf@y
    \pgf@xb=\pgf@x \pgf@yb=\pgf@y
    \pgf@xc=\pgf@x \pgf@yc=\pgf@y
  \pgfmathsetlength\pgf@x{0.8ex} % size depends on font size
  \advance\pgf@ya by \pgf@x
 \advance\pgf@xb by \pgf@x
    \advance\pgf@yc by -\pgf@x
   \pgfpathmoveto{\pgfpoint{\pgf@xa}{\pgf@ya}}
  \pgfpathlineto{\pgfpoint{\pgf@xb}{\pgf@yb}}
  \pgfpathlineto{\pgfpoint{\pgf@xc}{\pgf@yc}}
  \pgfclosepath

    % Draw port labels
    \begingroup
    \tikzset{flip flop/port labels} % Use font from this style
    \tikz@textfont

    \pgf@anchor@dff@D
    \pgftext[left,base,at={\pgfpoint{\pgf@x}{\pgf@y}},x=\pgfshapeinnerxsep]{\raisebox{-0.75ex}{\inpa}}
    
    \pgf@anchor@dff@K
    \pgftext[left,base,at={\pgfpoint{\pgf@x}{\pgf@y}},x=\pgfshapeinnerxsep]{\raisebox{-0.65ex}{\inpb}}

    \pgf@anchor@dff@Q
    \pgftext[right,base,at={\pgfpoint{\pgf@x}{\pgf@y}},x=-\pgfshapeinnerxsep]{\raisebox{-.75ex}{\outa}}
   \pgf@anchor@dff@Q
 
    \pgf@anchor@dff@Qn
    \pgftext[right,base,at={\pgfpoint{\pgf@x}{\pgf@y}},x=-\pgfshapeinnerxsep]{\raisebox{-.75ex}{\outb}}

    \endgroup
  }
}

% Key to add font macros to the current font
\tikzset{add font/.code={\expandafter\def\expandafter\tikz@textfont\expandafter{\tikz@textfont#1}}}

% Define default style for this node
\tikzset{flip flop/port labels/.style={font=\sffamily\scriptsize}}
\tikzset{every dff node/.style={draw,minimum width=2.5cm,minimum
    height=3.328427125cm,very thick,inner sep=1mm,outer sep=0pt,cap=round,add
    font=\sffamily}}

\makeatother

 
  \node [shape=dff] (DFF1) at (3,0) {};
  
  \draw[ thick] (DFF1.Q) -| ($(DFF1.Q)+(0.5,0.7)$) -- ($(DFF0.D)+(-0.5,0.7)$) |- (DFF0.D);
  \draw [-,thick] (DFF0.Q) -- (DFF1.D);
  
  \node (A) at (-2.1,-1.3) {\textbf{\footnotesize{Clock}}};
   \draw[thick] (-1.5,-1.3)-- (1.9,-1.3) ;
    \draw[-, thick] (-1.1,-1.3) -- (-1.1,0)  -- ++(0.5,0);
      \draw[-, thick] (1.9,-1.3) -- (1.9,0)  -- ++(0.5,0);
        
     \node[xshift=-0.28cm,text width=0.05cm]{\textbf{T Flip-Flop}};
     \node[text width=0.05cm, xshift=2.7cm]{\textbf{D Flip-Flop}};
     \node[xshift=1cm]{$Q_1$};
     \node[xshift=4cm]{$Q_0$};
 
\end{tikzpicture}


