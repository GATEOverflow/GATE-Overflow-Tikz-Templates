\usepackage{tikz}
\usepackage{karnaugh-map}
\usepackage{tikz-timing}[2009/05/15]
\usepackage{natbib}
\usepackage{graphicx}
\usepackage{rubikcube,rubikrotation,rubikpatterns}
\usepackage{circuitikz}
\usepackage{pgf}
\usepackage{makecell}
\usepackage{verbatim}
\usepackage{ulem,tikz}
\usepackage{amsmath}
\usepackage{pifont}
\usepackage{amssymb}
\usetikzlibrary{patterns}
\usepackage{pgf-pie}
\usepackage{xcolor}
\usepackage[margin=0.5in]{geometry}
\usepackage{pgfplots}
\usepgfplotslibrary{fillbetween}
\usepackage[active,tightpage]{preview}
\PreviewEnvironment{tikzpicture}
%\PreviewEnvironment{qrcode}
%\usepackage{tikz-qtree}
\usepackage{tkz-euclide}
\usetikzlibrary{decorations.pathmorphing,backgrounds,fit,petri}
\usetikzlibrary{positioning,arrows.meta}
%\usepackage{tikz-qtree}
\usetikzlibrary{trees,calc,arrows.meta,positioning,decorations.pathreplacing,bending}
\usetikzlibrary{arrows}
\usetikzlibrary{shapes.geometric}
\usepackage{pgfplots}
\newcommand{\dname}{$D_0$}
\newcommand{\qname}{$Q_0$}
 \newcommand{\inpa}{$J_0$}
  \newcommand{\inpb}{$K_0$}
  \newcommand{\outa}{$Q_0$}
  \newcommand{\outb}{$\overline{Q_0}$}
\newcommand{\midarrow}{\tikz \draw[-triangle 90] (0,0) -- +(.1,0);}
\newcommand{\implyarrow}{%
  \mathrel{\raisebox{1.3ex}{\rotatebox[origin=c]{90}{\mathhexbox37F}}}}

\usetikzlibrary{trees,calc,quotes} % this is to allow the fork right path


\setlength\PreviewBorder{5pt}%
\usetikzlibrary{decorations.pathmorphing,backgrounds,fit,petri}
\usetikzlibrary{arrows,shapes.gates.logic.US,shapes.gates.logic.IEC,calc,automata, patterns, angles, quotes}
\usetikzlibrary{decorations.pathreplacing,decorations.markings}
\usetikzlibrary{positioning,arrows.meta}

\usetikzlibrary{intersections}
%\usepackage{tkz-euclide}
\tikzset{
  multiplexer/.pic ={
    \coordinate (-out) at (.8,0);
    \coordinate (-in-up) at (-0.8,.5);
    \coordinate (-in-down) at (-0.8,-.5);
     \coordinate (-select) at (0,-1);
    \draw[pic actions] (-.5,1.25) -- ++(1,-.5) -- ++(0,-1.5) -- ++(-1,-.5) --cycle;
    \draw[pic actions] (.5,0) -- (-out) (-in-up) -- +(.3,0) (-in-down) -- +(.3,0);
    \node{ \tikzpictext};
  }
}
\def \setA{ (0,0) circle (1cm) }
\def \setB{ (1.5,0) circle (1cm) }
\def \setC{ (60:1.5) circle (1cm) }
\def \setU{ (-2, -1.5) rectangle (3.5, 2.75) }
%\pgfplotsset{compat=1.11}
%\usepgfplotslibrary{fillbetween}
\begin{comment}
:Title: Timing diagram with the tikz-timing package
:Slug: tikz-timing

This example shows how to make a timing diagram with the `tikz-timing package`_. 
This timing diagram was used by the package author in a recent work and 
shows several clock and pulse signals. The relationship between the clock and 
signal edges is shown using horizontal lines.

Timing diagrams like this can be done using the ``tikztimingtable`` 
environment which has the same syntax as a ``tabular`` environment with two 
columns. The first column holds the signal name, the second one the timing 
characters. See the `package manual`_  for detailed information about them.

.. _tikz-timing package: http://www.ctan.org/tex-archive/help/Catalogue/entries/tikz-timing.html
.. _package manual: http://www.ctan.org/tex-archive/graphics/pgf/contrib/tikz-timing/tikz-timing.pdf

\end{comment}

\pagestyle{empty}
\def\degr{${}^\circ$}

\newif\ifcomment

% Activate the following line to compile document with comments:
\commenttrue
\newcommand{\implicantcostats}[4][0]{
    \draw[rounded corners=3pt, fill=#4, opacity=0.3] ($(rf.east |- #2.north)+(90:#1)$)-| ($(#2.east)+(0:#1)$) |- ($(rf.east |- #3.south)+(-90:#1)$);
    \draw[rounded corners=3pt, fill=#4, opacity=0.3] ($(cf.west |- #2.north)+(90:#1)$) -| ($(#3.west)+(180:#1)$) |- ($(cf.west |- #3.south)+(-90:#1)$);
}

%group top-bottom borders
%#1 - Optional. Space between node and grouping line. Default=0
%#2 - top left node
%#3 - bottom right node
%#4 - filling color
\newcommand{\implicantdaltbaix}[4][0]{
    \draw[rounded corners=3pt, fill=#4, opacity=0.3] ($(cf.south -| #2.west)+(180:#1)$) |- ($(#2.south)+(-90:#1)$) -| ($(cf.south -| #3.east)+(0:#1)$);
    \draw[rounded corners=3pt, fill=#4, opacity=0.3] ($(rf.north -| #2.west)+(180:#1)$) |- ($(#3.north)+(90:#1)$) -| ($(rf.north -| #3.east)+(0:#1)$);
}

%group corners
%#1 - Optional. Space between node and grouping line. Default=0
%#2 - filling color
\newcommand{\implicantcantons}[2][0]{
    \draw[rounded corners=3pt, opacity=.3] ($(rf.east |- 0.south)+(-90:#1)$) -| ($(0.east |- cf.south)+(0:#1)$);
    \draw[rounded corners=3pt, opacity=.3] ($(rf.east |- 8.north)+(90:#1)$) -| ($(8.east |- rf.north)+(0:#1)$);
    \draw[rounded corners=3pt, opacity=.3] ($(cf.west |- 2.south)+(-90:#1)$) -| ($(2.west |- cf.south)+(180:#1)$);
    \draw[rounded corners=3pt, opacity=.3] ($(cf.west |- 10.north)+(90:#1)$) -| ($(10.west |- rf.north)+(180:#1)$);
    \fill[rounded corners=3pt, fill=#2, opacity=.3] ($(rf.east |- 0.south)+(-90:#1)$) -|  ($(0.east |- cf.south)+(0:#1)$) [sharp corners] ($(rf.east |- 0.south)+(-90:#1)$) |-  ($(0.east |- cf.south)+(0:#1)$) ;
    \fill[rounded corners=3pt, fill=#2, opacity=.3] ($(rf.east |- 8.north)+(90:#1)$) -| ($(8.east |- rf.north)+(0:#1)$) [sharp corners] ($(rf.east |- 8.north)+(90:#1)$) |- ($(8.east |- rf.north)+(0:#1)$) ;
    \fill[rounded corners=3pt, fill=#2, opacity=.3] ($(cf.west |- 2.south)+(-90:#1)$) -| ($(2.west |- cf.south)+(180:#1)$) [sharp corners]($(cf.west |- 2.south)+(-90:#1)$) |- ($(2.west |- cf.south)+(180:#1)$) ;
    \fill[rounded corners=3pt, fill=#2, opacity=.3] ($(cf.west |- 10.north)+(90:#1)$) -| ($(10.west |- rf.north)+(180:#1)$) [sharp corners] ($(cf.west |- 10.north)+(90:#1)$) |- ($(10.west |- rf.north)+(180:#1)$) ;
}

\usetikzlibrary{matrix,calc}

%isolated term
%#1 - Optional. Space between node and grouping line. Default=0
%#2 - node
%#3 - filling color
\newcommand{\implicantsol}[3][0]{
    \draw[rounded corners=3pt, fill=#3, opacity=0.3] ($(#2.north west)+(135:#1)$) rectangle ($(#2.south east)+(-45:#1)$);

    }

%Defines 8 or 16 values (0,1,X)
\newcommand{\contingutz}[1]{%
\foreach \x [count=\xi from 0]  in {#1}
     \path (\xi) node {\x};
}

%Places 1 in listed positions
\newcommand{\mintermsz}[1]{%
    \foreach \x in {#1}
        \path (\x) node {1};
}

%Places 0 in listed positions
\newcommand{\maxtermsz}[1]{%
    \foreach \x in {#1}
        \path (\x) node {0};
}

%Places X in listed positions
\newcommand{\indeterminatsz}[1]{%
    \foreach \x in {#1}
        \path (\x) node {X};
}



\newcommand{\stargraph}[2]{\begin{tikzpicture}
    \node[circle,draw=black,minimum size=0.75cm] at (360:0mm) (center) {0};
    \foreach \n in {1,...,#1}{
        \node[circle,draw=black,minimum size=0.75cm] at ({-(\n-1)*240/#1}:#2cm) (n\n) {\n};
        \draw (center)->(n\n);
        %\node at (0,-#2*1.5) ;%{$K_{1,#1}$}; % delete line to remove label
    }
     \node[circle,draw=black,minimum size=0.75cm] at ({-(#1)*240/(#1)}:#2cm) (n887) {$\ldots$};
        \draw[densely dotted] (center)->(n887);
        \node[circle,draw=black] at ({-(#1+1)*240/(#1)}:#2cm) (n888) {$\ldots$};
        \draw[densely dotted] (center)->(n888);
        \node[circle,draw=black,minimum size=0.75cm] at ({1*240/#1}:#2cm) (n889) {\tiny{n-1}};
        \draw (center)->(n889);
\end{tikzpicture}}

\usepackage{pgfplotstable}
\pgfplotsset{compat=1.11,
        /pgfplots/ybar legend/.style={
        /pgfplots/legend image code/.code={%
        %\draw[##1,/tikz/.cd,yshift=-0.25em]
                %(0cm,0cm) rectangle (3pt,0.8em);},
        \draw[##1,/tikz/.cd,bar width=3pt,yshift=-0.2em,bar shift=0pt]
                plot coordinates {(0cm,0.8em)};},
},
}