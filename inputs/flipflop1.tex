
\usetikzlibrary{calc,arrows}
\usetikzlibrary{arrows,shapes.gates.logic.US,shapes.gates.logic.IEC,calc}
\tikzstyle{branch}=[fill,shape=circle,minimum size=3pt,inner sep=0pt]
%\pagestyle{empty}

\makeatletter

% Data Flip Flip (DFF) shape
\pgfdeclareshape{dff}{
  % The 'minimum width' and 'minimum height' keys, not the content, determine
  % the size
  \savedanchor\northeast{%
    \pgfmathsetlength\pgf@x{\pgfshapeminwidth}%
    \pgfmathsetlength\pgf@y{\pgfshapeminheight}%
    \pgf@x=0.25\pgf@x
    \pgf@y=0.25\pgf@y
  }
  % This is redundant, but makes some things easier:
  \savedanchor\southwest{%
    \pgfmathsetlength\pgf@x{\pgfshapeminwidth}%
    \pgfmathsetlength\pgf@y{\pgfshapeminheight}%
    \pgf@x=-0.25\pgf@x
    \pgf@y=-0.25\pgf@y
  }
  % Inherit from rectangle
  \inheritanchorborder[from=rectangle]

  % Define same anchor a normal rectangle has
  \anchor{center}{\pgfpointorigin}
  \anchor{north}{\northeast \pgf@x=0pt}
  \anchor{east}{\northeast \pgf@y=0pt}
  \anchor{south}{\southwest \pgf@x=0pt}
  \anchor{west}{\southwest \pgf@y=0pt}
  \anchor{north east}{\northeast}
  \anchor{north west}{\northeast \pgf@x=-\pgf@x}
  \anchor{south west}{\southwest}
  \anchor{south east}{\southwest \pgf@x=-\pgf@x}
  \anchor{text}{
    \pgfpointorigin
    \advance\pgf@x by -.5\wd\pgfnodeparttextbox%
    \advance\pgf@y by -.5\ht\pgfnodeparttextbox%
    \advance\pgf@y by +.5\dp\pgfnodeparttextbox%
  }
%\dname{}{$D_1$}
  % Define anchors for signal ports
  \anchor{D}{
    \pgf@process{\northeast}%
    \pgf@x=-1\pgf@x%
    \pgf@y=.5\pgf@y%
  }
  \anchor{CLK}{
    \pgf@process{\northeast}%
    \pgf@x=-1\pgf@x%
    \pgf@y=-.5\pgf@y%
  }
  \anchor{Q}{
    \pgf@process{\northeast}%
    \pgf@y=.5\pgf@y%
  }
  \anchor{Qn}{
    \pgf@process{\northeast}%
    \pgf@y=-.5\pgf@y%
  }
  % Draw the rectangle box and the port labels
  \backgroundpath{
    % Rectangle box
    \pgfpathrectanglecorners{\southwest}{\northeast}
    % Angle (>) for clock input
  

    % Draw port labels
    \begingroup
    \tikzset{flip flop/port labels} % Use font from this style
    \tikz@textfont

    \pgf@anchor@dff@D
    \pgftext[left,base,at={\pgfpoint{\pgf@x}{\pgf@y}},x=\pgfshapeinnerxsep]{\raisebox{-0.75ex}{\dname}}
    
    \pgf@anchor@dff@CLK
    \pgftext[left,base,at={\pgfpoint{\pgf@x}{\pgf@y}},x=\pgfshapeinnerxsep]{\raisebox{-0.65ex}{ \cname}}

    \pgf@anchor@dff@Q
    \pgftext[right,base,at={\pgfpoint{\pgf@x}{\pgf@y}},x=-\pgfshapeinnerxsep]{\raisebox{-.75ex}{\qname}}
   \pgf@anchor@dff@Q
 
    \pgf@anchor@dff@Qn
    \pgftext[right,base,at={\pgfpoint{\pgf@x}{\pgf@y}},x=-\pgfshapeinnerxsep]{\raisebox{-.75ex}{$\overline{}$}}

    \endgroup
  }
}

% Key to add font macros to the current font
\tikzset{add font/.code={\expandafter\def\expandafter\tikz@textfont\expandafter{\tikz@textfont#1}}}

% Define default style for this node
\tikzset{flip flop/port labels/.style={font=\sffamily\scriptsize}}
\tikzset{every dff node/.style={draw,minimum width=3.32cm,minimum
    height=3.828427125cm,inner sep=1mm,outer sep=0pt,cap=round,add
    font=\sffamily}}

\makeatother